%%%%%%%%%%%%%%%%%%%%%%%%%%%%%%%%%%%%%%%%%%%%%%%%%%%%%%%%%%%%%%%%%%%%%%%%%%%%%%
% CS110: Introduction to Computing
% Copyright 2015 Pejman Ghorbanzade <mail@ghorbanzade.com>
% Creative Commons Attribution-ShareAlike 4.0 International License
% https://github.com/ghorbanzade/UMB-CS110-2015S/blob/master/LICENSE
%%%%%%%%%%%%%%%%%%%%%%%%%%%%%%%%%%%%%%%%%%%%%%%%%%%%%%%%%%%%%%%%%%%%%%%%%%%%%%

\def \topDirectory {.}
\def \texDirectory {\topDirectory/src/main/tex}

\documentclass[12pt,letterpaper,twoside]{article}
\usepackage{\texDirectory/template/style/directives}
\usepackage{\texDirectory/template/style/assignment}
%%%%%%%%%%%%%%%%%%%%%%%%%%%%%%%%%%%%%%%%%%%%%%%%%%%%%%%%%%%%%%%%%%%%%%%%%%%%%%
% CS110: Introduction to Computing
% Copyright 2015 Pejman Ghorbanzade <mail@ghorbanzade.com>
% Creative Commons Attribution-ShareAlike 4.0 International License
% https://github.com/ghorbanzade/UMB-CS110-2015S/blob/master/LICENSE
%%%%%%%%%%%%%%%%%%%%%%%%%%%%%%%%%%%%%%%%%%%%%%%%%%%%%%%%%%%%%%%%%%%%%%%%%%%%%%

\course{id}{CS110}
\course{name}{Introduction to Computing}
\course{venue}{Tue/Thu, 5:30 PM - 6:45 PM}
\course{semester}{Spring 2015}
\course{department}{Department of Computer Science}
\course{university}{University of Massachusetts Boston}

\instructor{name}{Pejman Ghorbanzade}
\instructor{title}{}
\instructor{position}{Student Instructor}
\instructor{email}{pejman@cs.umb.edu}
\instructor{phone}{617-287-6419}
\instructor{office}{S-3-124B}
\instructor{office-hours}{Tue/Thu 19:00-20:30}
\instructor{address}{University of Massachusetts Boston, 100 Morrissey Blvd., Boston, MA}


\begin{document}

\doc{title}{Solution to Assignment 2}
\doc{date-pub}{Feb 17, 2015 at 5:30 PM}
\doc{date-due}{Mar 03, 2015 at 5:30 PM}
\doc{points}{8}

\prepare{header}

\section*{Question 1}

Write a program \texttt{ShapeDiamond.java} that asks for an integer number \texttt{n} and using \texttt{*} characters, prints a diamond in \texttt{2n-1} rows, such that there is one \texttt{*} character in first row, three \texttt{*} characters in second row and $2n-1$ characters in $n$th row.
Figure \ref{fig1} shows a sample diamond created when user has entered 4 as input integer.
\begin{verbbox}
   *
  ***
 *****
*******
 *****
  ***
   *
\end{verbbox}
\begin{figure}[H]\centering
\theverbbox
\caption{Sample diamond generated in 7 rows} \label{fig1}
\end{figure}

\subsection*{Solution}

\lstset{language=Java}
\lstset{tabsize=2}
\begin{lstlisting}
import java.util.Scanner;
public class ShapeDiamond {
	public static void main (String[] args) {
		Scanner input = new Scanner(System.in);
		// ask user for row number
		System.out.print("Enter number of rows: ");
		int numRows = input.nextInt();
		// close the Scanner as we no longer need it
		input.close();
		int i; // outer loop counter
		int j; // inner loop counter
		for (i=1 ; i <= 2*numRows-1; i++) {
			if (i <= numRows) {
				for (j = numRows; j > i; j--)
					System.out.print(" ");
				for (j = 0; j < 2*i-1; j++)
					System.out.print("*");
				System.out.println("");
			}
			else {
				for (j = 0; j < i-numRows; j++)
					System.out.print(" ");
				for (j = 0; j < 2*(2*numRows-i)-1 ; j++)
					System.out.print("*");
				System.out.println("");
			}
		}
	}
}
\end{lstlisting}

\section*{Question 2}

Remember the Dalton Brothers? Neglecting their original heights, write a program \texttt{Daltons2.java} that asks for heights of Joe, William, Jack and Averell, as well as the order in which they have to be sorted in terms of height. The order is either \texttt{ASC} for shortest to tallest or \texttt{DESC} for tallest to shortest. The program should give names of the brothers in order of their heights as specified.

\subsection*{Solution}

\begin{lstlisting}
import java.util.Scanner;
public class Daltons2 {
	public static void main(String[] args) {
		int i; // loop counter
		int j; // loop counter
		// sorting order: true for ASC, false for DESC
		boolean orderAsc;
		// String array for names of Dalton Brothers
		String[] names = {"Joe", "William", "Jack", "Averell"};
		// Double array for heights of Dalton brothers
		// length of heights[] is determined by number of brothers
		double[] heights = new double[names.length];
		// declaring a Scanner object to get inputs from user
		Scanner input = new Scanner(System.in);

		// Getting height of each brother
		for (i = 0; i < names.length; i++) {
			System.out.print("What is the height of "+names[i]+"? ");
			heights[i] = input.nextDouble();
		}
		// Get order from user
		// a while(true) loop is used to ensure user inserts desired input
		// (This is not the best method to solve this problem, as will be discussed later in class)
		while (true) {
			System.out.print("In which order do you want to sort the Daltons [ASC, DESC]? ");
			// Get a String from user
			String orderName = input.next();
			// check if value of given String matches 'ASC'
			if (orderName.equals("ASC")) {
				orderAsc = true;
				// end while(true) loop
				break;
			}
			// check if value of given String matches 'DESC'
			else if (orderName.equals("DESC")) {
				orderAsc = false;
				// end while(true) loop
				break;
			}
			// if value of String is neither 'ASC' nor 'DESC'
			// prompt user that his input is unacceptable
			// note that the while(true) loop continues after this condition
			else {
				System.out.println("Invalid input.");
			}
		}
		// Close Scanner object as we don't need it anymore
		input.close();
		// if we wanted to show heights we could do it in one line: Arrays.sort(heights)
		// since we are asked to show names sorted based on heights
		// we'll find indices of the sorted heights array.
		double min;
		int[] indices = new int[heights.length];
		boolean[] counted = new boolean[heights.length];
		for (i = 0; i < heights.length; i++)
			counted[i] = false;
		for (i = 0; i < heights.length; i++) {
			min = Double.MAX_VALUE;
			for (j = 0; j < heights.length; j++) {
				if (!counted[j] && heights[j] <= min) {
					min = heights[j];
					indices[i] = j;
				}
			}
			counted[indices[i]] = true;
		}
		// now indices are found, show names based on indices
		for (i = 0; i < heights.length; i++) {
			System.out.println(orderAsc ? names[indices[i]] : names[indices[3-i]]);
		}
	}
}
\end{lstlisting}

\section*{Question 3}

Minnie and Mannie are students of mathematics. Their area of interest is prime numbers. They also work part time in a restaurant. To fight the extreme boredom of their routine tasks at work, they have invented a little somewhat-challenging game between themselves.

Minnie, the waitress, has numbered each item in the menu with a unique \textbf{random} prime number. When Minnie takes orders of a table, she will multiply the menu numbers of all orders placed by that table and gives the final multiplication to Mannie, the chef.

Thanks to his brilliant mind, Mannie will in less than two seconds decompose the given number to its prime factors and prepares the menu numbers corresponding to those factors.

Unfortunately, Mannie is absent this week and you have to replace him. As it is likely that you cannot decompose a number as fast as Mannie, you are requested to write a program \texttt{PrimeMenu.java} that asks for an integer number and prints all prime factors that number is divisible by. Obviously, your program should also indicate how many times the number is divisible by each prime number.

\subsection*{Solution}

\begin{lstlisting}
import java.util.Scanner;
public class PrimeMenu {
	public static void main(String[] args) {
		int i; // loop counter
		int j; // loop counter
		int[] primeList = new int[100000]; // array of primes
		int counter = 0; // counter for array of primes
		boolean definitelyNotPrime; // flag for prime checking
		Scanner input = new Scanner(System.in);

		System.out.print("What is the number? ");
		// ask Minnie the number
		int number = input.nextInt();
		// close scanner since we don't need it anymore
		input.close();
		// Now find all prime numbers less than or equal to half that number
		System.out.print("Calculating...");
		for (i = 2; i <= number/2; i++) {
			// check if a number is prime
			// a number is prime if is not divisible by any smaller prime number
			// note that we can't afford to check if any smaller number divides our number
			definitelyNotPrime = false;
			for (j = 0; j < counter; j++) {
				if (i % primeList[j] == 0) {
					definitelyNotPrime = true;
				}
			}
			if (!definitelyNotPrime) {
				primeList[counter++] = i;
			}
		}
		System.out.println(" OK.");
		System.out.println("Prime\tNumber");
		// Now we know all prime lists to check for
		for (i = 0; i < counter; i++) {
			j = 0;
			while ( number % primeList[i] == 0) {
				number /= primeList[i];
				j++;
			}
			if (j != 0) {
				System.out.println(primeList[i]+"\t"+j);
			}
		}
	}
}
\end{lstlisting}

\end{document}
