%%%%%%%%%%%%%%%%%%%%%%%%%%%%%%%%%%%%%%%%%%%%%%%%%%%%%%%%%%%%%%%%%%%%%%
% UMB-CS110-2015S: Introduction to Computing
% Copyright 2015 Pejman Ghorbanzade <pejman@ghorbanzade.com>
% Creative Commons Attribution-ShareAlike 4.0 International License
% More info: https://github.com/ghorbanzade/UMB-CS110-2015S
%%%%%%%%%%%%%%%%%%%%%%%%%%%%%%%%%%%%%%%%%%%%%%%%%%%%%%%%%%%%%%%%%%%%%%

\def \topDirectory {.}
\def \texDirectory {\topDirectory/src/main/tex}

\documentclass[12pt,letterpaper,twoside]{article}
\usepackage{\texDirectory/template/style/directives}
\usepackage{\texDirectory/template/style/assignment}
%%%%%%%%%%%%%%%%%%%%%%%%%%%%%%%%%%%%%%%%%%%%%%%%%%%%%%%%%%%%%%%%%%%%%%%%%%%%%%
% CS110: Introduction to Computing
% Copyright 2015 Pejman Ghorbanzade <mail@ghorbanzade.com>
% Creative Commons Attribution-ShareAlike 4.0 International License
% https://github.com/ghorbanzade/UMB-CS110-2015S/blob/master/LICENSE
%%%%%%%%%%%%%%%%%%%%%%%%%%%%%%%%%%%%%%%%%%%%%%%%%%%%%%%%%%%%%%%%%%%%%%%%%%%%%%

\course{id}{CS110}
\course{name}{Introduction to Computing}
\course{venue}{Tue/Thu, 5:30 PM - 6:45 PM}
\course{semester}{Spring 2015}
\course{department}{Department of Computer Science}
\course{university}{University of Massachusetts Boston}

\instructor{name}{Pejman Ghorbanzade}
\instructor{title}{}
\instructor{position}{Student Instructor}
\instructor{email}{pejman@cs.umb.edu}
\instructor{phone}{617-287-6419}
\instructor{office}{S-3-124B}
\instructor{office-hours}{Tue/Thu 19:00-20:30}
\instructor{address}{University of Massachusetts Boston, 100 Morrissey Blvd., Boston, MA}


\begin{document}

\doc{title}{Solution to Quiz 3(b)}
\doc{date-pub}{Apr 02, 2015 at 01:00 PM}
\doc{date-due}{Apr 03, 2015 at 11:00 PM}
\doc{points}{4}

\prepare{header}

\section*{Question 1}

The code given below is the main method of a program \texttt{EasyMatrix.java} that fills a two-dimensional array of 10 columns and 10 rows with distinct random integers in range 0 to 99. With no modification to the main method, develop the rest of this program.

\lstset{language=java}
\begin{lstlisting}
public static void main(String[] args) {
	int matrixSize = 10;
	int[] array = arrayInitialize(matrixSize);
	// shuffle array
	arrayShuffle(array);
	// convert shuffled array to square matrix
	int[][] matrix = toMatrix(array);
	// show matrix
	showMatrix(matrix);
}
\end{lstlisting}

\subsection*{Solution}

\lstset{language=java}
\begin{lstlisting}
public class EasyMatrix {
	public static int[] arrayInitialize(int matrixSize) {
		int[] array = new int[matrixSize * matrixSize];
		for (int i = 0; i < array.length; i++)
			array[i] = i;
		return array;
	}
	public static void arrayShuffle(int[] array) {
		for (int i = 0; i < array.length; i++) {
			int j = (int) (i + (array.length-i)*Math.random());
			int temp = array[j];
			array[j] = array[i];
			array[i] = temp;
		}
	}
	public static void showMatrix(int[][] matrix) {
		int row = matrix.length;
		int column = matrix[0].length;
		for (int i = 0; i < row; i++) {
			for (int j = 0; j < column; j++)
				System.out.printf("%2d ", matrix[i][j]);
			System.out.println();
		}
	}
	public static int[][] toMatrix(int[] array) {
		int num = (int) Math.sqrt(array.length);
		int[][] matrix = new int[num][num];
		for (int i = 0; i < array.length; i++) {
			matrix[i/num][i%num] = array[i];
		}
		return matrix;
	}
	public static void main(String[] args) {
		int matrixSize = 10;
		int[] array = arrayInitialize(matrixSize);
		// shuffle array
		arrayShuffle(array);
		// convert shuffled array to square matrix
		int[][] matrix = toMatrix(array);
		// show matrix
		showMatrix(matrix);
	}
}
\end{lstlisting}

\section*{Question 2}

Write a class \texttt{UserProfile.java} with username, password, name, height, weight and blood type as attributes.

Develop a program \texttt{MyProfile.java} with the following main method that uses \texttt{UserProfile} class to perform as described below.

Upon execution, the user is brought into the main menu where he can select one of the following options.
\begin{enumerate}[itemsep=0pt]
\parskip=0pt \parsep=0pt
\item Create new profile
\item Show my profile
\item Exit
\end{enumerate}

\lstset{language=java}
\begin{lstlisting}
public static void main(String[] args) {
	while (true) {
		int option = getMenuOption();
		if (option == 1)
			createProfile();
		else if (option == 2)
			showProfile();
		else
			break;
	}
	System.out.println("Goodbye!");
}
\end{lstlisting}

When creating a new account, user first enters username and password. After creating his profile, user is asked to enter his name, height, weight and bloodtype. Upon initialization of all profile attributes, program should return to the main menu.

To show a user profile, program will ask for a username and a password. If credentials are correct, program should prompt the information for that username. In any other case it would return to the main menu.

For simplicity, assume there are no more than 100 profiles at one run time.

\subsection*{Solution}

\begin{enumerate}
\item \texttt{UserProfile.java}
\lstset{language=java}
\begin{lstlisting}
public class UserProfile {
	private static int numProfiles = 0;
	// data members
	private String username;
	private String password;
	private String name;
	private double height;
	private double weight;
	private String bloodType;
	// constructor
	public UserProfile(String user, String pass) {
		this.username = user;
		this.password = pass;
		numProfiles ++;
	}
	// accessors and mutators
	public static int getNumProfiles() {
		return numProfiles;
	}
	public String getUsername() {
		return username;
	}
	public void setUsername(String username) {
		this.username = username;
	}
	public String getPassword() {
		return password;
	}
	public void setPassword(String password) {
		this.password = password;
	}
	public String getName() {
		return name;
	}
	public void setName(String name) {
		this.name = name;
	}
	public double getHeight() {
		return height;
	}
	public void setHeight(double height) {
		this.height = height;
	}
	public double getWeight() {
		return weight;
	}
	public void setWeight(double weight) {
		this.weight = weight;
	}
	public String getBloodType() {
		return bloodType;
	}
	public void setBloodType(String bloodType) {
		this.bloodType = bloodType;
	}
}
\end{lstlisting}
\item \texttt{MyProfile.java}
\lstset{language=java}
\begin{lstlisting}
public class MyProfile {
	private static UserProfile[] profiles = new UserProfile[100];
	private static Scanner input = new Scanner(System.in);
	public static void main(String[] args) {
		while (true) {
			int option = getMenuOption();
			if (option == 1)
				createProfile();
			else if (option == 2)
				showProfile();
			else
				break;
		}
		System.out.println("Goodbye!");
		// end of main method
	}
	public static int getMenuOption() {
		while (true) {
			System.out.println();
			System.out.println("You are in the main menu.");
			System.out.println("  1. Create new profile");
			System.out.println("  2. Show my profile");
			System.out.println("  3. Exit");
			System.out.print("Select your option [1-3] ");
			int option = input.nextInt();
			if (0 < option && option < 4)
				return option;
			System.out.println("Wrong Input.");
		}
	}
	public static void showProfile() {
		if (UserProfile.getNumProfiles() != 0) {
			System.out.print("Enter your username: ");
			String user = input.next();
			System.out.print("Enter your password: ");
			String pass = input.next();
			for (int i = 0; i < UserProfile.getNumProfiles(); i++) {
				if (profiles[i].getUsername().equals(user)) {
					if (profiles[i].getPassword().equals(pass)) {
						showInfo(profiles[i]);
					}
					else
						System.out.println("Password mismatch.");
				}
				else
					System.out.println("No such profile found.");
			}
		}
		else
			System.out.println("No profiles yet.");
	}
	public static void showInfo(UserProfile myProfile) {
		System.out.println("Hello " + myProfile.getName() + "!");
		System.out.print("Height: ");
		System.out.println(myProfile.getHeight());
		System.out.print("Weight: ");
		System.out.println(myProfile.getWeight());
		System.out.print("Blood Type: ");
		System.out.println(myProfile.getBloodType());
	}
	public static void createProfile() {
		if (UserProfile.getNumProfiles() != 100) {
			// get user and password
			System.out.print("Enter your username: ");
			String user = input.next();
			//input.nextLine();
			System.out.print("Enter your password: ");
			String pass = input.next();
			//input.nextLine();
			// create user profile
			int num = UserProfile.getNumProfiles();
			profiles[num] = new UserProfile(user, pass);
			System.out.println("You're accout is created.");
			// initialize name
			System.out.print("Enter your name: ");
			//input.nextLine();
			String name = input.next();
			input.nextLine();
			profiles[num].setName(name);
			// initialize height
			System.out.print("Enter your height: ");
			double height = input.nextDouble();
			profiles[num].setHeight(height);
			// initialize weight
			System.out.print("Enter your weight: ");
			double weight = input.nextDouble();
			profiles[num].setWeight(weight);
			// initialize blood type
			System.out.print("Enter your blood type: ");
			String bloodtype = input.next();
			profiles[num].setBloodType(bloodtype);
		}
	}
}
\end{lstlisting}
\end{enumerate}

\end{document}
