%%%%%%%%%%%%%%%%%%%%%%%%%%%%%%%%%%%%%%%%%%%%%%%%%%%%%%%%%%%%%%%%%%%%%%
% UMB-CS110-2015S: Introduction to Computing
% Copyright 2015 Pejman Ghorbanzade <pejman@ghorbanzade.com>
% Creative Commons Attribution-ShareAlike 4.0 International License
% More info: https://github.com/ghorbanzade/UMB-CS110-2015S
%%%%%%%%%%%%%%%%%%%%%%%%%%%%%%%%%%%%%%%%%%%%%%%%%%%%%%%%%%%%%%%%%%%%%%

\def \topDirectory {.}
\def \texDirectory {\topDirectory/src/main/tex}

\documentclass[12pt,letterpaper,twoside]{article}
\usepackage{\texDirectory/template/style/directives}
\usepackage{\texDirectory/template/style/assignment}
%%%%%%%%%%%%%%%%%%%%%%%%%%%%%%%%%%%%%%%%%%%%%%%%%%%%%%%%%%%%%%%%%%%%%%%%%%%%%%
% CS110: Introduction to Computing
% Copyright 2015 Pejman Ghorbanzade <mail@ghorbanzade.com>
% Creative Commons Attribution-ShareAlike 4.0 International License
% https://github.com/ghorbanzade/UMB-CS110-2015S/blob/master/LICENSE
%%%%%%%%%%%%%%%%%%%%%%%%%%%%%%%%%%%%%%%%%%%%%%%%%%%%%%%%%%%%%%%%%%%%%%%%%%%%%%

\course{id}{CS110}
\course{name}{Introduction to Computing}
\course{venue}{Tue/Thu, 5:30 PM - 6:45 PM}
\course{semester}{Spring 2015}
\course{department}{Department of Computer Science}
\course{university}{University of Massachusetts Boston}

\instructor{name}{Pejman Ghorbanzade}
\instructor{title}{}
\instructor{position}{Student Instructor}
\instructor{email}{pejman@cs.umb.edu}
\instructor{phone}{617-287-6419}
\instructor{office}{S-3-124B}
\instructor{office-hours}{Tue/Thu 19:00-20:30}
\instructor{address}{University of Massachusetts Boston, 100 Morrissey Blvd., Boston, MA}


\begin{document}

\doc{title}{Solution to Quiz 4}
\doc{date-pub}{Apr 21, 2015 at 01:00 PM}
\doc{date-due}{Apr 24, 2015 at 11:00 PM}
\doc{points}{4}

\prepare{header}

\section*{Question 1}

Write a program \texttt{AutoShow.java} that can control different automobiles. All cars have some seating capacity. As expected from any vehicle, they move provided that at least one person is in the car, the engine is on and the fuel tank is not empty. For safety reasons, no one can enter or exit the car when the engine is on.

Based on specified conditions, develop a class \texttt{Car.java} and instantiate two cars inside \texttt{AutoShow.java} program where \texttt{car1} has a tank size of 100 miles and can seat up to four people while \texttt{car2} has a tank size of 150 miles and can only seat up to 2 people. Move 200 miles with each car. Make sure your program will print an output for all methods regardless of the order in which they are invoked.

\section*{Solution}

\begin{enumerate}[label=(\alph*)]
\item Class \texttt{Car.java}
\lstset{language=java, tabsize=2}
\begin{lstlisting}
public class Car {
	// Attributes
	private int maxSeat;
	private int numSeated;
	private boolean engineOn;
	private double tankCapacity; // per mile
	private double tank; // per mile
	// Constructor
	public Car(int maxSeat, double tankCapacity) {
		this.maxSeat = maxSeat;
		this.tankCapacity = tankCapacity;
		engineOn = false;
		numSeated = 0;
		tank = tankCapacity;
	}
	// Methods
	public void refuel() {
		this.tank = tankCapacity;
		System.out.println("Tank refilled!");
	}
	public void switchEngine() {
		if (numSeated > 0) {
			this.engineOn = !engineOn;
			System.out.println("Engine turned "+((engineOn)?"on":"off")+".");
		}
		else {
			System.out.println("No one in the car to switch engine!");
		}
	}
	public void move(double miles) {
		if (engineOn) {
			if (tank == 0) {
				System.out.println("Tank empty!");
			}
			else {
				double distance = Math.min(this.tank, miles);
				this.tank -= distance;
				System.out.printf("Moved %.2f miles!\n", distance);
			}
		}
		else {
			System.out.printf("Engine off!\n");
		}
	}
	public void board(int num) {
		if (!engineOn) {
			if (numSeated + num <= maxSeat) {
				numSeated += num;
				System.out.println(numSeated + " people on board!");
			}
			else {
				System.out.printf("Cannot exceed max seating capacity!\n");
			}
		}
		else {
			System.out.println("Engine must be off before we can board!\n");
		}
	}
	public void getOff(int num) {
		if (!engineOn) {
			if (num <= numSeated) {
				numSeated -= num;
				System.out.println(num + " people got off the car.");
			}
			else {
				System.out.printf("There are only %d people in the car!\n", numSeated);
			}
		}
		else {
			System.out.println("Engine must be off before we can get off!");
		}
	}
	// Getters and Setters
	public boolean isEngineOn() {
		return engineOn;
	}
	public int getNumSeated() {
		return numSeated;
	}
	public double getFuel() {
		return tank;
	}
}
\end{lstlisting}

\item Class \texttt{AutoShow.java}
\lstset{language=java, tabsize=2}
\begin{lstlisting}
public class AutoShow {
	public static void main(String[] args) {
		// first instantiate two cars
		Car car1 = new Car(4, 100);
		// for the race car we cannot just move since engine is off
		car1.move(10);
		// we cannot just start engine since no one is in the car yet
		car1.switchEngine();
		// if at least one goes in the car
		car1.board(1);
		// but we still cannot move yet
		car1.move(10);
		// we will then be able to start engine
		car1.switchEngine();;
		// we can then move
		car1.move(10);
		// but if we move for too long, we'll run out of fuel
		car1.move(100);
		// and we cannot move anymore since tank is empty
		car1.move(10);
		// so we have to refuel
		car1.refuel();
		// and then we can move again
		car1.move(100);
		// now we moved 200 miles, but we cannot just get out
		car1.getOff(1);
		// since we should turn the engine off first
		car1.switchEngine();
		// and we can get off now
		car1.getOff(1);
		Car car2 = new Car(2, 150);
		// and the control procedure is just the same
		car2.board(1);
		car2.switchEngine();
		car2.move(150);
		car2.refuel();
		car2.move(50);
		car2.switchEngine();
		car2.getOff(1);
	}
}
\end{lstlisting}

\end{enumerate}

\end{document}
