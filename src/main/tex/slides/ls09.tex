%%%%%%%%%%%%%%%%%%%%%%%%%%%%%%%%%%%%%%%%%%%%%%%%%%%%%%%%%%%%%%%%%%%%%%
% UMB-CS110-2015S: Introduction to Computing
% Copyright 2015 Pejman Ghorbanzade <pejman@ghorbanzade.com>
% Creative Commons Attribution-ShareAlike 4.0 International License
% More info: https://github.com/ghorbanzade/UMB-CS110-2015S
%%%%%%%%%%%%%%%%%%%%%%%%%%%%%%%%%%%%%%%%%%%%%%%%%%%%%%%%%%%%%%%%%%%%%%

\def \topDirectory {../..}
\def \texDirectory {\topDirectory/src/main/tex}

\documentclass[10pt, compress]{beamer}

\usepackage{\texDirectory/template/style/directives}
%%%%%%%%%%%%%%%%%%%%%%%%%%%%%%%%%%%%%%%%%%%%%%%%%%%%%%%%%%%%%%%%%%%%%%%%%%%%%%
% CS110: Introduction to Computing
% Copyright 2015 Pejman Ghorbanzade <mail@ghorbanzade.com>
% Creative Commons Attribution-ShareAlike 4.0 International License
% https://github.com/ghorbanzade/UMB-CS110-2015S/blob/master/LICENSE
%%%%%%%%%%%%%%%%%%%%%%%%%%%%%%%%%%%%%%%%%%%%%%%%%%%%%%%%%%%%%%%%%%%%%%%%%%%%%%

\course{id}{CS110}
\course{name}{Introduction to Computing}
\course{venue}{Tue/Thu, 5:30 PM - 6:45 PM}
\course{semester}{Spring 2015}
\course{department}{Department of Computer Science}
\course{university}{University of Massachusetts Boston}

\instructor{name}{Pejman Ghorbanzade}
\instructor{title}{}
\instructor{position}{Student Instructor}
\instructor{email}{pejman@cs.umb.edu}
\instructor{phone}{617-287-6419}
\instructor{office}{S-3-124B}
\instructor{office-hours}{Tue/Thu 19:00-20:30}
\instructor{address}{University of Massachusetts Boston, 100 Morrissey Blvd., Boston, MA}

\usepackage{\texDirectory/template/style/beamerthemeUmassLecture}
\doc{number}{9}
%\setbeamertemplate{footline}[text line]{}

\begin{document}
\prepareCover

\section{Course Administration}

\begin{frame}[fragile]
\frametitle{Course Administration}
Assignment 3 to due April 02 at 5:30 PM.

Final Exam to be held May 19 at 6:30 PM.
\end{frame}

\begin{frame}[fragile]
	\frametitle{Overview}
	\begin{itemize}
		\item[] Constructors
		\item[] Access Modifiers
		\item[] Encapsulation
	\end{itemize}
\end{frame}

\section{Constructors}

\begin{frame}[fragile]
	\frametitle{Constructors}
	\begin{block}{Objective}
		Write a program \texttt{Cat1Dtest.java} that controls position of a cat named kitty on the x-axis.
	\end{block}
\end{frame}

\begin{frame}[fragile]
	\frametitle{Constructors}
	\begin{block}{\texttt{Cat.java} (v1.3)}
		\begin{minted}[fontsize=\small,tabsize=8, linenos, firstnumber=1]{java}
public class Cat {
	// class variables (fields):
	public static int numCats = 0;
	// instance variables (states):
	public String name;
	public double position;
	// methods:
	public void move(int dist) {
		this.position += dist;
	}
}
		\end{minted}
	\end{block}
\end{frame}

\begin{frame}[fragile]
	\frametitle{Constructors}
	\begin{block}{Instantiation}
	Once we define our blueprints, we need to instantiate objects from the class.
	\end{block}
	\begin{block}{\texttt{Cat1Dtest.java} (v1.0)}
		\begin{minted}[fontsize=\small,tabsize=8, linenos, firstnumber=1]{java}
public class Cat1Dtest {
	public static void main(String[] args) {
		Cat mycat = new Cat();
		// rest of the code
	}
}
		\end{minted}
	\end{block}
\end{frame}

\begin{frame}[fragile]
	\frametitle{Constructors}
	\begin{block}{Closer Look}
		\begin{minted}[fontsize=\small,tabsize=8, linenos, firstnumber=3]{java}
		Cat mycat = new Cat();
		\end{minted}
		\begin{description}
		\item[Cat] is a reference data type.
		\item[mycat] is the identifier for our object.
		\item[new] is the keyword to instantiate an object from class \texttt{Cat}, allocating memory for the object.
		\item[Cat()] is a constructor to initialize states of the object.
		\end{description}
	\end{block}
\end{frame}

\begin{frame}[fragile]
	\frametitle{Constructors}
	\begin{block}{Problem Statement}
		What is the name and position of \texttt{mycat} after it is created?

		Upon instantiation, values of attributes of the object are not initialized.
	\end{block}
	\begin{block}{Possible Solution}
		Initialize states of the object after instantiation.
	\end{block}
\end{frame}

\begin{frame}[fragile]
	\frametitle{Constructors}
	\begin{block}{\texttt{Cat1Dtest.java} (v1.1)}
		\begin{minted}[fontsize=\small,tabsize=8, linenos, firstnumber=1]{java}
public class Cat1Dtest {
	public static void main(String[] args) {
		Cat myCat = new Cat();
		myCat.position = 0;
		myCat.name = "Kitty";
	}
}
		\end{minted}
	\end{block}
\end{frame}

\begin{frame}[fragile]
	\frametitle{Constructors}
	\begin{block}{How do we think?}
		\begin{quote}
		Aliens visited the International Space Station last night and drank a cup of tea with Scott Kelly.
		\end{quote}
	\end{block}
	\begin{block}{Problem Statement}
		Initializing states of an object after instantiation is inconsistent with how we think.
	\end{block}
\end{frame}

\begin{frame}[fragile]
	\frametitle{Constructors}
	\begin{block}{Definition}
		A constructor is a block of code that is automatically invoked upon instantiation.
	\end{block}
	\begin{block}{Closer Look}
		\begin{minted}[fontsize=\small,tabsize=8, linenos, firstnumber=3]{java}
		Cat mycat = new Cat();
		\end{minted}
		If no constructor is declared in a class, JVM will append a default no-argument constructor of the following form and will invoke it each time an object is instantiated from that class.
		\begin{minted}[fontsize=\small,tabsize=8, linenos, firstnumber=11]{java}
		public Cat() {
		}
		\end{minted}
	\end{block}
\end{frame}

\begin{frame}[fragile]
	\frametitle{Constructors}
	\begin{block}{Syntax}
		\begin{minted}[fontsize=\small,tabsize=8, linenos, firstnumber=11]{java}
		public Cat(String catName, double catPosition) {
			name = catName;
			position = catPosition;
			numCats++;
			System.out.println("A new Cat is born!");
		}
		\end{minted}
	\end{block}
	\begin{block}{Structure}
	\begin{itemize}
		\item[] modifier
		\item[] \emph{parameter list}
		\item[] constructor body
	\end{itemize}
	\end{block}
\end{frame}

\begin{frame}[fragile]
	\frametitle{Constructors}
	\begin{block}{\texttt{Cat.java} (v1.4)}
		\begin{minted}[fontsize=\small,tabsize=8, linenos, firstnumber=1]{java}
public class Cat {
	public static int numCats = 0;
	public String name;
	public double position;
	public void move(int dist) {
		this.position += dist;
	}
	public Cat(String catName, double catPosition) {
		name = catName;
		position = catPosition;
		numCats++;
		System.out.println("A new Cat is born!");
	}
}
		\end{minted}
	\end{block}
\end{frame}

\begin{frame}[fragile]
	\frametitle{Constructors}
	\begin{block}{\texttt{Cat1Dtest.java} (v1.2)}
		\begin{minted}[fontsize=\small,tabsize=8, linenos, firstnumber=1]{java}
public class Cat1Dtest {
	public void showPosition(Cat theCat) {
		System.out.println(theCat.name +" is in "+ theCat.position);
	}
	public static void main(String[] args) {
		Cat myCat = new Cat("Kitty", 0);
		showPosition(Cat myCat);
		myCat.move(2);
		showPosition(Cat myCat);
	}
}
		\end{minted}
	\end{block}
\end{frame}

\begin{frame}[fragile]
	\frametitle{Constructors}
	\begin{block}{Signature}
		Java can distinguish between constructors with different signatures. Signature of a constructor is defined by its parameter list.
	\end{block}
	\begin{block}{Syntax}
		\begin{minted}[fontsize=\small,tabsize=8, linenos, firstnumber=11]{java}
		public Cat(String catName) {
			name = catName;
			position = 0;
			System.out.println("A new Cat is born!");
		}
		public Cat(String catName, double catPosition) {
			name = catName;
			position = catPosition;
			numCats++;
			System.out.println("A new Cat is born!");
		}
		\end{minted}
	\end{block}
\end{frame}

\section{Access Modifiers}

\begin{frame}[fragile]
	\frametitle{Access Modifers}
	\begin{block}{Definition}
		Access modifiers specify scope of classes or class members (data memebrs, methods, constructors). Access modifiers determine whether other classes can use a particular data member or invoke a particular method.
	\end{block}
	\begin{block}{Types}
		\begin{columns}
		\column{0.5\textwidth}
			\begin{itemize}
				\item[] public
				\item[] default (no modifier)
			\end{itemize}
		\column{0.5\textwidth}
			\begin{itemize}
			\item[] protected
			\item[] private
			\end{itemize}
		\end{columns}
	\end{block}
\end{frame}

\begin{frame}[fragile]
	\frametitle{Access Modifers}
	\begin{block}{\texttt{Cat.java} (v1.4) Revisited}
		\begin{minted}[fontsize=\small,tabsize=8, linenos, firstnumber=1]{java}
public class Cat {
	public static int numCats = 0;
	public String name;
	public double position;
	public void move(int dist) {
		this.position += dist;
	}
	public Cat(String catName, double catPosition) {
		name = catName;
		position = catPosition;
		numCats++;
		System.out.println("A new Cat is born!");
	}
}
		\end{minted}
	\end{block}
\end{frame}

\begin{frame}[fragile]
	\frametitle{Access Modifers}
	\begin{block}{\texttt{public} modifier}
		Classes or class members may be declared as \alert{public} in which case they are accessible to all classes everywhere.
	\end{block}
	\begin{block}{\texttt{default} modifier}
		Classes or class members may be declared with no explicit modifiers, in which case they are only accessible within their package.
	\end{block}
\end{frame}

\begin{frame}[fragile]
	\frametitle{Access Modifers}
	\begin{block}{\texttt{protected} modifier}
		Class members may be declared as \alert{protected} in which case they are accessible within their package.
	\end{block}
	\begin{block}{\texttt{private} modifier}
		Class members may be declared as \alert{private} in which case they are only accessible within their class.
	\end{block}
\end{frame}

\begin{frame}[fragile]
	\frametitle{Access Modifers}
	\begin{block}{Remember}
		\begin{itemize}
		\item[] Always use the most restrictive access level that makes sense for a particular member.
		\end{itemize}
	\end{block}
\end{frame}

\begin{frame}[fragile]
	\frametitle{Access Modifers}
	\begin{block}{\texttt{Cat.java} (v1.4) Revisited}
		\begin{minted}[fontsize=\small,tabsize=8, linenos, firstnumber=1]{java}
public class Cat {
	public static int numCats = 0;
	public String name;
	public double position;
	public void move(int dist) {
		this.position += dist;
	}
	public Cat(String catName, double catPosition) {
		name = catName;
		position = catPosition;
		numCats++;
		System.out.println("A new Cat is born!");
	}
}
		\end{minted}
	\end{block}
\end{frame}

\begin{frame}[fragile]
	\frametitle{Access Modifers}
	\begin{block}{Problem Statement}
		Class and class members of \texttt{Cat.java} (v1.4) all have public modifiers.
		This makes it possible for everyone to access and modify data in \texttt{Cat.java} class or data of its objects, as they please.
	\end{block}
	\begin{block}{\texttt{Evil.java} (v1.0)}
		\begin{minted}[fontsize=\small,tabsize=8, linenos, firstnumber=1]{java}
public class Evil {
	public static void harm(Cat someCat) {
		someCat.position = 500;
		someCat.name = "HACKED";
	}
}
		\end{minted}
	\end{block}
\end{frame}

\section{Encapsulation}

\begin{frame}[fragile]
	\frametitle{Encapsulation}
	\begin{block}{Objective}
		Write a program \texttt{Cat1Dtest.java} that controls position of a cat named kitty on the x-axis from -100 to 100.
	\end{block}
\end{frame}

\begin{frame}[fragile]
	\frametitle{Encapsulation}
	\begin{block}{\texttt{Evil.java} (v1.0) Revisited}
		\begin{minted}[fontsize=\small,tabsize=8, linenos, firstnumber=1]{java}
public class Evil {
	public static void harm(Cat someCat) {
		someCat.position = 500;
		someCat.name = "HACKED";
	}
}
		\end{minted}
	\end{block}
\end{frame}

\begin{frame}[fragile]
	\frametitle{Encapsulation}
	\begin{block}{Definition}
	Encapsulation is one of the fundamental principles of object oriented programming which promotes controlling data of an object by restricting its access to methods of its class.
	\end{block}
\end{frame}

\begin{frame}[fragile]
	\frametitle{Encapsulation}
	\begin{block}{Motivation}
	To encapsulate data attributes of an object are declared \textit{private} and methods are proposed to make object in control of how the outside world is allowed to use it.
	\end{block}
\end{frame}

\begin{frame}[fragile]
	\frametitle{Encapsulation}
	\begin{block}{Advantages}
	\begin{description}
	\item[Modularity] Objects are passed to classes instead of data.
	\item[Data Hiding] Details of internal implementation are not disclosed.
	\item[Code Re-Use] Once verified, a developed class can be trusted.
	\item[Pluggability] Problems will be specific to classes.
	\end{description}
	\end{block}
\end{frame}

\begin{frame}[fragile]
	\frametitle{Encapsulation}
	\begin{block}{Approach}
	\begin{enumerate}
	\item[] Use private access modifier for data members
	\item[] Develop accessors and mutators for data members
	\item[] Impose restrictions for accessors and mutators
	\item[] Develop additional methods for data modification if required
	\end{enumerate}
	\end{block}
\end{frame}

\begin{frame}[fragile]
	\frametitle{Encapsulation}
	\begin{block}{\texttt{Cat.java} (v1.5)}
		\begin{minted}[fontsize=\small,tabsize=8, linenos, firstnumber=1]{java}
public class Cat {
	private static int numCats = 0;
	private String name;
	private double position;
	// develop accessors
	// develop mutators
	public void move(int dist) {
		this.position += dist;
	}
	public Cat(String catName, double catPosition) {
		name = catName;
		position = catPosition;
		numCats++;
		System.out.println("A new Cat is born!");
	}
}
		\end{minted}
	\end{block}
\end{frame}

\begin{frame}[fragile]
	\frametitle{Encapsulation}
	\begin{block}{Accessors}
	Accessors (getters) return some property of an object.
		\begin{minted}[fontsize=\small,tabsize=8, linenos, firstnumber=5]{java}
	public double getPosition() {
		return position;
	}
	public String getName() {
		return name;
	}
		\end{minted}
	\end{block}
\end{frame}

\begin{frame}[fragile]
	\frametitle{Encapsulation}
	\begin{block}{Mutators}
	Mutators (setters) change some property of an object.
		\begin{minted}[fontsize=\small,tabsize=8, linenos, firstnumber=11]{java}
	public void setName(String name) {
		this.name = name;
	}
	public void setPosition(double position) {
		this.position = position;
	}
		\end{minted}
	\end{block}
\end{frame}

\begin{frame}[fragile]
	\frametitle{Encapsulation}
	\begin{block}{Naming}
	There is no limit in how you name your accessors and mutators. By convention however, we usually use \texttt{getPropertyName} and \texttt{setPropertyName} to name accessors and mutators respectively. For properties with \texttt{boolean} data type, \texttt{isPropertyName} is used by convention.
	\end{block}
\end{frame}

\begin{frame}[fragile]
	\frametitle{Encapsulation}
	\begin{block}{Remember}
	Accessors and mutators often are declared public and used to access the property outside the object.
	\end{block}
	\begin{block}{Immutable Properties}
	Mutators can be omitted to prohibit modifications of certain attributes.
	\end{block}
\end{frame}

\begin{frame}[fragile]
	\frametitle{Encapsulation}
	\begin{block}{\texttt{Cat.java} (v1.6) (Part 1)}
		\begin{minted}[fontsize=\small,tabsize=8, linenos, firstnumber=1]{java}
public class Cat {
	private static int numCats = 0;
	private String name;
	private double position;
	public double getPosition() {
		return position;
	}
	public String getName() {
		return name;
	}
	public static int getNumCats() {
		return numCats;
	}
		\end{minted}
	\end{block}
\end{frame}

\begin{frame}[fragile]
	\frametitle{Encapsulation}
	\begin{block}{\texttt{Cat.java} (v1.6) (Part 2)}
		\begin{minted}[fontsize=\small,tabsize=8, linenos, firstnumber=14]{java}
	public void setPosition(double position) {
		this.position = position;
	}
	public void move(int dist) {
		this.position += dist;
	}
	public Cat(String catName, double catPosition) {
		name = catName;
		position = catPosition;
		numCats++;
		System.out.println("A new Cat is born!");
	}
}
		\end{minted}
	\end{block}
\end{frame}

\begin{frame}[fragile]
	\frametitle{Encapsulation}
	\begin{block}{\texttt{Cat1Dtest.java} (v1.3)}
		\begin{minted}[fontsize=\small,tabsize=8, linenos, firstnumber=1]{java}
public class Cat1Dtest {
	public void showPosition(Cat theCat) {
		System.out.print(theCat.getName() +" is in ");
		System.out.println(theCat.getPosition());
	}
	public static void main(String[] args) {
		Cat myCat = new Cat("Kitty", 0);
		showPosition(Cat myCat);
		myCat.move(2);
		showPosition(Cat myCat);
	}
}
		\end{minted}
	\end{block}
\end{frame}

\plain{}{Keep Calm\\and\\Think Object-Oriented}

\end{document}
