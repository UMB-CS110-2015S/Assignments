%%%%%%%%%%%%%%%%%%%%%%%%%%%%%%%%%%%%%%%%%%%%%%%%%%%%%%%%%%%%%%%%%%%%%%
% UMB-CS110-2015S: Introduction to Computing
% Copyright 2015 Pejman Ghorbanzade <pejman@ghorbanzade.com>
% Creative Commons Attribution-ShareAlike 4.0 International License
% More info: https://github.com/ghorbanzade/UMB-CS110-2015S
%%%%%%%%%%%%%%%%%%%%%%%%%%%%%%%%%%%%%%%%%%%%%%%%%%%%%%%%%%%%%%%%%%%%%%

\def \topDirectory {../..}
\def \texDirectory {\topDirectory/src/main/tex}

\documentclass[10pt, compress]{beamer}

\usepackage{\texDirectory/template/style/directives}
%%%%%%%%%%%%%%%%%%%%%%%%%%%%%%%%%%%%%%%%%%%%%%%%%%%%%%%%%%%%%%%%%%%%%%%%%%%%%%
% CS110: Introduction to Computing
% Copyright 2015 Pejman Ghorbanzade <mail@ghorbanzade.com>
% Creative Commons Attribution-ShareAlike 4.0 International License
% https://github.com/ghorbanzade/UMB-CS110-2015S/blob/master/LICENSE
%%%%%%%%%%%%%%%%%%%%%%%%%%%%%%%%%%%%%%%%%%%%%%%%%%%%%%%%%%%%%%%%%%%%%%%%%%%%%%

\course{id}{CS110}
\course{name}{Introduction to Computing}
\course{venue}{Tue/Thu, 5:30 PM - 6:45 PM}
\course{semester}{Spring 2015}
\course{department}{Department of Computer Science}
\course{university}{University of Massachusetts Boston}

\instructor{name}{Pejman Ghorbanzade}
\instructor{title}{}
\instructor{position}{Student Instructor}
\instructor{email}{pejman@cs.umb.edu}
\instructor{phone}{617-287-6419}
\instructor{office}{S-3-124B}
\instructor{office-hours}{Tue/Thu 19:00-20:30}
\instructor{address}{University of Massachusetts Boston, 100 Morrissey Blvd., Boston, MA}

\usepackage{\texDirectory/template/style/beamerthemeUmassLecture}
\doc{number}{18}
%\setbeamertemplate{footline}[text line]{}

\begin{document}
\prepareCover

\section{Course Administration}

\begin{frame}[fragile]
\frametitle{Course Administration}
	\begin{block}{Announcements}
		\begin{itemize}
			\item[] Quiz 5 Released. Due on May 08, 2015 at 23:00 PM.
			\item[] Hands-On Programming Session on May 07, 2015. Attendance strongly encouraged.
			\item[] Student Evaluation on May 12, 2015. Attendance mandatory.
		\end{itemize}
	\end{block}
\end{frame}

\begin{frame}[fragile]
	\frametitle{Overview}
	\begin{itemize}
		\item[] Introduction to Data Structures
		\begin{itemize}
			\item[] Introduction
			\item[] Class LinkedList
			\item[] Class Queue
			\item[] Class Stack
		\end{itemize}
	\end{itemize}
\end{frame}

\plain{}{Introduction to\\ Data Structures}

\section{Introduction}

\begin{frame}[fragile]
	\frametitle{Intro to Data Structures}
	\begin{block}{Philosophy}
	Data structures help us organize large amount of data for efficient use.
	\end{block}
	\begin{block}{Data Types}
		\begin{itemize}
			\item[] Primitive Data Types
			\item[] Composite Data Types
			\item[] Abstract Data Types
		\end{itemize}
	\end{block}
\end{frame}

\begin{frame}[fragile]
	\frametitle{Intro to Data Structures}
	\begin{block}{Composite Data Types}
		Any data type which can be constructed using primitive data types and other composite data types. Data with composite type is often called a \texttt{record}. Sometimes referred as Plain Old Data Structures (PODS), using composite data types are discouraged in modern object-oriented programming languages.
	\end{block}
\end{frame}

\begin{frame}[fragile]
	\frametitle{Intro to Data Structures}
	\begin{block}{Abstract Data Types}
		Any group of data with specificly defined implementation can be constructed as an Abstract Data Type (ADT). ADTs separate specification and implementation to make programs easier to understand, easier to modify and easier to reuse. ADTs have key role in Object-Oriented Programming that will be discussed later.
	\end{block}
\end{frame}

\begin{frame}[fragile]
	\frametitle{Intro to Data Structures}
	\begin{block}{Key Abstract Data Types}
		\begin{columns}
			\begin{column}{0.5\textwidth}
				\begin{itemize}
					\item[] Container
					\item[] List
					\item[] Map
					\item[] Queue
				\end{itemize}
			\end{column}
			\begin{column}{0.5\textwidth}
				\begin{itemize}
					\item[] Stack
					\item[] Tree
					\item[] Graph
					\item[] Set
				\end{itemize}
			\end{column}
		\end{columns}
	\end{block}
\end{frame}

\section{LinkedLists}

\begin{frame}[fragile]
	\frametitle{LinkedLists}
	\begin{block}{Definition}
		Linked List is a limited access, sequential data structure that can contain arbitrary number of objects of similar data types.
	\end{block}
	\begin{block}{Recursive Definition}
		A linked list is a data structure that is either empty or it consists of a node pointing to a linked list.
	\end{block}
\end{frame}

\section{Queues}

\begin{frame}[fragile]
	\frametitle{Queues}
	\begin{block}{Definition}
		Queue is a \alert{first-in first-out}, limited access, sequential data structure that can contain objects of similar data types.
	\end{block}
	\begin{block}{Recursive Definition}
		A queue is a first-in first-out data structure that is either empty or it consists of a head and a queue.
	\end{block}
\end{frame}

\begin{frame}[fragile]
	\frametitle{Queues}
	\begin{block}{Abstract Methods}
		\begin{itemize}
			\item[] \texttt{public boolean add(E item) throws IllegalStateException;}
			\item[] \texttt{public E element() throws NoSuchElementException;}
			\item[] \texttt{public boolean offer(E item);}
			\item[] \texttt{pubilc E peek();}
			\item[] \texttt{public E poll();}
			\item[] \texttt{public E remove() throws NoSuchElementException;}
		\end{itemize}
	\end{block}
\end{frame}

\section{Stacks}

\begin{frame}[fragile]
	\frametitle{Stacks}
	\begin{block}{Definition}
		Stack is a \alert{last-in first-out}, limited access, sequential data structure that can contain objects of similar data types.
	\end{block}
	\begin{block}{Recursive Definition}
		A stack is a last-in first-out data structure that is either empty or it consists of a top and a stack.
	\end{block}
\end{frame}

\begin{frame}[fragile]
	\frametitle{Stacks}
	\begin{block}{Methods}
		\begin{itemize}
			\item[] \texttt{public E push(E item);}
			\item[] \texttt{public E pop() throws EmptyStackException;}
			\item[] \texttt{public E peek() throws EmptyStackException;}
			\item[] \texttt{pubilc int search(Object o);}
			\item[] \texttt{public boolean empty();}
		\end{itemize}
	\end{block}
\end{frame}

\begin{frame}[fragile]
	\frametitle{Stacks}
	\begin{block}{Objective}
		Write a program \texttt{Reverse.java} that asks a String from user and prints its reverse.
	\end{block}
\end{frame}

\begin{frame}[fragile]
	\frametitle{Stacks}
	\begin{block}{\texttt{Reverse.java} (v1.0)}
		\begin{minted}[fontsize=\small,linenos,firstnumber=1]{java}
import java.util.Scanner;
import java.util.Stack;
public class Reverse2 {
	public static void main(String[] args) {
		Scanner input = new Scanner(System.in);
		System.out.print("Enter Word: ");
		String word = input.nextLine();
		Stack<Character> myStack = new Stack<Character>();
		for (int i = 0; i < word.length(); i++)
			myStack.push(word.charAt(i));
		System.out.print("Reverse Word: ");
		for (int i = 0; i < word.length(); i++)
			System.out.print(myStack.pop());
		input.close();
	}
}
		\end{minted}
	\end{block}
\end{frame}

\begin{frame}[fragile]
	\frametitle{Stacks}
	\begin{block}{Remember}
		Using simpler, more efficient, built-in data structures to solve a problem are always preferred. Do not overuse advanced data structures.
	\end{block}
\end{frame}

\begin{frame}[fragile]
	\frametitle{Stacks}
	\begin{block}{\texttt{Reverse.java} (v2.0)}
		\begin{minted}[fontsize=\small,linenos,firstnumber=1]{java}
import java.util.Scanner;
public class Reverse1 {
	public static void main(String[] args) {
		Scanner input = new Scanner(System.in);
		System.out.print("Enter Word: ");
		String word = input.nextLine();
		System.out.print("Reverse Word: ");
		for (int i = word.length(); i > 0; i--)
			System.out.print(word.charAt(i-1));
		input.close();
	}
}
		\end{minted}
	\end{block}
\end{frame}

\plain{}{Keep Calm\\and\\Enjoy Programming}

\end{document}
