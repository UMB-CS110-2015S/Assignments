%%%%%%%%%%%%%%%%%%%%%%%%%%%%%%%%%%%%%%%%%%%%%%%%%%%%%%%%%%%%%%%%%%%%%%%%%%%%%%
% CS110: Introduction to Computing
% Copyright 2015 Pejman Ghorbanzade <mail@ghorbanzade.com>
% Creative Commons Attribution-ShareAlike 4.0 International License
% https://github.com/ghorbanzade/UMB-CS110-2015S/blob/master/LICENSE
%%%%%%%%%%%%%%%%%%%%%%%%%%%%%%%%%%%%%%%%%%%%%%%%%%%%%%%%%%%%%%%%%%%%%%%%%%%%%%

\def \topDirectory {../..}
\def \texDirectory {\topDirectory/src/main/tex}

\documentclass[10pt, compress]{beamer}

\usepackage{\texDirectory/template/style/directives}
%%%%%%%%%%%%%%%%%%%%%%%%%%%%%%%%%%%%%%%%%%%%%%%%%%%%%%%%%%%%%%%%%%%%%%%%%%%%%%
% CS110: Introduction to Computing
% Copyright 2015 Pejman Ghorbanzade <mail@ghorbanzade.com>
% Creative Commons Attribution-ShareAlike 4.0 International License
% https://github.com/ghorbanzade/UMB-CS110-2015S/blob/master/LICENSE
%%%%%%%%%%%%%%%%%%%%%%%%%%%%%%%%%%%%%%%%%%%%%%%%%%%%%%%%%%%%%%%%%%%%%%%%%%%%%%

\course{id}{CS110}
\course{name}{Introduction to Computing}
\course{venue}{Tue/Thu, 5:30 PM - 6:45 PM}
\course{semester}{Spring 2015}
\course{department}{Department of Computer Science}
\course{university}{University of Massachusetts Boston}

\instructor{name}{Pejman Ghorbanzade}
\instructor{title}{}
\instructor{position}{Student Instructor}
\instructor{email}{pejman@cs.umb.edu}
\instructor{phone}{617-287-6419}
\instructor{office}{S-3-124B}
\instructor{office-hours}{Tue/Thu 19:00-20:30}
\instructor{address}{University of Massachusetts Boston, 100 Morrissey Blvd., Boston, MA}

\usepackage{\texDirectory/template/style/beamerthemeUmassLecture}
\doc{number}{16}
%\setbeamertemplate{footline}[text line]{}

\begin{document}
\prepareCover

\section{Course Administration}

\begin{frame}[fragile]
\frametitle{Course Administration}
Deadline for Assignment 5 extended to May 3, 2015 at 11:59 PM.
\end{frame}

\begin{frame}[fragile]
	\frametitle{Overview}
	\begin{itemize}
		\item[] Class BigInteger
		\item[] Recursion
	\end{itemize}
\end{frame}

\section{Class BigInteger}

\begin{frame}[fragile]
\frametitle{Class BigInteger}
	\begin{block}{Objective}
		Write an algorithm that asks user for a number $x$ and gives as output the value of $x!$ where $x$ is any number between 0 to 50.
	\end{block}
\end{frame}

\begin{frame}[fragile]
\frametitle{Class BigInteger}
	\begin{block}{Class \texttt{Factorial.java} (v1.0) (Page 1)}
		\begin{minted}[fontsize=\small,linenos,firstnumber=1]{java}
import java.util.InputMismatchException;
import java.util.Scanner;
public class Factorial1 {
	public static long calculateFactorial(int num)
			throws OutOfRangeException {
		if (num > 50 || num < 0)
			throw new OutOfRangeException("Out of Range");
		long result = num;
		for (int i = num-1; i >= 1; i--)
			 result *= i;
		return result;
	}
		\end{minted}
	\end{block}
\end{frame}

\begin{frame}[fragile]
\frametitle{Class BigInteger}
	\begin{block}{Class \texttt{Factorial.java} (v1.0) (Page 2)}
		\begin{minted}[fontsize=\small,linenos,firstnumber=13]{java}
	public static void main(String[] args) {
		System.out.print("Enter number: ");
		Scanner input = new Scanner(System.in);
		try {
			int num = input.nextInt();
			System.out.println(num+"! is: "
					+calculateFactorial(num));
		}
		catch (InputMismatchException|OutOfRangeException e) {
			e.getMessage();
			e.printStackTrace();
		}
		input.close();
	}
}
		\end{minted}
	\end{block}
\end{frame}

\begin{frame}[fragile]
\frametitle{Class BigInteger}
	\begin{block}{\texttt{OutOfRangeException.java}}
		\begin{minted}[fontsize=\small,linenos,firstnumber=1]{java}
@SuppressWarnings("serial")
public class OutOfRangeException extends Exception {
	private String message;
	public OutOfRangeException(String message) {
		 this.message = message;
	}
	public String getMessage() {
		return message;
	}
}
		\end{minted}
	\end{block}
\end{frame}

\begin{frame}[fragile]
\frametitle{Class BigInteger}
	\begin{block}{Problem Statement}
		No primitive data type can represent $x!$ when $x > 20$. Using primitive data types will lead to data overflow runtime error.
	\end{block}
	\begin{block}{Output}
	\begin{verbatim}
	> Enter number: 20
	20! is: 2432902008176640000
	> Enter number: 21
	21! is: -4249290049419214848
	\end{verbatim}
	\end{block}
\end{frame}

\begin{frame}[fragile]
\frametitle{Class BigInteger}
	\begin{block}{Proposed Solution}
		Class \textbf{BigInteger} is provided for handling extremely large numbers whose size exceeds that of primitive data types.
	\end{block}
\end{frame}

\begin{frame}[fragile]
\frametitle{Class BigInteger}
	\begin{block}{Drawbacks}
		\begin{itemize}
			\item[] Java does not support operation overloading. All arithmetic operations to be carried out by provided methods.
			\item[] Objects of type BigInteger are immutable. Stored value of an object can never be changed.
			\item[] Arithmetic operations are substantially slower.
			\item[] Memory requirements are substantially higher.
		\end{itemize}
	\end{block}
\end{frame}

\begin{frame}[fragile]
\frametitle{Class BigInteger}
	\begin{block}{\texttt{BigInt.java}}
		\begin{minted}[fontsize=\small,linenos,firstnumber=1]{java}
import java.math.BigInteger;
public class BigInt {
	public static void main(String[] args) {
		BigInteger b1 = new BigInteger(
				String.valueOf(Integer.MAX_VALUE));
		BigInteger b2 = new BigInteger(
				String.valueOf(Integer.MAX_VALUE));
		System.out.println(b1.multiply(b2));
		System.out.println(b1.divide(b2));
		System.out.println(b1.pow(500));
	}
}
		\end{minted}
	\end{block}
\end{frame}

\begin{frame}[fragile]
\frametitle{Class BigInteger}
	\begin{block}{Methods}
		\begin{columns}
			\begin{column}{0.3\textwidth}
				\begin{itemize}
					\item[] abs()
					\item[] not()
					\item[] add()
					\item[] divide()
					\item[] multiple()
					\item[] subtract()
				\end{itemize}
			\end{column}
			\begin{column}{0.3\textwidth}
				\begin{itemize}
					\item[] pow()
					\item[] mod()
					\item[] max()
					\item[] min()
					\item[] compareTo()
					\item[] remainder()
				\end{itemize}
			\end{column}
			\begin{column}{0.3\textwidth}
				\begin{itemize}
					\item[] gcd()
					\item[] signum()
					\item[] clearBit()
					\item[] setBit()
					\item[] testBit()
					\item[] toByteArray()
				\end{itemize}
			\end{column}
		\end{columns}
	\end{block}
\end{frame}

\begin{frame}[fragile]
\frametitle{Class BigInteger}
	\begin{block}{Class \texttt{Factorial.java} (v2.0) (Page 1)}
		\begin{minted}[fontsize=\small,linenos,firstnumber=1]{java}
import java.util.InputMismatchException;
import java.util.Scanner;
import java.math.BigInteger;
public class Factorial2 {
	public static BigInteger calculateFactorial(int num)
			throws OutOfRangeException {
		if (num > 50 || num < 0)
			throw new OutOfRangeException("Out of Range");
		BigInteger result = BigInteger.ONE;
		for (int i=1; i<=num; i++)
			result = result.multiply(BigInteger.valueOf(i));
		return result;
	}
		\end{minted}
	\end{block}
\end{frame}

\begin{frame}[fragile]
\frametitle{Class BigInteger}
	\begin{block}{Class \texttt{Factorial.java} (v2.0) (Page 2)}		\begin{minted}[fontsize=\small,linenos,firstnumber=14]{java}
	public static void main(String[] args) {
		System.out.print("Enter number: ");
		Scanner input = new Scanner(System.in);
		try {
			int num = input.nextInt();
			System.out.println(num + "! = "
					+calculateFactorial(num));
		}
		catch (InputMismatchException|OutOfRangeException e) {
			e.getMessage();
			e.printStackTrace();
		}
		input.close();
	}
}
		\end{minted}
	\end{block}
\end{frame}

\section{Recursion}

\begin{frame}[fragile]
\frametitle{Recursion}
\begin{quote}
To understand what recursion is, you must first understand recursion.
\end{quote}
\end{frame}

\begin{frame}[fragile]
\frametitle{Recursion}
	\begin{block}{Objective}
		Write a program \texttt{Factorial.java} that asks user for a number $x$ and gives as output the value of $x!$ where $x$ is any number between 0 to 50.
	\end{block}
\end{frame}

\begin{frame}[fragile]
\frametitle{Recursion}
	\begin{block}{Alternative Solution}
		Calculating factorial for any positive number $x$ ($x \neq 1$) requires multiplication of $x$ with factorial of $x-1$.
		\begin{equation}
		n! = n \times (n-1)!
		\end{equation}
	\end{block}
\end{frame}

\begin{frame}[fragile]
\frametitle{Class BigInteger}
	\begin{block}{Class \texttt{Factorial.java} (v3.0) (Page 1)}
		\begin{minted}[fontsize=\small,linenos,firstnumber=1]{java}
import java.util.InputMismatchException;
import java.util.Scanner;
import java.math.BigInteger;
public class Factorial3 {
	public static BigInteger calculateFactorial(int num) {
		if (num == 0)
			return BigInteger.ONE;
		else
			return BigInteger.valueOf(num).multiply(
					calculateFactorial(num-1));
	}
		\end{minted}
	\end{block}
\end{frame}

\begin{frame}[fragile]
\frametitle{Class BigInteger}
	\begin{block}{Class \texttt{Factorial.java} (v3.0) (Page 2)}
		\begin{minted}[fontsize=\small,linenos,firstnumber=12]{java}
	public static void main(String[] args) {
		System.out.print("Enter number: ");
		Scanner input = new Scanner(System.in);
		try {
			int num = input.nextInt();
			if (num > 50 || num < 0)
				throw new OutOfRangeException("Out of Range");
			System.out.println(num+"! = "+calculateFactorial(num));
		}
		catch (InputMismatchException|OutOfRangeException e) {
			e.getMessage();
			e.printStackTrace();
		}
		input.close();
	}
}
		\end{minted}
	\end{block}
\end{frame}

\begin{frame}[fragile]
\frametitle{Recursion}
\begin{figure}[H]\centering
\newcounter{density}
\setcounter{density}{20}
\definecolor{turquoise}{HTML}{48D1CC}
\def\couleur{turquoise}
\begin{tikzpicture}
		\path[coordinate] (0,0) coordinate(A)
			++( 60:8cm) coordinate(B)
			++(-60:8cm) coordinate(C);
		\draw[fill=\couleur!\thedensity] (A) -- (B) -- (C) -- cycle;
		\foreach \x in {1,...,25}{%
				\pgfmathsetcounter{density}{\thedensity+10}
				\setcounter{density}{\thedensity}
				\path[coordinate] coordinate(X) at (A){};
				\path[coordinate] (A)
					-- (B) coordinate[pos=.10](A)
					-- (C) coordinate[pos=.10](B)
					-- (X) coordinate[pos=.10](C);
				\draw[fill=\couleur!\thedensity] (A)--(B)--(C)--cycle;
		}
\end{tikzpicture}
\end{figure}
\end{frame}

\begin{frame}[fragile]
\frametitle{Recursion}
	\begin{block}{Definition}
		Recursion is a function-call mechanism in which a method returns a call to itself, usually with a different input.
	\end{block}
\end{frame}

\begin{frame}[fragile]
\frametitle{Recursion}
	\begin{block}{Motivation}
		\begin{itemize}
			\item[] Recursive models make many real-life phenomena easy to explain.
			\item[] Recursive models make many computational problems easy to solve.
		\end{itemize}
	\end{block}
\end{frame}

\begin{frame}[fragile]
\frametitle{Recursion}
	\begin{block}{Objective}
		Write a program \texttt{Fibonacci.java} that asks user for a number $x$ and gives as output the value of $x^{th}$ element of the Fibonacci series.
	\end{block}
\end{frame}

\begin{frame}[fragile]
\frametitle{Class BigInteger}
	\begin{block}{Class \texttt{Fibonacci.java} (Page 1)}
		\begin{minted}[fontsize=\small,linenos,firstnumber=1]{java}
import java.util.InputMismatchException;
import java.util.Scanner;
public class Fibonacci {
	public static int fib(int n) {
		if (n < 2)
				return n;
			else
				return fib(n - 1) + fib(n - 2);
	}
		\end{minted}
	\end{block}
\end{frame}

\begin{frame}[fragile]
\frametitle{Class BigInteger}
	\begin{block}{Class \texttt{Fibonacci.java} (Page 2)}
		\begin{minted}[fontsize=\small,linenos,firstnumber=10]{java}
	public static void main(String[] args) {
		System.out.print("Enter number: ");
		Scanner input = new Scanner(System.in);
		try {
			int num = input.nextInt();
			System.out.println(num+"! is: "+fib(num));
		}
		catch (InputMismatchException e) {
			e.getMessage();
			e.printStackTrace();
		}
		input.close();
	}
}
		\end{minted}
	\end{block}
\end{frame}

\begin{frame}[fragile]
\frametitle{Recursion}
	\begin{block}{Conditions}
		\begin{itemize}
			\item[] \textbf{Base Case:}\\A recursive method must end calling itself at some point.
			\item[] \textbf{Convergence:}\\A recursive call must be computationally less complicated.
			\item[] \textbf{Overlapping:}\\No two recursive calls must overlap.
		\end{itemize}
	\end{block}
\end{frame}

\begin{frame}[fragile]
\frametitle{Recursion}
	\begin{block}{Caution}
		\begin{itemize}
			\item[] A recursive method must return before memory allocations become prohibitive causing \texttt{\small StackOverflowError}.
			\item[] Although easy to understand, a recursive method is not guaranteed to have reasonable run-time performance.
		\end{itemize}
	\end{block}
\end{frame}

\plain{}{Keep Calm\\and\\Enjoy Programming}

\end{document}
