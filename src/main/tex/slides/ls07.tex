%%%%%%%%%%%%%%%%%%%%%%%%%%%%%%%%%%%%%%%%%%%%%%%%%%%%%%%%%%%%%%%%%%%%%%%%%%%%%%
% CS110: Introduction to Computing
% Copyright 2015 Pejman Ghorbanzade <mail@ghorbanzade.com>
% Creative Commons Attribution-ShareAlike 4.0 International License
% https://github.com/ghorbanzade/UMB-CS110-2015S/blob/master/LICENSE
%%%%%%%%%%%%%%%%%%%%%%%%%%%%%%%%%%%%%%%%%%%%%%%%%%%%%%%%%%%%%%%%%%%%%%%%%%%%%%

\def \topDirectory {../..}
\def \texDirectory {\topDirectory/src/main/tex}

\documentclass[10pt, compress]{beamer}

\usepackage{\texDirectory/template/style/directives}
%%%%%%%%%%%%%%%%%%%%%%%%%%%%%%%%%%%%%%%%%%%%%%%%%%%%%%%%%%%%%%%%%%%%%%%%%%%%%%
% CS110: Introduction to Computing
% Copyright 2015 Pejman Ghorbanzade <mail@ghorbanzade.com>
% Creative Commons Attribution-ShareAlike 4.0 International License
% https://github.com/ghorbanzade/UMB-CS110-2015S/blob/master/LICENSE
%%%%%%%%%%%%%%%%%%%%%%%%%%%%%%%%%%%%%%%%%%%%%%%%%%%%%%%%%%%%%%%%%%%%%%%%%%%%%%

\course{id}{CS110}
\course{name}{Introduction to Computing}
\course{venue}{Tue/Thu, 5:30 PM - 6:45 PM}
\course{semester}{Spring 2015}
\course{department}{Department of Computer Science}
\course{university}{University of Massachusetts Boston}

\instructor{name}{Pejman Ghorbanzade}
\instructor{title}{}
\instructor{position}{Student Instructor}
\instructor{email}{pejman@cs.umb.edu}
\instructor{phone}{617-287-6419}
\instructor{office}{S-3-124B}
\instructor{office-hours}{Tue/Thu 19:00-20:30}
\instructor{address}{University of Massachusetts Boston, 100 Morrissey Blvd., Boston, MA}

\usepackage{\texDirectory/template/style/beamerthemeUmassLecture}
\doc{number}{7}
%\setbeamertemplate{footline}[text line]{}

\begin{document}
\prepareCover

\section{Course Administration}

\begin{frame}[fragile]
\frametitle{Course Administration}
	Assignment 2 deadline extended to Mar 5, 2015 at 17:30 PM.

	All Submissions should contain only .java files.

	Make sure your submission was sucessful by downloading it from server.
\end{frame}

\begin{frame}[fragile]
	\frametitle{Overview}
	\begin{itemize}
		\item[] Arrays
			\begin{itemize}
				\item[] One-Dimensional Arrays
				\item[] Multi-Dimensional Arrays
			\end{itemize}
	\end{itemize}
\end{frame}

\plain{}{Arrays}

\section{One-Dimensional Arrays}

\begin{frame}[fragile]
	\frametitle{One-Dimensional Arrays}
	\begin{block}{Objective}
		Write a program \texttt{RandomCard.java} that randomly selects a card from a standard 52-card deck and prints its name.
	\end{block}
\end{frame}

\begin{frame}[fragile]
	\frametitle{One-Dimensional Arrays}
	\begin{block}{\texttt{RandomCard.java} (v1.0)}
		\begin{minted}[fontsize=\small,tabsize=8, linenos, firstnumber=1]{java}
public class RandomCard {
	public static void main(String[] args) {
		int numCard = 52;
		int randNum = 1 +
				(int) Math.floor( Math.random() * numCard );
		// Now I have to print name of the card
		switch (randNum) {
			case 1 : cardName = "Ace of Spades"; break;
			case 2 : cardName = "Two of Spades"; break;
			//... seriously?
			case 51 : cardName = "Queen of Hearts"; break;
			case 52 : cardName = "King of Hearts"; break;
		}
	}
}
		\end{minted}
	\end{block}
\end{frame}

\begin{frame}[fragile]
	\frametitle{One-Dimensional Arrays}
	\begin{block}{\texttt{RandomCard.java} (v2.0)}
		\begin{minted}[fontsize=\small,tabsize=8, linenos, firstnumber=5]{java}
suitNum = randNum / 13;
switch (suitNum) {
	case 1 : suitName = "Spades"; break;
	case 2 : suitName = "Diamonds"; break;
	case 3 : suidName = "Clubs"; break;
	case 4 : suitName = "Hearts"; break;
}
switch cardNum = numCard % 13;
switch (cardNum) {
	case 1 : cardName = "Ace"; break;
	case 2 : cardName = "Two"; break;
	//... not cool!
	case 12 : cardName = "Queen"; break;
	case 13 : cardName = "King"; break;
}
System.out.println(cardName + " of " + suitName);
		\end{minted}
	\end{block}
\end{frame}

\begin{frame}[fragile]
	\frametitle{One-Dimensional Arrays}
	\begin{block}{Problem Statement}
		Access to names of values and suits are limited.

		Same code should be executed every time name of card is needed.

		Solution is not expandable.
	\end{block}
	\begin{block}{Clue}
		All cases assign string literals to one single variable.

		Name of $13^{th}$ card numbers for every suit is "King".
	\end{block}
\end{frame}

\begin{frame}[fragile]
	\frametitle{One-Dimensional Arrays}
	\begin{block}{Proposed Solution}
		Define an ordered collection of card names and get name of a card by calling its position in collection.
		\begin{minted}[fontsize=\small,tabsize=8]{java}
String[] SuitNames = {
	"Spades", "Diamonds", "Clubs", "Hearts"};
String[] CardNames = {
	"Ace", "Two", "Three", "Four", "Five",
	"Six", "Seven", "Eight", "Nine", "Ten",
	"Soldier", "Queen", "King"};
		\end{minted}
	\end{block}
\end{frame}

\begin{frame}[fragile]
	\frametitle{One-Dimensional Arrays}
	\begin{block}{Definition}
		Array is a data structure that holds a \texttt{fixed} number of elements of a \texttt{single} type.

		Each array element is identified by its \texttt{numerical index}.

		An array is stored based on position of its \texttt{first} element.

		First element of an array has index 0.
	\end{block}
\end{frame}

\begin{frame}[fragile]
	\frametitle{One-Dimensional Arrays}
	\begin{block}{Definition}
		\begin{minted}[fontsize=\small,tabsize=8]{java}
int[] primes;
primes = new int[4];
primes[0] = 2;
primes[1] = 3;
primes[2] = 5;
primes[3] = 7;
for (int i = 0; i < 4; i++)
	System.out.println("Prime " + (i+1) + " is " + primes[i]);
		\end{minted}
	\end{block}
\end{frame}

\begin{frame}[fragile]
	\frametitle{One-Dimensional Arrays}
	\begin{block}{\texttt{RandomCard.java} (v3.0)}
		\begin{minted}[fontsize=\small,tabsize=8, linenos, firstnumber=5]{java}
String[] SuitNames = {
	"Spades", "Diamonds", "Clubs", "Hearts"};
String[] CardNames = {
	"Ace", "Two", "Three", "Four", "Five",
	"Six", "Seven", "Eight", "Nine", "Ten",
	"Soldier", "Queen", "King"};
suitNum = randNum / 13;
cardNum = numCard % 13;
suitName = suitNames[suitNum];
cardName = cardNames[cardNum];
System.out.println(cardName + " of " + suitName);
		\end{minted}
	\end{block}
\end{frame}

\begin{frame}[fragile]
	\frametitle{One-Dimensional Arrays}
	\begin{block}{Advantages}
		\begin{itemize}
			\item[] Data is more organized.
			\item[] Code reuse is now possible.
			\item[] Solution is expandable.
		\end{itemize}
	\end{block}
	\begin{block}{Disadvantages}
		\begin{itemize}
			\item[] Once declared, size of array is fixed.
			\item[] Insertion and deletion are not possible.
		\end{itemize}
	\end{block}
\end{frame}

\begin{frame}[fragile]
	\frametitle{One-Dimensional Arrays}
	\begin{block}{Objective}
		Write a program \texttt{PrimeNumber.java} that first prints all first 1000 prime numbers.
	\end{block}
\end{frame}

\begin{frame}[fragile]
	\frametitle{One-Dimensional Arrays}
	\begin{block}{\texttt{PrimeNumber.java} (v1.0)}
		\begin{minted}[fontsize=\small,tabsize=8, linenos, firstnumber=3]{java}
int counter = 0;
int number = 2;
boolean definitelyNotPrime;
while (counter < 1000) {
	definitelyNotPrime = false;
	for (int i = 2; i < number/2; i++)
		if (number % i == 0){
			definitelyNotPrime = true;
			break;
		}
	if (!definitelyNotPrime)
		System.out.printf("%4d: %d\n",++counter,number);
	number++;
}
		\end{minted}
	\end{block}
\end{frame}

\begin{frame}[fragile]
	\frametitle{One-Dimensional Arrays}
	\begin{block}{Problem Statement}
		To check if number \texttt{N} is prime, it suffices to check whether it is devisible by at least one prime number less than or equal to \texttt{N/2}.
		\begin{itemize}
			\item[] Not necessary to check all numbers less than \texttt{N/2}.
		\end{itemize}
		What if we are asked to print $1^{st}$, $3^{rd}$, $5^{th}$, ... prime numbers, after printing all first 1000 prime numbers?
	\end{block}
\end{frame}

\begin{frame}[fragile]
	\frametitle{One-Dimensional Arrays}
	\begin{block}{Objective}
		Write a program \texttt{PrimeNumber.java} that first prints all first 1000 prime numbers then prints 500th prime number.
	\end{block}
\end{frame}

\begin{frame}[fragile]
	\frametitle{One-Dimensional Arrays}
	\begin{block}{\texttt{PrimeNumber.java} (v2.0)}
		\begin{minted}[fontsize=\small,tabsize=8, linenos, firstnumber=3]{java}
int[] primeNumbers = new int[1000];
int counter = 0;
int number = 2;
boolean definitelyNotPrime;
while (counter < primeNumbers.length) {
	definitelyNotPrime = false;
	for (int i = 0; i < counter && !definitelyNotPrime; i++)
		if (number % primeNumbers[i] == 0)
			definitelyNotPrime = true;
	if (!definitelyNotPrime) {
		primeNumbers[counter++] = number;
		System.out.printf("%4d: %d\n",counter,number);
	}
	number++;
}
System.out.printf("%4d: %d\n",500,primeNumbers[499]);
		\end{minted}
	\end{block}
\end{frame}

\section{Multi-Dimensional Arrays}

\begin{frame}[fragile]
	\frametitle{Multi-Dimensional Arrays}
	\begin{block}{Dimension Definition}
		The dimension of an array is the number of indices needed to select an element.
	\end{block}
	\begin{block}{Initialization}
		\begin{minted}[fontsize=\small,tabsize=8]{java}
int[][] matrix = new int[3][3];
for (int i = 0; i < 3; i++)
	for (int j = 0; j < 3; j++)
		matrix[i][j] = 0;
		\end{minted}
	\end{block}
\end{frame}

\begin{frame}[fragile]
	\frametitle{Multi-Dimensional Arrays}
	\begin{block}{Objective}
		Write a program \texttt{MagicMatrix.java} that takes number \texttt{N} and generates a magic matrix of size \texttt{N} using numbers $1$ to $N^2$.
	\end{block}
\end{frame}

\begin{frame}[fragile]
	\frametitle{Multi-Dimensional Arrays}
	\begin{block}{Simplification}
		\begin{itemize}
			\item[] Get matrix size N from user
			\item[] Build an array of length $N^2$ from 1 to N
			\item[] Shuffle the array
			\item[] Convert 1-D Array to 2-D Array
			\item[] Print elements of array
		\end{itemize}
	\end{block}
\end{frame}

\begin{frame}[fragile]
	\frametitle{Multi-Dimensional Arrays}
	\begin{block}{\texttt{MagicMatrix.java} (part 1)}
		\begin{minted}[fontsize=\small,tabsize=8, linenos, firstnumber=1]{java}
public class Array {
	public static void main(String[] args) {
		// get matrix size
		Scanner input = new Scanner(System.in);
		int num = input.nextInt();
		input.close();

		// initialize array
		int[] numbers = new int[num * num];
		for (int i = 0; i < numbers.length; i++)
			numbers[i] = i+1;
		\end{minted}
	\end{block}
\end{frame}

\begin{frame}[fragile]
	\frametitle{Multi-Dimensional Arrays}
	\begin{block}{\texttt{MagicMatrix.java} (part 2)}
		\begin{minted}[fontsize=\small,tabsize=8, linenos, firstnumber=12]{java}
		// shuffle array
		for (int i = 0; i < numbers.length; i++) {
			int j = (int) (i + (numbers.length-i)*Math.random());
			int temp = numbers[j];
			numbers[j] = numbers[i];
			numbers[i] = temp;
		}
		\end{minted}
	\end{block}
\end{frame}

\begin{frame}[fragile]
	\frametitle{Multi-Dimensional Arrays}
	\begin{block}{\texttt{MagicMatrix.java} (part 3)}
		\begin{minted}[fontsize=\small,tabsize=8, linenos, firstnumber=19]{java}
		// convert shuffled array to matrix
		int[][] matrix = new int[num][num];
		for (int i = 0; i < numbers.length; i++) {
			matrix[i/num][i%num] = numbers[i];
		}

		// show matrix
		for (int i = 0; i < num; i++) {
			for (int j = 0; j < num; j++)
				System.out.print(matrix[i][j]);
			System.out.println();
		}
	}
}
		\end{minted}
	\end{block}
\end{frame}

\begin{frame}[fragile]
	\frametitle{Multi-Dimensional Arrays}
	\begin{block}{Higher Dimensional Arrays}
		\begin{minted}[fontsize=\small,tabsize=8, linenos, firstnumber=1]{java}
int num = 1;
int[][][] cubicMatrix = new int[3][3][3];
for (int i = 1; i < 3; i++)
	for (int j = 1; j < 3; j++)
		for (int k = 1; k < 3; k++)
			cubicMatrix[i][j][k] = num++;
		\end{minted}
	\end{block}
\end{frame}

\begin{frame}[fragile]
	\frametitle{Multi-Dimensional Arrays}
	\begin{block}{Remember}
		\begin{itemize}
			\item[] Arrays are simple data structures for data organization
			\item[] Arrays provide an efficient way to access data
			\item[] Elements of array are of same type
			\item[] Arrays have fixed-size
		\end{itemize}
	\end{block}
\end{frame}

\plain{}{Keep Calm\\and\\Practice}

\end{document}
