%%%%%%%%%%%%%%%%%%%%%%%%%%%%%%%%%%%%%%%%%%%%%%%%%%%%%%%%%%%%%%%%%%%%%%
% UMB-CS110-2015S: Introduction to Computing
% Copyright 2015 Pejman Ghorbanzade <pejman@ghorbanzade.com>
% Creative Commons Attribution-ShareAlike 4.0 International License
% More info: https://github.com/ghorbanzade/UMB-CS110-2015S
%%%%%%%%%%%%%%%%%%%%%%%%%%%%%%%%%%%%%%%%%%%%%%%%%%%%%%%%%%%%%%%%%%%%%%

\def \topDirectory {../..}
\def \texDirectory {\topDirectory/src/main/tex}

\documentclass[10pt, compress]{beamer}

\usepackage{\texDirectory/template/style/directives}
%%%%%%%%%%%%%%%%%%%%%%%%%%%%%%%%%%%%%%%%%%%%%%%%%%%%%%%%%%%%%%%%%%%%%%%%%%%%%%
% CS110: Introduction to Computing
% Copyright 2015 Pejman Ghorbanzade <mail@ghorbanzade.com>
% Creative Commons Attribution-ShareAlike 4.0 International License
% https://github.com/ghorbanzade/UMB-CS110-2015S/blob/master/LICENSE
%%%%%%%%%%%%%%%%%%%%%%%%%%%%%%%%%%%%%%%%%%%%%%%%%%%%%%%%%%%%%%%%%%%%%%%%%%%%%%

\course{id}{CS110}
\course{name}{Introduction to Computing}
\course{venue}{Tue/Thu, 5:30 PM - 6:45 PM}
\course{semester}{Spring 2015}
\course{department}{Department of Computer Science}
\course{university}{University of Massachusetts Boston}

\instructor{name}{Pejman Ghorbanzade}
\instructor{title}{}
\instructor{position}{Student Instructor}
\instructor{email}{pejman@cs.umb.edu}
\instructor{phone}{617-287-6419}
\instructor{office}{S-3-124B}
\instructor{office-hours}{Tue/Thu 19:00-20:30}
\instructor{address}{University of Massachusetts Boston, 100 Morrissey Blvd., Boston, MA}

\usepackage{\texDirectory/template/style/beamerthemeUmassLecture}
\usepackage[school]{\texDirectory/template/pgf-umlcd/pgf-umlcd}
\doc{number}{13}
%\setbeamertemplate{footline}[text line]{}

\begin{document}
\prepareCover

\section{Course Administration}

\begin{frame}[fragile]
\frametitle{Course Administration}
Deadline for Assignment 4 extended to April 18, 2015 at 11:59 PM.
Assignment 5 to be released on April 19, 2015 at 12:00 AM.
\end{frame}

\begin{frame}[fragile]
  \frametitle{Overview}
  \begin{itemize}
    \item[] Interfaces
  \end{itemize}
\end{frame}

\section{Interfaces}

\begin{frame}[fragile]
  \frametitle{Interfaces}
  \begin{block}{How Do You Think?}
    How do we drive?\\
    How do we ride a bike?\\
    How do we use a computer?\\
    How do we use a smart phone?
  \end{block}
\end{frame}

\begin{frame}[fragile]
  \frametitle{Interfaces}
  \begin{block}{Definition}
    An interface is an abstract type developed to specify method signatures that classes must implement.
    \begin{minted}[fontsize=\small,tabsize=8, linenos, firstnumber=1]{java}
public interface Controller {
  // constants
  // method signatures
}
    \end{minted}
  \end{block}
\end{frame}

\begin{frame}[fragile]
  \frametitle{Interfaces}
  \begin{block}{Implementation}
    A class implementing an interface, either promises to implement all its method signatures or is declared as abstract.
    \begin{minted}[fontsize=\small,tabsize=8, linenos, firstnumber=1]{java}
public class Television implements Controller {
  // attributes and fields
  // implementation of interface's method signatures
  // other methods
}
    \end{minted}
  \end{block}
\end{frame}

\begin{frame}[fragile]
  \frametitle{Interfaces}
  \begin{block}{Objective}
    Write a program \texttt{ControlTV.java} in which user can control a Samsung television using a remote controller.
  \end{block}
\end{frame}

\begin{frame}[fragile]
  \frametitle{Interfaces}
  \begin{block}{Object-Oriented Design}
    \begin{itemize}
      \item[] Identify objects
      \item[] Identify classes
      \item[] Identify abstract classes
      \item[] Identify interactions between objects
      \item[] Design relation between classes
      \item[] Implement methods
      \item[] Instantiate objects
      \item[] Use objects to achieve your goal
    \end{itemize}
  \end{block}
\end{frame}

\begin{frame}[fragile]
  \frametitle{Interfaces}
  \begin{block}{Object-Oriented Design}
    \begin{itemize}
      \item[] \textbf{Objects}\\myTv
      \item[] \textbf{Classes}\\Samsung
      \item[] \textbf{Abstract Classes}\\Television
      \item[] \textbf{Interfaces}\\Remote controller
    \end{itemize}
  \end{block}
\end{frame}

\begin{frame}[fragile]
  \frametitle{Interfaces}
  \begin{block}{Class Relations}
    Any television supports primary keys of a remote controller to change volume, current channel and whether it's on or off.
    Although, any television knows what to do when client presses power button, different television companies may implement other keys differently.
    Samsung is a television company from which our object is instantiated.
  \end{block}
\end{frame}

\begin{frame}[fragile]
  \frametitle{Interfaces}
  \begin{block}{Controller.java}
    \begin{minted}[fontsize=\small,tabsize=8, linenos, firstnumber=1]{java}
public interface Controller {
  public void keyPower();
  public void keyRight();
  public void keyLeft();
  public void keyUp();
  public void keyDown();
  public void number(int input);
}
    \end{minted}
  \end{block}
\end{frame}

\begin{frame}[fragile]
  \frametitle{Interfaces}
  \begin{block}{Television .java (Page 1)}
    \begin{minted}[fontsize=\small,tabsize=8, linenos, firstnumber=1]{java}
public abstract class Television implements Controller {
  // fields and attributes
  private int channel;
  private int volume;
  private boolean on = false;
  // abstract methods
  @Override
  public void keyPower() {
    on = on ? false : true;
    System.out.print("TV turned ");
    System.out.println( on ? "on." : "off.");
  }
    \end{minted}
  \end{block}
\end{frame}

\begin{frame}[fragile]
  \frametitle{Interfaces}
  \begin{block}{Television.java (Page 2)}
    \begin{minted}[fontsize=\small,tabsize=8, linenos, firstnumber=13]{java}
  // setters and getters
  public boolean isOn() {
    return on;
  }
  public void setChannel(int currentChannel) {
    this.channel = currentChannel;
  }
  public int getChannel() {
    return this.channel;
  }
  public void setVolume(int currentVolume) {
    this.volume = currentVolume;
  }
  public int getVolume() {
    return this.volume;
  }
}
    \end{minted}
  \end{block}
\end{frame}

\begin{frame}[fragile]
  \frametitle{Interfaces}
  \begin{block}{Samsung.java (Page 1)}
    \begin{minted}[fontsize=\small,tabsize=8, linenos, firstnumber=1]{java}
public class Samsung extends Television {
  // Attributes
  private final int channelUpperLimit = 100;
  private final int channelLowerLimit = 0;
  private final int volumeUpperLimit = 30;
  private final int volumeLowerLimit = 0;
    \end{minted}
  \end{block}
\end{frame}


\begin{frame}[fragile]
  \frametitle{Interfaces}
  \begin{block}{Samsung.java (Page 2)}
    \begin{minted}[fontsize=\small,tabsize=8, linenos, firstnumber=7]{java}
  // Overriding Abstract Methods
  @Override
  public void keyRight() {
    if (this.isOn()) {
      if (this.getVolume() < volumeUpperLimit) {
        this.setVolume(getVolume() + 1);
        System.out.println("Samsung volume changed to "
            +this.getVolume()+".");
      }
    }
  }
    \end{minted}
  \end{block}
\end{frame}

\begin{frame}[fragile]
  \frametitle{Interfaces}
  \begin{block}{Samsung.java (Page 3)}
    \begin{minted}[fontsize=\small,tabsize=8, linenos, firstnumber=18]{java}
  @Override
  public void keyLeft() {
    if (this.isOn()) {
      if (this.getVolume() > volumeLowerLimit) {
        this.setVolume(getVolume() - 1);
        System.out.println("Samsung volume changed to "
            +this.getVolume()+".");
      }
    }
  }
    \end{minted}
  \end{block}
\end{frame}

\begin{frame}[fragile]
  \frametitle{Interfaces}
  \begin{block}{Samsung.java (Page 4)}
    \begin{minted}[fontsize=\small,tabsize=8, linenos, firstnumber=28]{java}
  @Override
  public void keyUp() {
    if (this.isOn()) {
      if (this.getChannel() == channelUpperLimit)
        this.setChannel(channelLowerLimit);
      else
        this.setChannel(this.getChannel() + 1);
      System.out.println("Samsung channel change to "
          +this.getChannel()+".");
    }
  }
    \end{minted}
  \end{block}
\end{frame}


\begin{frame}[fragile]
  \frametitle{Interfaces}
  \begin{block}{Samsung.java (Page 5)}
    \begin{minted}[fontsize=\small,tabsize=8, linenos, firstnumber=39]{java}
  @Override
  public void keyDown() {
    if (this.isOn()) {
      if (this.getChannel() == channelLowerLimit)
        this.setChannel(channelUpperLimit);
      else
        this.setChannel(this.getChannel() - 1);
      System.out.println("Samsung channel change to "
          +this.getChannel()+".");
    }
  }
    \end{minted}
  \end{block}
\end{frame}

\begin{frame}[fragile]
  \frametitle{Interfaces}
  \begin{block}{Samsung.java (Page 6)}
    \begin{minted}[fontsize=\small,tabsize=8, linenos, firstnumber=50]{java}
  @Override
  public void number(int input) {
    if (this.isOn()) {
      this.setChannel(input);
      System.out.println("Samsung channel change to "
          +this.getChannel()+".");
    }
  }
  // Constructors
  public Samsung() {
    setChannel(1);
    setVolume(10);
    System.out.println("Just bougth a new Samsung TV.");
  }
}
    \end{minted}
  \end{block}
\end{frame}

\begin{frame}[fragile]
  \frametitle{Interfaces}
  \begin{block}{ControlTV.java (Page 1)}
    \begin{minted}[fontsize=\small,tabsize=8, linenos, firstnumber=1]{java}
import java.util.Scanner;
public class Main {
  public static Scanner input = new Scanner(System.in);
  public static void main(String[] args) {
    Samsung myTv = new Samsung();
    while (true)
      if (showMenu(myTv))
        break;
  }
    \end{minted}
  \end{block}
\end{frame}

\begin{frame}[fragile]
  \frametitle{Interfaces}
  \begin{block}{ControlTV.java (Page 2)}
    \begin{minted}[fontsize=\small,tabsize=8, linenos, firstnumber=10]{java}
  public static boolean showMenu(Television myTv) {
    System.out.printf("\n\nHere are the options:\n");
    System.out.printf("\t1. Press Power Key\n");
    System.out.printf("\t2. Press Right Key\n");
    System.out.printf("\t3. Press Left Key\n");
    System.out.printf("\t4. Press Up Key\n");
    System.out.printf("\t5. Press Down Key\n");
    System.out.printf("\t6. Enter a number\n");
    System.out.printf("\t7. Put down remote controller\n");
    System.out.printf("What do you want to do [1-7]? ");
    return askOption(myTv);
  }
    \end{minted}
  \end{block}
\end{frame}

\begin{frame}[fragile]
  \frametitle{Interfaces}
  \begin{block}{ControlTV.java (Page 3)}
    \begin{minted}[fontsize=\small,tabsize=8, linenos, firstnumber=22]{java}
  public static boolean askOption(Television myTv) {
    switch (input.nextInt()) {
      case 1: myTv.keyPower(); break;
      case 2: myTv.keyRight(); break;
      case 3: myTv.keyLeft(); break;
      case 4: myTv.keyUp(); break;
      case 5: myTv.keyDown(); break;
      case 6:
        System.out.printf("Enter number: ");
        int num = input.nextInt();
        myTv.number(num); break;
      case 7: return true;
      default: break;
    }
    return false;
  }
}
    \end{minted}
  \end{block}
\end{frame}

\begin{frame}[fragile]
  \frametitle{Interfaces}
  \begin{block}{Achievement}
    Our TV supports the controller user interface (as any other television would), with an implementation specific to its brand.
  \end{block}
\end{frame}

\begin{frame}[fragile]
  \frametitle{Interfaces}
  \begin{block}{Design Advantage}
    Any class implementing an interface is guaranteed to implement method signatures developed in that interface.
    This helps organize structure of the program and facilitates interactions of objects of different classes.
  \end{block}
\end{frame}

\begin{frame}[fragile]
  \frametitle{Interfaces}
  \begin{block}{Remember}
    \begin{itemize}
      \item[] \textbf{Interface vs Abstract Class}\\
      Although a class may not extend multiple classes, it can implement multiple interfaces.

      \item[] \textbf{Relation}\\
      For a class to implement an interface, there must be an IS-A relationship from that class to the interface. Weird but true!
    \end{itemize}
  \end{block}
\end{frame}

\plain{}{Keep Calm\\and\\Think Object-Oriented}

\end{document}
