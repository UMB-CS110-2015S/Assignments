%%%%%%%%%%%%%%%%%%%%%%%%%%%%%%%%%%%%%%%%%%%%%%%%%%%%%%%%%%%%%%%%%%%%%%
% UMB-CS110-2015S: Introduction to Computing
% Copyright 2015 Pejman Ghorbanzade <pejman@ghorbanzade.com>
% Creative Commons Attribution-ShareAlike 4.0 International License
% More info: https://github.com/ghorbanzade/UMB-CS110-2015S
%%%%%%%%%%%%%%%%%%%%%%%%%%%%%%%%%%%%%%%%%%%%%%%%%%%%%%%%%%%%%%%%%%%%%%

\def \topDirectory {../..}
\def \texDirectory {\topDirectory/src/main/tex}

\documentclass[10pt, compress]{beamer}

\usepackage{\texDirectory/template/style/directives}
%%%%%%%%%%%%%%%%%%%%%%%%%%%%%%%%%%%%%%%%%%%%%%%%%%%%%%%%%%%%%%%%%%%%%%%%%%%%%%
% CS110: Introduction to Computing
% Copyright 2015 Pejman Ghorbanzade <mail@ghorbanzade.com>
% Creative Commons Attribution-ShareAlike 4.0 International License
% https://github.com/ghorbanzade/UMB-CS110-2015S/blob/master/LICENSE
%%%%%%%%%%%%%%%%%%%%%%%%%%%%%%%%%%%%%%%%%%%%%%%%%%%%%%%%%%%%%%%%%%%%%%%%%%%%%%

\course{id}{CS110}
\course{name}{Introduction to Computing}
\course{venue}{Tue/Thu, 5:30 PM - 6:45 PM}
\course{semester}{Spring 2015}
\course{department}{Department of Computer Science}
\course{university}{University of Massachusetts Boston}

\instructor{name}{Pejman Ghorbanzade}
\instructor{title}{}
\instructor{position}{Student Instructor}
\instructor{email}{pejman@cs.umb.edu}
\instructor{phone}{617-287-6419}
\instructor{office}{S-3-124B}
\instructor{office-hours}{Tue/Thu 19:00-20:30}
\instructor{address}{University of Massachusetts Boston, 100 Morrissey Blvd., Boston, MA}

\usepackage{\texDirectory/template/style/beamerthemeUmassLecture}
\doc{number}{1}
%\setbeamertemplate{footline}[text line]{}

\begin{document}
\prepareCover

\section{Course Administration}

\begin{frame}[fragile]
	\frametitle{Course Administration}
	\begin{block}{Lecture Venue}
		Tuesdays and Thursdays, 5:30 PM to 6:45 PM\\
		Science Building, 1st Floor, Room No. 6\\
		Small Science Auditorium\\
		Department of Computer Science\\
		University of Massachusetts Boston
	\end{block}
\end{frame}

\begin{frame}[fragile]
	\frametitle{Course Administration}
	\begin{block}{Lab Sessions}
		\begin{enumerate}
			\item Tuesdays, 4:00 PM -- 5:15 PM, W-02-0126
			\item Thursdays, 4:00 PM -- 5:15 PM, W-01-0064
			\item Tuesdays, 7:00 PM -- 8:15 PM, W-01-0042
			\item Thursdays, 7:00 PM -- 8:15 PM, M-01-0210
		\end{enumerate}
	\end{block}
\end{frame}

\begin{frame}[fragile]
	\frametitle{Course Administration}
	\begin{block}{Instructor}
		Pejman Ghorbanzade\\\href{mailto:mail@ghorbanzade.com}{(mail [at] ghorbanzade [dot] com)}

		\textbf{Office}\\S-3-124B (Network Information Systems Laboratory)

		\textbf{Office Hours}\\Tue, Thu 7:00 PM -- 8:30 PM (Other Times by Appointment)

		\textbf{Phone}\\617-287-6419
	\end{block}
\end{frame}

\begin{frame}[fragile]
	\frametitle{Course Administration}
	\begin{block}{Teaching Assistants}
		Yang Mu\\\href{mailto:yangmu@cs.umb.edu}{(yangmu [at] cs.umb [dot] edu)}

		\textbf{Office}\\S-3-158A (Knowledge Discovery Laboratory)

		\textbf{Office Hours}\\Tue, Thu 7:00 PM -- 8:15 PM

		\textbf{Phone}\\617-287-6438
	\end{block}
\end{frame}

\begin{frame}[fragile]
	\frametitle{Course Administration}
	\begin{block}{Teaching Assistants}
		Yahui Di\\\href{mailto:yahuidi@cs.umb.edu}{(yahuidi [at] cs.umb [dot] edu)}

		\textbf{Office}\\S-3-158A (Knowledge Discovery Laboratory)

		\textbf{Office Hours}\\Tue, Thu 4:00 PM -- 5:15 PM

		\textbf{Phone}\\617-287-6438
	\end{block}
\end{frame}

\begin{frame}[fragile]
	\frametitle{Course Administration}
	\begin{block}{Supplemental Instructions}
		Maya Cheriyan\\\href{mailto:maya.cheriyan001@umb.edu}{(maya.cheriyan [at] umb [dot] edu)}

		\textbf{Office}\\S-2-SSC (Student Success Center)

		\textbf{Supplemental Instruction Sessiosn}
		\begin{itemize}
			\item[] Mondays, 10:00 AM -- 12:00 PM, S-2-SSC
			\item[] Wednesdays, 10:00 AM -- 11:50 PM, M-2-213
		\end{itemize}
	\end{block}
\end{frame}

\begin{frame}
	\frametitle{Course Administration}
	\begin{block}{Course Website}
		Students are advised to check \href{http://www.ghorbanzade.com/?tab=5&course=1}{\alert{Course Website}} regularly for announcements, assignments, learning materials and forum discussions.
	\end{block}
\end{frame}

\begin{frame}
	\frametitle{Course Administration}
	\begin{block}{Textbook}
		\begin{itemize}
			\item[] Robert Sedgewick, Kevin Wayne, \emph{Introduction to Programming in Java: An Interdisciplinary Approach}, Addison-Wesley, 2007.
		\end{itemize}
		Textbook is recommended but not required.

		Lectures will cover topics in the textbook and beyond.

		A list of useful resources are posted on course website.
	\end{block}
\end{frame}

\begin{frame}
	\frametitle{Course Administration}
	\begin{block}{Prerequisites}
		Students taking this course should have passed the Math Placement Exam (accessible through WISER system) with score 130 or higher.

		Previous computer programming skills are not required.

		Creative mind is a must.
	\end{block}
\end{frame}

\begin{frame}
	\frametitle{Course Administration}
	\begin{block}{Grading}
		\begin{itemize}
			\item[-] Lab Quizzes (20 points)\\5 programming quizzes each 4 points
			\item[-] Assignments (40 points)\\5 assignments each 8 points
			\item[-] Midterm Exam (20 points)\\20 points
			\item[-] Final Exam (20 points)\\20 points
		\end{itemize}
	\end{block}
\end{frame}

\begin{frame}
	\frametitle{Course Administration}
	\begin{block}{Policies}
		Lecture and lab attendance is optional but strongly encouraged. Students are strongly advised to attend both. Quizzes are released online and solutions for which should be submitted online, but students are encouraged to discuss their solutions with teaching assistants during lab sessions. Slides do not reflect all materials covered in lectures.

		University policies on academic standards and cheating, plagiarism and documentation of written work and code of student conduct are strictly enforced.
	\end{block}
\end{frame}

\begin{frame}
	\frametitle{Course Administration}
	\begin{block}{Tips and Tricks}
		\begin{itemize}
			\item[] Prepare yourself by thinking ahead.
			\item[] Get your hands dirty. Code during lectures.
			\item[] Start assignments the day they are posted.
			\item[] Explore different programming languages.
			\item[] Marry Google. Make friends with Wikipedia.
			\item[] Explore Different Editors.
		\end{itemize}
	\end{block}
\end{frame}

\section{Computer Hardware}

\begin{frame}
	\frametitle{Computer Hardware}
	\begin{block}{Key Components}
		\begin{itemize}
			\item[] Central Processing Unit (CPU)
			\item[] Input Output (I/O) Devices
			\item[] Memory
		\end{itemize}
	\end{block}
\end{frame}

\begin{frame}
	\frametitle{Computer Hardware}
	\begin{block}{Key Design Challenges}
		\begin{itemize}
			\item[] Making Processing Units as small and powerful as possible
			\item[] Making Input Output (I/O) Devices as heterogenous as possible
			\item[] Making Memory as abundant as possible
		\end{itemize}
	\end{block}
\end{frame}

\begin{frame}
	\frametitle{Computer Hardware}
	\begin{block}{Basic Operation Procedure}
		\begin{enumerate}
			\item[] Instructions read either from I/O devices or from memory
			\item[] Each statement processed in processor
			\item[] Output passed either to I/O devices or to memory
		\end{enumerate}
	\end{block}
\end{frame}

\begin{frame}
	\frametitle{Computer Hardware}
	\begin{block}{Mircroprocessor Internal Architecture}
		Key Components
		\begin{enumerate}
			\item[] Arithmetic Logic Unit (ALU)
			\item[] Control Logic Station
			\item[] Status Registers
		\end{enumerate}
	\end{block}
\end{frame}

\begin{frame}
	\frametitle{Computer Hardware}
	\begin{block}{Nature of Information}
		Analog vs. Digital Signals

		2-Base Numeral Systems
		\begin{enumerate}
			\item[] Relation to Digital Electronic Circuitry
			\item[] Reliable Distingushing Property
		\end{enumerate}
	\end{block}
\end{frame}

\plain{}{Keep Calm\\and\\Love Programming}

\end{document}
