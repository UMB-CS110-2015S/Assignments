%%%%%%%%%%%%%%%%%%%%%%%%%%%%%%%%%%%%%%%%%%%%%%%%%%%%%%%%%%%%%%%%%%%%%%%%%%%%%%
% CS110: Introduction to Computing
% Copyright 2015 Pejman Ghorbanzade <mail@ghorbanzade.com>
% Creative Commons Attribution-ShareAlike 4.0 International License
% https://github.com/ghorbanzade/UMB-CS110-2015S/blob/master/LICENSE
%%%%%%%%%%%%%%%%%%%%%%%%%%%%%%%%%%%%%%%%%%%%%%%%%%%%%%%%%%%%%%%%%%%%%%%%%%%%%%

\def \topDirectory {.}
\def \texDirectory {\topDirectory/src/main/tex}

\documentclass[12pt,letterpaper,twoside]{article}
\usepackage{\texDirectory/template/style/directives}
\usepackage{\texDirectory/template/style/assignment}
%%%%%%%%%%%%%%%%%%%%%%%%%%%%%%%%%%%%%%%%%%%%%%%%%%%%%%%%%%%%%%%%%%%%%%%%%%%%%%
% CS110: Introduction to Computing
% Copyright 2015 Pejman Ghorbanzade <mail@ghorbanzade.com>
% Creative Commons Attribution-ShareAlike 4.0 International License
% https://github.com/ghorbanzade/UMB-CS110-2015S/blob/master/LICENSE
%%%%%%%%%%%%%%%%%%%%%%%%%%%%%%%%%%%%%%%%%%%%%%%%%%%%%%%%%%%%%%%%%%%%%%%%%%%%%%

\course{id}{CS110}
\course{name}{Introduction to Computing}
\course{venue}{Tue/Thu, 5:30 PM - 6:45 PM}
\course{semester}{Spring 2015}
\course{department}{Department of Computer Science}
\course{university}{University of Massachusetts Boston}

\instructor{name}{Pejman Ghorbanzade}
\instructor{title}{}
\instructor{position}{Student Instructor}
\instructor{email}{pejman@cs.umb.edu}
\instructor{phone}{617-287-6419}
\instructor{office}{S-3-124B}
\instructor{office-hours}{Tue/Thu 19:00-20:30}
\instructor{address}{University of Massachusetts Boston, 100 Morrissey Blvd., Boston, MA}

\setlist{leftmargin=10pt, itemsep=0pt, parsep=0pt}

\begin{document}

\doc{title}{Course Syllabus}
\doc{points}{0}

\prepare{header}

\subsection*{Objectives}
Students will learn basics of computer programming and its fundamentals. Object-Oriented paradaigm is introduced and its design fundamentals are extensively studied. Students will learn know-how of developing simple Java programs addressing diverse computer science applications.

\subsection*{Topics}
\begin{itemize}
\item[] Introduction to Programming Languages
\item[] Fundamentals of Programming Languages
\item[] Object-Oriented Programming
\item[] Object-Oriented Design
\item[] Graphical User Interface
\item[] Data Structures
\end{itemize}

\subsection*{Prerequisites}
Students taking this course should have passed \textit{MA130: Precalculus} or should have scored at least 59\% in the \textit{Math Placement Exam} which is accessible through the \textit{WISER} system of University of Massachusetts Boston.

\subsection*{Venue}
\begin{itemize}
\item[] Classroom: S-1-006 (Small Science Auditorium)
\item[] Weekdays: Tuesday, Thursday
\item[] Time: 17:30 - 18:45
\end{itemize}

\subsection*{Instructor}
\begin{itemize}
\item[] Pejman Ghorbanzade
\item[] Mail Address: \texttt{mail@ghorbanzade.com}
\item[] Office: S-3-124
\item[] Office Hours: Mondays, Wednesdays 16:00 to 17:30
\end{itemize}

\subsection*{Teaching Assistants}
\begin{itemize}
\item[] Yahui Di
\item[] Mail Address: \texttt{yahuidi@cs.umb.edu}
\item[] Office: S-3-158A
\item[] Office Hours: Tuesdays, Thursdays 16:00 to 17:15
\end{itemize}
\begin{itemize}
\item[] Yahui Di
\item[] Mail Address: \texttt{yangmu@cs.umb.edu}
\item[] Office: S-3-158A
\item[] Office Hours: Tuesdays, Thursdays 19:00 to 20:15
\end{itemize}

\subsection*{Grading}
\begin{itemize}
\item[] Quizzes: 20 points
\item[] Homeworks: 40 points
\item[] Midterm Exam: 20 points
\item[] Final Exam: 20 points
\end{itemize}

\subsection*{Important Dates}
\begin{itemize}
\item[] First Lecture: Jan 27, 2015
\item[] Midterm Exam: Mar 12, 2015
\item[] Last Lecture: May 12, 2015
\item[] Final Exam: May 19, 2015
\end{itemize}

\subsection*{Recommended Textbook}
Robert Sedgewick, Kevin Wayne, \textit{Introduction to Programming in Java: An Interdisciplinary Approach}. Addison-Wesley, 2007.

\subsection*{Supplemental Instruction}
As part of the College of Science and Mathematics Freshman Success Program, Supplemental Instruction (SI) is available to all CS110 students free of charge. During the SI sessions, the SI leader will review the material we discuss in class and will also answer any questions you may have regarding concepts or assignments. You may attend as many or as few sessions as you want or feel that you need. Attending these sessions is encouraged.

\subsubsection*{Supplemental Instruction Leader}
\begin{itemize}
\item[] Maya Cheriyan
\item[] Mail Address: \texttt{Maya.Cheriyan001@umb@edu}
\end{itemize}

\subsubsection*{Supplemental Instruction Sessions}
\begin{itemize}
\item[] Mondays, 10:00 to 12:00, S-2-SSC
\item[] Wednesdays, 10:00 to 11:50, M-2-213
\end{itemize}


\subsection*{Attendance Policy}
Lecture and lab attendance is optional but strongly encouraged. Students are strongly advised to attend both. Slides do not reflect all materials covered in lectures. Students are responsible for material covered in any class that they do not attend. Students are also expected to check this website regularly for latest announcements, upcoming events and released homeworks.

\subsection*{Late Submission Policy}
Homeworks may be submitted late by up to two days; the penalty for late submission increasing linearly from 20\% to 100\% of the homework score.

\subsection*{Makeup Policy}
Unless a good reason and its supporting evidence are given - e.g. due to illness or emergency events - no makeup is acceptable for students missing a homework assignment or an exam.

\subsection*{Accomodations}
Section 504 of the Americans with Disabilities Act of 1990 offers guidelines for curriculum modifications and adaptations for students with documented disabilities. If applicable, students may obtain adaptation recommendations from the \textit{Ross Center for Disability Services}, M-1-401, (617-287-7430). The student must present these recommendations and discuss them with each professor within a reasonable period, preferably by the end of Drop/Add period.

\subsection*{Student Conduct}
Students are required to adhere to the University Policy on Academic Standards and Cheating, to the University Statement on Plagiarism and the Documentation of Written Work, and to the \textit{Code of Student Conduct} as delineated in the catalog of Undergraduate Programs, pp. 44-45, and 48-52.

\prepare{footer}

\end{document}
