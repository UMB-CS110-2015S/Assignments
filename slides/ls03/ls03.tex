% -----------------------------------------------------------------------------
% The MIT License (MIT)
%
% Copyright (c) 2015 Pejman Ghorbanzade
%
% Permission is hereby granted, free of charge, to any person obtaining a copy
% of this software and associated documentation files (the "Software"), to deal
% in the Software without restriction, including without limitation the rights
% to use, copy, modify, merge, publish, distribute, sublicense, and/or sell
% copies of the Software, and to permit persons to whom the Software is
% furnished to do so, subject to the following conditions:
%
% The above copyright notice and this permission notice shall be included in
% all copies or substantial portions of the Software.
%
% THE SOFTWARE IS PROVIDED "AS IS", WITHOUT WARRANTY OF ANY KIND, EXPRESS OR
% IMPLIED, INCLUDING BUT NOT LIMITED TO THE WARRANTIES OF MERCHANTABILITY,
% FITNESS FOR A PARTICULAR PURPOSE AND NONINFRINGEMENT. IN NO EVENT SHALL THE
% AUTHORS OR COPYRIGHT HOLDERS BE LIABLE FOR ANY CLAIM, DAMAGES OR OTHER
% LIABILITY, WHETHER IN AN ACTION OF CONTRACT, TORT OR OTHERWISE, ARISING FROM,
% OUT OF OR IN CONNECTION WITH THE SOFTWARE OR THE USE OR OTHER DEALINGS IN
% THE SOFTWARE.
% -----------------------------------------------------------------------------

\def \topDirectory {../..}

\documentclass[10pt, compress]{beamer}

\usepackage{\topDirectory/template/style/directives}
%%%%%%%%%%%%%%%%%%%%%%%%%%%%%%%%%%%%%%%%%%%%%%%%%%%%%%%%%%%%%%%%%%%%%%%%%%%%%%
% CS110: Introduction to Computing
% Copyright 2015 Pejman Ghorbanzade <mail@ghorbanzade.com>
% Creative Commons Attribution-ShareAlike 4.0 International License
% https://github.com/ghorbanzade/UMB-CS110-2015S/blob/master/LICENSE
%%%%%%%%%%%%%%%%%%%%%%%%%%%%%%%%%%%%%%%%%%%%%%%%%%%%%%%%%%%%%%%%%%%%%%%%%%%%%%

\course{id}{CS110}
\course{name}{Introduction to Computing}
\course{venue}{Tue/Thu, 5:30 PM - 6:45 PM}
\course{semester}{Spring 2015}
\course{department}{Department of Computer Science}
\course{university}{University of Massachusetts Boston}

\instructor{name}{Pejman Ghorbanzade}
\instructor{title}{}
\instructor{position}{Student Instructor}
\instructor{email}{pejman@cs.umb.edu}
\instructor{phone}{617-287-6419}
\instructor{office}{S-3-124B}
\instructor{office-hours}{Tue/Thu 19:00-20:30}
\instructor{address}{University of Massachusetts Boston, 100 Morrissey Blvd., Boston, MA}

\usepackage{\topDirectory/template/style/beamerthemeUmassLecture}
\doc{number}{3}
%\setbeamertemplate{footline}[text line]{}

\begin{document}
\prepareCover

\section{Course Administration}

\begin{frame}[fragile]
	\frametitle{Course Administration}
	\begin{itemize}
		\item[] 50/64 student account so far at \href{http://ghorbanzade.com/}{course website}.
		\item[] 12/64 online questionnaire submissions so far.
		\item[] Assignment 1 released. Due on Feb 19 at 5:30 PM.
		\item[] Course Forum now available. Use it. Please.
	\end{itemize}
\end{frame}

\begin{frame}{Overview}
	\frametitle{Course Administration}
	\begin{itemize}
		\item[] Hello Java Program
		\item[] Command-line Arguments
		\item[] Data Types
		\item[] Operators
	\end{itemize}
\end{frame}

\section{"Hello World!"}

\begin{frame}[fragile]
	\frametitle{Hello Java Program}
	\begin{block}{HelloWorld.java}
		\begin{minted}[fontsize=\small,tabsize=8]{java}
			public class HelloWorld {
			    public static void main(String[] args) {
			        System.out.println("Hello World!");
			    }
			}
		\end{minted}
	\end{block}
	\begin{block}{Key Words}
		\begin{columns}
		\begin{column}{0.5\textwidth}
			\begin{itemize}
				\item[] Classes
				\item[] Methods
			\end{itemize}
		\end{column}
		\begin{column}{0.5\textwidth}
			\begin{itemize}
				\item[] Statements
				\item[] Arguments
			\end{itemize}
		\end{column}
		\end{columns}
	\end{block}
\end{frame}

\begin{frame}[fragile]
	\frametitle{Hello Java Program}
	\begin{block}{HelloWorld.java}
		\begin{minted}[fontsize=\small,tabsize=8]{java}
			public class HelloWorld {
			    public static void main(String[] args) {
			        System.out.println("Hello World!");
			    }
			}
		\end{minted}
	\end{block}
	\begin{block}{Terminal}
		\begin{minted}[fontsize=\small,tabsize=8]{bash}
			\$ javac HelloWorld.java
			\$ java HelloWorld
			Hello World!
		\end{minted}
	\end{block}
\end{frame}

\section{Command-line Arguments}

\begin{frame}[fragile]
	\frametitle{Command-line Arguments}
	\begin{block}{SayHello.java}
		\begin{minted}[fontsize=\small,tabsize=8]{java}
			public class SayHello {
			    public static void main(String[] args) {
			        System.out.println("Hello! How are you?");
			    }
			}
		\end{minted}
	\end{block}
	\begin{block}{Terminal}
		\begin{minted}[fontsize=\small,tabsize=8]{bash}
			\$ javac SayHello.java
			\$ java SayHello
			Hello! How are you?
		\end{minted}
	\end{block}
\end{frame}

\begin{frame}[fragile]
	\frametitle{Command-line Arguments}
	\begin{block}{SayHello.java}
		\begin{minted}[fontsize=\small,tabsize=8]{java}
			public class SayHello {
			    public static void main(String[] args) {
			        System.out.println("Hello " + args[0] + "! ");
			        System.out.println("How are you?");
			    }
			}
		\end{minted}
	\end{block}
	\begin{block}{Terminal}
		\begin{minted}[fontsize=\small,tabsize=8]{bash}
			\$ javac SayHello.java
			\$ java SayHello Pejman
			Hello Pejman! How are you?
			\$ java SayHello 12@#1
			Hello 12@#1! How are you?
		\end{minted}
	\end{block}
\end{frame}

\begin{frame}[fragile]
	\frametitle{Command-line Arguments}
	\begin{block}{Motivation}
		\begin{itemize}
			\item[] Efficient Interthread Communication in JVM by means of piping and redirection
		\end{itemize}
	\end{block}
	\begin{block}{Advantages}
		\begin{itemize}
			\item[] Provides a way to give input to program
			\item[] Simple to begin with
		\end{itemize}
	\end{block}
	\begin{block}{Disadvantages}
		\begin{itemize}
			\item[] All user-specified arguments to be given before execution of program
			\item[] All arguments are of type String
		\end{itemize}
	\end{block}
\end{frame}

\section{Data Types}

\begin{frame}[fragile]
	\frametitle{Data Types}
	\begin{block}{Variables}
		A variable is a name for a location in memory used to hold a data value. All variables in java should be declared. Once.
		\begin{minted}[fontsize=\small,tabsize=8]{java}
			int a = 123;
			int b = 99;
			int c = a + b;
			a = 25;
		\end{minted}
	\end{block}
	\begin{block}{Constants}
		A constant is a special variable that holds a data whose value does not change throughout the program.
		\begin{minted}[fontsize=\small,tabsize=8]{java}
			final int d = 2015;
		\end{minted}
	\end{block}
\end{frame}

\begin{frame}[fragile]
	\frametitle{Data Types}
	\begin{block}{Identifiers}
		An identifier is a sequence of letters, digits, underscore signs and dollar signs; the first of which is not a digit.
		\begin{minted}[fontsize=\small,tabsize=8]{java}
			int HeLlO = 123;
			int $_Q2d$o = 99;
			int _ASc = HeLlO + $_Q2d$o;
			$_Q2d$o = 25;
		\end{minted}
	\end{block}
	\begin{itemize}
		\item[] Identifiers are case sensitive.
		\item[] You cannot use Java \textbf{reserved} words for identifiers
		\item[] Using dollar and underscore signs are discouraged.
	\end{itemize}
\end{frame}

\begin{frame}[fragile]
	\frametitle{Data Types}
	Java is statically-typed: all \textbf{variables} must first be declared.
	\begin{block}{Primitive Data Types}
		\begin{columns}
		\begin{column}{0.5\textwidth}
			\begin{itemize}
				\item[] boolean
				\item[] byte
				\item[] short
				\item[] int
			\end{itemize}
		\end{column}
		\begin{column}{0.5\textwidth}
			\begin{itemize}
				\item[] long
				\item[] float
				\item[] double
				\item[] char
			\end{itemize}
		\end{column}
		\end{columns}
	\end{block}
	\begin{block}{Built-in Data Types}
		\begin{itemize}
			\item[] Primitive Data Types
			\item[] String
		\end{itemize}
	\end{block}
\end{frame}

\begin{frame}[fragile]
	\frametitle{Data Types}
	\begin{block}{Declaration}
		A declaration associates a variable with a type at compile time.

		Declaration should be made before first use of variable.
		\begin{minted}[fontsize=\small,tabsize=8]{java}
			int a, b;
			a = 13;
			b = 25;
			int c = 10;
			int d = b - c;
			d = d - c;
		\end{minted}
	\end{block}
\end{frame}

\begin{frame}[fragile]
	\frametitle{Data Types}
	\begin{block}{Declaration}
		Declared data type should be consistent with the value assigned.
		\begin{minted}[fontsize=\small,tabsize=8]{java}
			boolean snowDay = false;
			byte numStudentClass = 64;
			short numStudentTotal = 15741;
			int populationWorld = 318881992;
			long gangnamYoutube = 2238643747;
			float snowHeight = 19.75;
			double avogadro = 6.022e23;
			char letter = 'Q';
			String word = "Beacons";
		\end{minted}
	\end{block}
\end{frame}

\begin{frame}[fragile]
	\frametitle{Data Types}
	\begin{block}{Assignment}
		An assignment associates a data type value with a variable.
		\begin{minted}[fontsize=\small,tabsize=8]{java}
			double discriminant = b*b - 4*a*c;
		\end{minted}
		The meaning of = is decidedly \textbf{not} the same as in mathematical equations.
		\begin{minted}[fontsize=\small,tabsize=8]{java}
			int a, b, c;
			a = 23;
			b = 25;
			c = a;
			a = b;
			b = c;
		\end{minted}
	\end{block}
\end{frame}

\section{Operators}

\begin{frame}[fragile]
	\frametitle{Operators}
	An operation is an action performed on one or more values either to modify the value or to produce a new one.
	\begin{block}{Classification}
		\begin{columns}
			\begin{column}{0.5\textwidth}
				\begin{itemize}
					\item[] Arithmetic Operators
					\item[] Relational Operators
					\item[] Bitwise Operators
				\end{itemize}
			\end{column}
			\begin{column}{0.5\textwidth}
				\begin{itemize}
					\item[] Logical Operators
					\item[] Assignment Operators
					\item[] Ternary Operators
				\end{itemize}
			\end{column}
		\end{columns}
	\end{block}
\end{frame}

\begin{frame}[fragile]
	\frametitle{Operators}
	\begin{block}{Arithmetic Operators}
		\begin{minted}[fontsize=\small,tabsize=8]{java}
			int A = 10;
			int B = 25;
		\end{minted}
		\begin{table}
			\begin{tabular}{lcc}
				\toprule
				Operator & Example & Output\\
				\midrule
				Addition & A + B & 35 \\
				Subtraction & A - B & -15\\
				Multiplication & A * B & 250\\
				Division & B / A & 2\\
				Modulus & B \% A & 5\\
				Increment & A++ & 11\\
				Decrement & B-{}- & 24\\
				\bottomrule
			\end{tabular}
			\caption{Arithmetic Operators}
		\end{table}
	\end{block}
\end{frame}

\begin{frame}[fragile]
	\frametitle{Operators}
	\begin{block}{Relational Operators}
		\begin{minted}[fontsize=\small,tabsize=8]{java}
			int A = 10;
			int B = 25;
		\end{minted}
		\begin{table}
			\begin{tabular}{lcc}
				\toprule
				Operator & Example & Output\\
				\midrule
				Equality & A == B & false\\
				Inequality & A != B & true\\
				Greater than & A > B & false\\
				Less than & A < B & true\\
				Greater than or equal to & A >= B & false\\
				Less than or equal to & A <= B & true\\
				\bottomrule
			\end{tabular}
			\caption{Relational Operators}
		\end{table}
	\end{block}
\end{frame}

\begin{frame}[fragile]
	\frametitle{Operators}
	\begin{block}{Bitwise Operators}
		\begin{minted}[fontsize=\small,tabsize=8]{java}
			int A = 0b11111111;
			int B = 0b00001010;
		\end{minted}
		\begin{table}
			\begin{tabular}{lcc}
				\toprule
				Operator & Example & Output\\
				\midrule
				Binary AND & A \& B & 0b00001010\\
				Binary OR & A | B & 0b11111111\\
				Binary XOR & A \textasciicircum B & 0b11110101\\
				Binary Ones Complement & \textasciitilde B & 0b11110101\\
				Binary Left Shift & B <{}< 2 & 0b00101000\\
				Binary Right Shift & B >{}> 2 & 0b10000010\\
				Shift Right Zero Fill & A >{}>{}> 2 & 0b00000010 \\
				\bottomrule
			\end{tabular}
			\caption{Bitwise Operators}
		\end{table}
	\end{block}
\end{frame}

\begin{frame}[fragile]
	\frametitle{Operators}
	\begin{block}{Logical Operators}
		\begin{minted}[fontsize=\small,tabsize=8]{java}
			boolean A = true;
			boolean B = false;
		\end{minted}
		\begin{table}
			\begin{tabular}{lcc}
				\toprule
				Operator & Example & Output\\
				\midrule
				Logical AND & A \&\& B & false\\
				Logical OR & A || B & true\\
				Logical NOT & !B & true\\
				\bottomrule
			\end{tabular}
			\caption{Logical Operators}
		\end{table}
	\end{block}
\end{frame}

\plain{}{Keep Calm\\and\\Learn Java}

\end{document}
