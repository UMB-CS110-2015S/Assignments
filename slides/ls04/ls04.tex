% -----------------------------------------------------------------------------
% The MIT License (MIT)
%
% Copyright (c) 2015 Pejman Ghorbanzade
%
% Permission is hereby granted, free of charge, to any person obtaining a copy
% of this software and associated documentation files (the "Software"), to deal
% in the Software without restriction, including without limitation the rights
% to use, copy, modify, merge, publish, distribute, sublicense, and/or sell
% copies of the Software, and to permit persons to whom the Software is
% furnished to do so, subject to the following conditions:
%
% The above copyright notice and this permission notice shall be included in
% all copies or substantial portions of the Software.
%
% THE SOFTWARE IS PROVIDED "AS IS", WITHOUT WARRANTY OF ANY KIND, EXPRESS OR
% IMPLIED, INCLUDING BUT NOT LIMITED TO THE WARRANTIES OF MERCHANTABILITY,
% FITNESS FOR A PARTICULAR PURPOSE AND NONINFRINGEMENT. IN NO EVENT SHALL THE
% AUTHORS OR COPYRIGHT HOLDERS BE LIABLE FOR ANY CLAIM, DAMAGES OR OTHER
% LIABILITY, WHETHER IN AN ACTION OF CONTRACT, TORT OR OTHERWISE, ARISING FROM,
% OUT OF OR IN CONNECTION WITH THE SOFTWARE OR THE USE OR OTHER DEALINGS IN
% THE SOFTWARE.
% -----------------------------------------------------------------------------

\def \topDirectory {../..}

\documentclass[10pt, compress]{beamer}

\usepackage{\topDirectory/template/style/directives}
%%%%%%%%%%%%%%%%%%%%%%%%%%%%%%%%%%%%%%%%%%%%%%%%%%%%%%%%%%%%%%%%%%%%%%%%%%%%%%
% CS110: Introduction to Computing
% Copyright 2015 Pejman Ghorbanzade <mail@ghorbanzade.com>
% Creative Commons Attribution-ShareAlike 4.0 International License
% https://github.com/ghorbanzade/UMB-CS110-2015S/blob/master/LICENSE
%%%%%%%%%%%%%%%%%%%%%%%%%%%%%%%%%%%%%%%%%%%%%%%%%%%%%%%%%%%%%%%%%%%%%%%%%%%%%%

\course{id}{CS110}
\course{name}{Introduction to Computing}
\course{venue}{Tue/Thu, 5:30 PM - 6:45 PM}
\course{semester}{Spring 2015}
\course{department}{Department of Computer Science}
\course{university}{University of Massachusetts Boston}

\instructor{name}{Pejman Ghorbanzade}
\instructor{title}{}
\instructor{position}{Student Instructor}
\instructor{email}{pejman@cs.umb.edu}
\instructor{phone}{617-287-6419}
\instructor{office}{S-3-124B}
\instructor{office-hours}{Tue/Thu 19:00-20:30}
\instructor{address}{University of Massachusetts Boston, 100 Morrissey Blvd., Boston, MA}

\usepackage{\topDirectory/template/style/beamerthemeUmassLecture}
\doc{number}{4}
%\setbeamertemplate{footline}[text line]{}

\begin{document}
\prepareCover

\section{Course Administration}

\begin{frame}[fragile]
\frametitle{Course Administration}
	59/64 student account so far at \href{http://ghorbanzade.com/}{course website}.

	29/64 online questionnaire submissions so far.

	Due date for assignment 1 extended to Feb 21 at 11:59 PM.

	Assignment 2 released. Due on Mar 03 at 17:30 PM.
\end{frame}

\begin{frame}[fragile]
	\frametitle{Overview}
	\begin{itemize}
		\item[] Class PrintStream
		\item[] Class Math
		\item[] Class String
		\item[] Data Conversion
	\end{itemize}
\end{frame}

\section{Class PrintStream}

\begin{frame}[fragile]
	\frametitle{Class PrintStream}
	\begin{block}{HelloWorld.java}
		\begin{minted}[fontsize=\small,tabsize=8]{java}
			public class HelloWorld {
				public static void main(String[] args) {
					System.out.println("Hello World!");
					System.out.print("Hello World!\n");
					System.out.printf("Hello %s!\n","World");
				}
			}
		\end{minted}
	\end{block}
\end{frame}

\begin{frame}[fragile]
	\frametitle{Class PrintStream}
	String is not a primitive data type. Yet, String concatenation operation is a built-in Java feature.
	\begin{block}{HelloStudents.java}
	\begin{minted}[fontsize=\small,tabsize=8]{java}
	public class HelloStudents {
		public static void main(String[] args) {
			System.out.println("Hello CS110 Students!");
			System.out.println("Hello " + "CS110" + " Students!");
			int code = 110;
			System.out.println("Hello CS" + code + " Students!");
			String message = "Hello CS110 Students!";
			System.out.println(message);
		}
	}
	\end{minted}
	\end{block}
\end{frame}

\begin{frame}[fragile]
	\frametitle{Class PrintStream}
	\begin{block}{PrintStream}
		PrintStream is a built-in class in java.io package. It contains pre-defined methods and one field (out) to print computer output.
	\end{block}
	\begin{block}{System.out.println()}
		\begin{itemize}
			\item[] \texttt{System} is a built-in class in java.lang package. It contains pre-defined methods and fields.
			\item[] \texttt{out} is a field in System class and a reference of PrintStream class.
			\item[] \texttt{println()} is a method of Printstream class.
		\end{itemize}
	\end{block}
\end{frame}

\begin{frame}[fragile]
	\frametitle{Class PrintStream}
	println() is not the only method of PrintStream class.
	\begin{block}{HelloUniverse.java}
	\begin{minted}[fontsize=\small,tabsize=8]{java}
	public class HelloUniverse {
		public static void main(String[] args) {
			System.out.print("Hello \'Universe\'!\n");
			System.out.print("Hello " + "\"Univer" + "se\"!\n");
			System.out.print("Hello \t Universe!");
			System.out.print("Hello \n Universe!\n");
		}
	}
	\end{minted}
	\end{block}
\end{frame}

\begin{frame}[fragile]
	\frametitle{Class PrintStream}
	printf() is a more powerful method of PrintStream class.
	\begin{block}{HelloNumber.java}
	\begin{minted}[fontsize=\small,tabsize=8]{java}
	public class HelloNumber {
		public static void main(String[] args) {
			double num = 142.2567;
			System.out.printf("Hello " + num + "!\n");
			System.out.printf("Hello %f!\n",num);
			System.out.printf("Hello %4.2f!\n",num);
			System.out.printf("Hello %08.f!",num);
		}
	}
	\end{minted}
	\end{block}
\end{frame}

\section{Class Math}

\begin{frame}[fragile]
	\frametitle{Class Math}
	Class Math provides methods to perform basic numeric operations, efficiently.
	\begin{block}{Fields}
		\begin{columns}
			\begin{column}{0.5\textwidth}
				\begin{itemize}
					\item[] Neperian Number
				\end{itemize}
			\end{column}
			\begin{column}{0.5\textwidth}
				\begin{itemize}
					\item[] Pi Number
				\end{itemize}
			\end{column}
		\end{columns}
	\end{block}
	\begin{block}{Methods}
		\begin{columns}
			\begin{column}{0.5\textwidth}
				\begin{itemize}
					\item[] abs(a)
					\item[] floor(a)
					\item[] ceil(a)
					\item[] round(a)
					\item[] sqrt(a)
					\item[] cbrt(a)
				\end{itemize}
			\end{column}
			\begin{column}{0.5\textwidth}
				\begin{itemize}
					\item[] sin(a)
					\item[] asin(a)
					\item[] sinh(a)
					\item[] cos(a)
					\item[] acos(a)
					\item[] cosh(a)
				\end{itemize}
			\end{column}
		\end{columns}
	\end{block}
\end{frame}

\begin{frame}[fragile]
	\frametitle{Class Math}
	\begin{block}{Methods (cont'd)}
		\begin{columns}
			\begin{column}{0.5\textwidth}
				\begin{itemize}
					\item[] exp(a)
					\item[] log(a)
					\item[] tan(a)
					\item[] atan(a)
					\item[] atan2(a)
					\item[] tanh(a)
				\end{itemize}
			\end{column}
			\begin{column}{0.5\textwidth}
				\begin{itemize}
					\item[] max(a, b)
					\item[] min(a, b)
					\item[] pow(a, b)
					\item[] random()
					\item[] toRadians(a)
					\item[] toDegrees(a)
				\end{itemize}
			\end{column}
		\end{columns}
	\end{block}
	\begin{block}{Usage}
		\begin{minted}[fontsize=\small,tabsize=8]{java}
			double randomNum = Math.random();
			int upperBound = Math.ceil(randomNum);
			int lowerBound = Math.floor(randomNum);
		\end{minted}
	\end{block}
\end{frame}

\begin{frame}[fragile]
	\frametitle{Class Math}
	\begin{block}{How random is Math.random()?}
		Method Math.random() is a little biased. We will discuss better alternatives later.
	\end{block}
	\begin{block}{Method Math.random()}
		\begin{minted}[fontsize=\small,tabsize=8]{java}
			double randomNum1 = Math.random();
		\end{minted}
		Output is always less than or equal to zero and less than one. What if we want a random number between $x$ and $y$?
		\begin{minted}[fontsize=\small,tabsize=8]{java}
			double randomNum2 = x + (y-x) * Math.random();
		\end{minted}
	\end{block}
\end{frame}

\section{Class String}

\begin{frame}[fragile]
	\frametitle{Class String}
	String is a class and a built-in data type. It provides methods to modify, handle or analyze string literals. Every string literal is an instance (object) of class String.
	\begin{block}{Methods}
		\begin{columns}
			\begin{column}{0.5\textwidth}
				\begin{itemize}
					\item[] length()
					\item[] trim()
					\item[] startsWith()
					\item[] endsWith()
					\item[] charAt()
					\item[] indexOf()
					\item[] lastIndexOf()
				\end{itemize}
			\end{column}
			\begin{column}{0.5\textwidth}
				\begin{itemize}
					\item[] equals()
					\item[] equalsIgnoreCase()
					\item[] compareTo()
					\item[] compareToIgnoreCase()
					\item[] contains()
					\item[] concat()
					\item[] substring()
				\end{itemize}
			\end{column}
		\end{columns}
	\end{block}
\end{frame}

\begin{frame}[fragile]
	\frametitle{Class String}
	\begin{block}{Methods (cont'd)}
		\begin{columns}
			\begin{column}{0.5\textwidth}
				\begin{itemize}
					\item[] isEmpty()
					\item[] toUpperCase()
					\item[] toLowerCase()
					\item[] join()
					\item[] match()
				\end{itemize}
			\end{column}
			\begin{column}{0.5\textwidth}
				\begin{itemize}
					\item[] valueOf()
					\item[] format()
					\item[] split()
					\item[] replace()
				\end{itemize}
			\end{column}
		\end{columns}
	\end{block}
	\begin{block}{Remember}
		String literals are objects. It is not a good idea to compare them with ={}= relational operator. It is wise to use equals() method instead.
	\end{block}
\end{frame}

\section{Data Conversion}

\begin{frame}[fragile]
	\frametitle{Data Conversion}
	\begin{itemize}
		\item[] All variables should be declared only once.
		\item[] Variables of a declared type can be assigned values of that type.
	\end{itemize}
	\begin{block}{Converting Primitive Types to String}
		\begin{minted}[fontsize=\small,tabsize=8]{java}
			int a = 64;
			double b = 124.52;
			boolean c = false;
			String strA = String.valueOf(a);
			String strB = String.valueOf(b);
			String strC = String.valueOf(c);
		\end{minted}
	\end{block}
\end{frame}

\begin{frame}[fragile]
	\frametitle{Data Conversion}
	\begin{block}{Explicit Type Conversion}
		\begin{minted}[fontsize=\small,tabsize=8]{java}
			String strA = "64";
			String strB = "124.52";
			String strC = "false";
			int a = Integer.parseInt(strA);
			double b = Double.parseDouble(strB);
			boolean c = Boolean.parseBoolean(strC);
		\end{minted}
		Many methods perform a type conversion by receiving input of one type and giving output of another type.
	\end{block}
\end{frame}

\begin{frame}[fragile]
	\frametitle{Data Conversion}
	\begin{block}{Explicit Type Casting}
		\begin{minted}[fontsize=\small,tabsize=8]{java}
			double a = 124.52;
			int b = (int) a;
			int c = (int) a + b;
			int d = (int) (a + b);
		\end{minted}
		Type casting has higher precedance than other operations. Beware! Type casting may involve information loss.
	\end{block}
\end{frame}

\begin{frame}[fragile]
	\frametitle{Data Conversion}
	\begin{block}{Automatic Promotion}
		\begin{minted}[fontsize=\small,tabsize=8]{java}
			int a = 5;
			int b = -15;
			int c = 9;
			double discriminant = Math.pow(b,2) - 4 * a *c;
		\end{minted}
		Java automatically promotes data types where it assumes we have been light-headed. In above example, variables a, b and c will first be converted to type double and then other operations are performed. Automatic promotion is safe as it doesn't involve information loss.
	\end{block}
\end{frame}

\plain{}{Keep Calm\\and\\Practice}

\end{document}
