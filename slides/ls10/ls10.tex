% -----------------------------------------------------------------------------
% The MIT License (MIT)
%
% Copyright (c) 2015 Pejman Ghorbanzade
%
% Permission is hereby granted, free of charge, to any person obtaining a copy
% of this software and associated documentation files (the "Software"), to deal
% in the Software without restriction, including without limitation the rights
% to use, copy, modify, merge, publish, distribute, sublicense, and/or sell
% copies of the Software, and to permit persons to whom the Software is
% furnished to do so, subject to the following conditions:
%
% The above copyright notice and this permission notice shall be included in
% all copies or substantial portions of the Software.
%
% THE SOFTWARE IS PROVIDED "AS IS", WITHOUT WARRANTY OF ANY KIND, EXPRESS OR
% IMPLIED, INCLUDING BUT NOT LIMITED TO THE WARRANTIES OF MERCHANTABILITY,
% FITNESS FOR A PARTICULAR PURPOSE AND NONINFRINGEMENT. IN NO EVENT SHALL THE
% AUTHORS OR COPYRIGHT HOLDERS BE LIABLE FOR ANY CLAIM, DAMAGES OR OTHER
% LIABILITY, WHETHER IN AN ACTION OF CONTRACT, TORT OR OTHERWISE, ARISING FROM,
% OUT OF OR IN CONNECTION WITH THE SOFTWARE OR THE USE OR OTHER DEALINGS IN
% THE SOFTWARE.
% -----------------------------------------------------------------------------

\def \topDirectory {../..}

\documentclass[10pt, compress]{beamer}

\usepackage{\topDirectory/template/style/directives}
%%%%%%%%%%%%%%%%%%%%%%%%%%%%%%%%%%%%%%%%%%%%%%%%%%%%%%%%%%%%%%%%%%%%%%%%%%%%%%
% CS110: Introduction to Computing
% Copyright 2015 Pejman Ghorbanzade <mail@ghorbanzade.com>
% Creative Commons Attribution-ShareAlike 4.0 International License
% https://github.com/ghorbanzade/UMB-CS110-2015S/blob/master/LICENSE
%%%%%%%%%%%%%%%%%%%%%%%%%%%%%%%%%%%%%%%%%%%%%%%%%%%%%%%%%%%%%%%%%%%%%%%%%%%%%%

\course{id}{CS110}
\course{name}{Introduction to Computing}
\course{venue}{Tue/Thu, 5:30 PM - 6:45 PM}
\course{semester}{Spring 2015}
\course{department}{Department of Computer Science}
\course{university}{University of Massachusetts Boston}

\instructor{name}{Pejman Ghorbanzade}
\instructor{title}{}
\instructor{position}{Student Instructor}
\instructor{email}{pejman@cs.umb.edu}
\instructor{phone}{617-287-6419}
\instructor{office}{S-3-124B}
\instructor{office-hours}{Tue/Thu 19:00-20:30}
\instructor{address}{University of Massachusetts Boston, 100 Morrissey Blvd., Boston, MA}

\usepackage{\topDirectory/template/style/beamerthemeUmassLecture}
\usepackage[school]{\topDirectory/template/style/pgf-umlcd}
\doc{number}{10}
%\setbeamertemplate{footline}[text line]{}

\begin{document}
\prepareCover

\section{Course Administration}

\begin{frame}[fragile]
\frametitle{Course Administration}
Quiz 3(b) Released. Due on Apr 03, 2015 at 23:00 PM.

Assignment 3 deadline extended to Apr 05, 2015 at 23:59 PM.
\end{frame}

\begin{frame}[fragile]
  \frametitle{Overview}
  \begin{itemize}
    \item[] Inheritance
    \begin{itemize}
      \item[] Motivation
      \item[] Implementation
      \item[] Constructor Chaining
      \item[] Method Overriding
    \end{itemize}
  \end{itemize}
\end{frame}

\plain{}{Inheritance}

\section{Motivation}

\begin{frame}[fragile]
  \frametitle{Inheritance}
  \begin{block}{Objective}
  Write a program \texttt{Geometry.java} in which you can create cricles and squares and get their area, perimeter, color and whether they are filled or not.
  \end{block}
\end{frame}

\begin{frame}[fragile]
  \frametitle{Inheritance}
  \begin{block}{Approach}
    \begin{itemize}
      \item[] Identify objects
      \item[] Identify states and behaviors of objects
      \item[] Make blueprints of objects (Classes)
      \item[] Instantiate objects from blueprints
      \item[] Use objects to achieve programming goal
    \end{itemize}
  \end{block}
\end{frame}

\begin{frame}[fragile]
  \frametitle{Inheritance}
  \begin{block}{Object-Oriented Approach (v1.0)}
  \begin{description}
    \item[Objects of Concern] myCircle
    \item[States of Objects] color, filled, radius
    \item[Behavior of Objects] getArea(), getPerimeter()
    \item[Required Classes] Circle
  \end{description}
  \end{block}
\end{frame}

\begin{frame}[fragile]
  \frametitle{Inheritance}
  \begin{block}{Object-Oriented Approach (v1.0)}
  \begin{description}
    \item[Objects of Concern] mySquare
    \item[States of Objects] color, filled, length
    \item[Behavior of Objects] getArea(), getPerimeter()
    \item[Required Classes] Square
  \end{description}
  \end{block}
\end{frame}

\begin{frame}[fragile]
  \frametitle{Inheritance}
  \begin{block}{\texttt{Circle.java} (v1.0) (Part 1)}
    \begin{minted}[fontsize=\small,tabsize=8, linenos, firstnumber=1]{java}
public class Circle {
  // fields
  private static int numCircles = 0;
  // attributes
  private String color;
  private boolean filled;
  private double radius;
  // methods
  public double getArea() {
    return Math.PI*Math.pow(radius,2);
  }
  public double getPerimeter() {
    return 2*Math.PI*radius;
  }
    \end{minted}
  \end{block}
\end{frame}

\begin{frame}[fragile]
  \frametitle{Inheritance}
  \begin{block}{\texttt{Circle.java} (v1.0) (Part 2)}
    \begin{minted}[fontsize=\small,tabsize=8, linenos, firstnumber=15]{java}
  // constructors
  public Circle(double someRadius) {
    radius = someRadius;
    numCircles++;
  }
  // setter and getter for color
  public void setColor(String someColor) {
    this.color = someColor;
  }
  public String getColor() {
    return color;
  }
    \end{minted}
  \end{block}
\end{frame}

\begin{frame}[fragile]
  \frametitle{Inheritance}
  \begin{block}{\texttt{Circle.java} (v1.0) (Part 3)}
    \begin{minted}[fontsize=\small,tabsize=8, linenos, firstnumber=27]{java}
  // setter and getter for filled
  public void setFilled(boolean state) {
    this.filled = state;
  }
  public boolean isFilled() {
    return filled;
  }
  // setter and getter for radius
  public void setRadius(double someRadius) {
    this.radius = someRadius;
  }
  public double getRadius() {
    return radius;
  }
}
    \end{minted}
  \end{block}
\end{frame}

\begin{frame}[fragile]
  \frametitle{Inheritance}
  \begin{block}{\texttt{Square.java} (v1.0) (Part 1)}
    \begin{minted}[fontsize=\small,tabsize=8, linenos, firstnumber=1]{java}
public class Square {
  // fields
  private static int numSquares = 0;
  // attributes
  private String color;
  private boolean filled;
  private double length;
  // methods
  public double getArea() {
    return Math.pow(length,2);
  }
  public double getPerimeter() {
    return length*4;
  }
    \end{minted}
  \end{block}
\end{frame}

\begin{frame}[fragile]
  \frametitle{Inheritance}
  \begin{block}{\texttt{Square.java} (v1.0) (Part 2)}
    \begin{minted}[fontsize=\small,tabsize=8, linenos, firstnumber=15]{java}
  // constructors
  public Square(double someLength) {
    length = someLength;
    numSquares++;
  }
  // setter and getter for color
  public void setColor(String someColor) {
    this.color = someColor;
  }
  public String getColor() {
    return color;
  }
    \end{minted}
  \end{block}
\end{frame}

\begin{frame}[fragile]
  \frametitle{Inheritance}
  \begin{block}{\texttt{Square.java} (v1.0) (Part 3)}
    \begin{minted}[fontsize=\small,tabsize=8, linenos, firstnumber=27]{java}
  // setter and getter for filled
  public void setFilled(boolean state) {
    this.filled = state;
  }
  public boolean isFilled() {
    return filled;
  }
  // setter and getter for length
  public void setLength(double someLength) {
    this.length = someLength;
  }
  public double getLength() {
    return length;
  }
}
    \end{minted}
  \end{block}
\end{frame}

\begin{frame}[fragile]
  \frametitle{Inheritance}
  \begin{block}{\texttt{Geometry.java} (v1.0)}
    \begin{minted}[fontsize=\small,tabsize=8, linenos, firstnumber=1]{java}
public class Geometry {
  public static void main(String[] args) {
    Circle myCircle = new Circle(5);
    myCircle.setColor("Red");
    myCircle.setFilled(true);
    double result1 = myCircle.getPerimeter();
    System.out.println("Perimeter of circle: "+result1);
    Square mySquare = new Square(4);
    mySquare.setColor("Blue");
    mySquare.setFilled(false);
    double result2 = mySquare.getArea();
    System.out.println("Area of square: "+result2);
  }
}
    \end{minted}
  \end{block}
\end{frame}

\begin{frame}[fragile]
  \frametitle{Inheritance}
  \begin{block}{Problem Statement}
  Common attributes and methods are developed twice.
  \begin{itemize}
    \item[] Unnecessary Redundancies
    \item[] Difficult to Develop
    \item[] Difficult to Modify
    \item[] Difficult to Troubleshoot
  \end{itemize}
  \end{block}
\end{frame}

\begin{frame}[fragile]
  \frametitle{Inheritance}
  \begin{block}{Proposed Solution}
    Develop additional classes to cover common data members.
  \end{block}
  \begin{block}{Advantages}
  \begin{description}
    \item[Reusability] Methods are inherited from base class
    \item[Extensibility] Sub class may develop its own methods
    \item[Data Hiding] Base class has control over its data
    \item[Overriding] Sub class has control over how methods are inherited
  \end{description}
  \end{block}
\end{frame}

\section{Implementation}

\begin{frame}[fragile]
  \frametitle{Inheritance}
  \begin{block}{Approach}
    \begin{itemize}
      \item[] Identify objects
      \item[] Identify data members for objects
      \item[] Identify common data members
      \item[] Develop base classes
      \item[] Derive sub classes for objects
      \item[] Instantiate objects from blueprints
      \item[] Use objects to achieve programming goal
    \end{itemize}
  \end{block}
\end{frame}

\begin{frame}[fragile]
  \frametitle{Inheritance}
  \begin{block}{Object-Oriented Approach (v2.0)}
  Identify data members for objects
    \begin{itemize}
      \item[] \textbf{Objects of Concern}\\myCircle
      \item[] \textbf{States of Objects}\\color, filled, radius
      \item[] \textbf{Behavior of Objects}\\getArea(), getPerimeter()
      \item[] \textbf{Required Classes}\\Circle
    \end{itemize}
  \end{block}
\end{frame}

\begin{frame}[fragile]
  \frametitle{Inheritance}
  \begin{block}{Object-Oriented Approach (v2.0)}
  Identify data members for objects
    \begin{itemize}
      \item[] \textbf{Objects of Concern}\\mySquare
      \item[] \textbf{States of Objects}\\color, filled, length
      \item[] \textbf{Behavior of Objects}\\getArea(), getPerimeter()
      \item[] \textbf{Required Classes}\\Square
    \end{itemize}
  \end{block}
\end{frame}

\begin{frame}[fragile]
  \frametitle{Inheritance}
  \begin{block}{Object-Oriented Approach (v2.0)}
  Identify common data members
  \begin{itemize}
    \item[] \texttt{Common States} color, filled
    \item[] \texttt{Common Behaviors} none
    \item[] \texttt{Required Classes} Shape
  \end{itemize}
  \end{block}
\end{frame}

\begin{frame}[fragile]
  \frametitle{Inheritance}
  \begin{block}{UML Diagram for Class \texttt{Circle}}
  \begin{figure}
  \centering
    \begin{tikzpicture}
      \begin{class}[]{Circle}{0, 0}
        \attribute{- numCircles: int}
        \attribute{- radius: double}
        \operation{+ Circle()}
        \operation{+ getPerimeter(): double}
        \operation{+ getArea(): double}
        \operation{+ getNumCircles(): int}
        \operation{+ getRadius(): double}
        \operation{+ setRadius(double radius)}
      \end{class}
    \end{tikzpicture}
  \end{figure}
  \end{block}
\end{frame}

\begin{frame}[fragile]
  \frametitle{Inheritance}
  \begin{block}{UML Diagram for Class \texttt{Square}}
  \begin{figure}
  \centering
    \begin{tikzpicture}
      \begin{class}[]{Square}{0, 0}
        \attribute{- numSquares: int}
        \attribute{- length: double}
        \operation{+ Square()}
        \operation{+ getPerimeter(): double}
        \operation{+ getArea(): double}
        \operation{+ getNumSquares(): int}
        \operation{+ getLength(): double}
        \operation{+ setLength(double radius)}
      \end{class}
    \end{tikzpicture}
  \end{figure}
  \end{block}
\end{frame}

\begin{frame}[fragile]
  \frametitle{Inheritance}
  \begin{block}{UML Diagram for Class \texttt{Shape}}
  \begin{figure}
  \centering
    \begin{tikzpicture}
      \begin{class}[]{Shape}{0, 0}
        \attribute{- numShapes: int}
        \attribute{- filled: boolean}
        \attribute{- color: String}
        \operation{+ Shape()}
        \operation{+ getNumShape(): int}
        \operation{+ isFilled(): boolean}
        \operation{+ setFilled(boolean filled)}
        \operation{+ getColor(): String}
        \operation{+ setColor(String color)}
      \end{class}
    \end{tikzpicture}
  \end{figure}
  \end{block}
\end{frame}

\begin{frame}[fragile]
  \frametitle{Inheritance}
  \begin{block}{\texttt{Shape.java} (v2.0) (Part 1)}
    \begin{minted}[fontsize=\small,tabsize=8, linenos, firstnumber=1]{java}
public class Shape {
  // attributes and fields
  private static int numShapes = 0;
  private String color;
  private boolean filled;
  // constructors
  public Shape() {
    setNumShapes(getNumShapes() + 1);
  }
  // setter and getter for color
  public void setColor(String someColor) {
    this.color = someColor;
  }
  public String getColor() {
    return color;
  }
    \end{minted}
  \end{block}
\end{frame}

\begin{frame}[fragile]
  \frametitle{Inheritance}
  \begin{block}{\texttt{Shape.java} (v2.0) (Part 2)}
    \begin{minted}[fontsize=\small,tabsize=8, linenos, firstnumber=17]{java}
  // setter and getter for filled
  public void setFilled(boolean state) {
    this.filled = state;
  }
  public boolean isFilled() {
    return filled;
  }
  // setter and getter for numShapes
  public static int getNumShapes() {
    return numShapes;
  }
  public static void setNumShapes(int numShapes) {
    Shape.numShapes = numShapes;
  }
}
    \end{minted}
  \end{block}
\end{frame}

\begin{frame}[fragile]
  \frametitle{Inheritance}
  \begin{block}{\texttt{Circle.java} (v2.0) (Part 1)}
    \begin{minted}[fontsize=\small,tabsize=8, linenos, firstnumber=1]{java}
public class Circle extends Shape {
  // fields and attributes
  private static int numCircles = 0;
  private double radius;
  // methods
  public double getArea() {
    return Math.PI*Math.pow(radius,2);
  }
  public double getPerimeter() {
    return 2*Math.PI*radius;
  }
  // constructors
  public Circle(double someRadius) {
    radius = someRadius;
    setNumCircles(getNumCircles() + 1);
  }
    \end{minted}
  \end{block}
\end{frame}

\begin{frame}[fragile]
  \frametitle{Inheritance}
  \begin{block}{\texttt{Circle.java} (v2.0) (Part 2)}
    \begin{minted}[fontsize=\small,tabsize=8, linenos, firstnumber=17]{java}
  // setter and getter for radius
  public void setRadius(double someRadius) {
    this.radius = someRadius;
  }
  public double getRadius() {
    return radius;
  }
  // setter and getter for numCircles
  public static int getNumCircles() {
    return numCircles;
  }
  public static void setNumCircles(int numCircles) {
    Circle.numCircles = numCircles;
  }
}
    \end{minted}
  \end{block}
\end{frame}

\begin{frame}[fragile]
  \frametitle{Inheritance}
  \begin{block}{\texttt{Square.java} (v2.0) (Part 1)}
    \begin{minted}[fontsize=\small,tabsize=8, linenos, firstnumber=1]{java}
public class Square extends Shape {
  // fields and attributes
  private static int numSquares = 0;
  private double length;
  // methods
  public double getArea() {
    return Math.pow(length,2);
  }
  public double getPerimeter() {
    return length*4;
  }
  // constructors
  public Square(double someLength) {
    length = someLength;
    setNumSquares(getNumSquares() + 1);
  }
    \end{minted}
  \end{block}
\end{frame}

\begin{frame}[fragile]
  \frametitle{Inheritance}
  \begin{block}{\texttt{Square.java} (v2.0) (Part 2)}
    \begin{minted}[fontsize=\small,tabsize=8, linenos, firstnumber=17]{java}
  // setter and getter for length
  public void setLength(double someLength) {
    this.length = someLength;
  }
  public double getLength() {
    return length;
  }
  // setter and getter for numSquares
  public static int getNumSquares() {
    return numSquares;
  }
  public static void setNumSquares(int numSquares) {
    Square.numSquares = numSquares;
  }
}
    \end{minted}
  \end{block}
\end{frame}

\begin{frame}[fragile]
  \frametitle{Inheritance}
  \begin{block}{\texttt{Geometry.java} (v2.0)}
    \begin{minted}[fontsize=\small,tabsize=8, linenos, firstnumber=1]{java}
public class Geometry {
  public static void main(String[] args) {
    Circle myCircle = new Circle(5);
    myCircle.setColor("Red");
    myCircle.setFilled(true);
    double result1 = myCircle.getPerimeter();
    System.out.println("Perimeter of circle: "+result1);
    Square mySquare = new Square(4);
    mySquare.setColor("Blue");
    mySquare.setFilled(false);
    double result2 = mySquare.getArea();
    System.out.println("Area of square: "+result2);
    System.out.println(Shape.getNumShapes());
  }
}
    \end{minted}
  \end{block}
\end{frame}

\begin{frame}[fragile]
  \frametitle{Inheritance}
  \begin{block}{\texttt{Overall UML Diagram}}
  \begin{figure}
  \centering
    \begin{tikzpicture}
      \begin{class}[]{Shape}{0, 0}
        \attribute{numShapes: int}
        \attribute{filled: boolean}
        \attribute{color: String}
        \operation{Shape()}
      \end{class}
      \begin{class}[]{Circle}{-3, -3.25}
        \inherit{Shape}
        \attribute{numCircles: int}
        \attribute{radius: double}
        \operation{Circle()}
        \operation{getPerimeter(): double}
        \operation{getArea(): double}
      \end{class}
      \begin{class}[]{Square}{3, -3.25}
        \inherit{Shape}
        \attribute{numSquares: int}
        \attribute{length: double}
        \operation{Square()}
        \operation{getPerimeter(): double}
        \operation{getArea(): double}
      \end{class}
    \end{tikzpicture}
  \end{figure}
  \end{block}
\end{frame}

\begin{frame}[fragile]
  \frametitle{Inheritance}
  \begin{block}{Inheritance}
    \begin{enumerate}
      \item Child class inherits accessible data members and methods from its parent class.
      \item Child class may develop its own data members and methods if required.
      \item Child class may later modify inherited methods if required.
    \end{enumerate}
  \end{block}
\end{frame}

\begin{frame}[fragile]
  \frametitle{Inheritance}
  \begin{block}{IS-A Relationship}
    \alert{IS-A} is a uni-directional relationship between abstractions (classes and interfaces) where one class A is a subclass of another class B.
  \end{block}
  \begin{block}{Example}
  Junior Software Developer > Employee > Adult > Human
  \end{block}
\end{frame}

\begin{frame}[fragile]
  \frametitle{Inheritance}
  \begin{block}{HAS-A Relationship}
    \alert{HAS-A} is a uni-directional relationship between objects where one object A belongs to another object B.
  \end{block}
  \begin{block}{Example}
  Personal Computer > Motherboard > CPU > ALU
  \end{block}
\end{frame}

\begin{frame}[fragile]
  \frametitle{Inheritance}
  \begin{block}{Remember}
    For Class A to inherit from Class B, there must be an \alert{IS-A} relationship between A and B. In this case, Class A is said to extend Class B.
  \end{block}
\end{frame}

\begin{frame}[fragile]
  \frametitle{Inheritance}
  \begin{block}{Single Inheritance Restriction}
    Java does not allow inheritance from multiple classes, i.e. each class has only one parent.
  \end{block}
\end{frame}

\section{Constructor Chaining}

\begin{frame}[fragile]
  \frametitle{Inheritance}
  \begin{block}{Instantiation from Subclass}
    A subclass inherits all accessible data members and methods from its superclass but it does not inherit its constructor.

    Upon instantiation from the subclass, constructor of its superclass is invoked by default.

    Even if the superclass constructor is not explicitly called in subclass constructor, compiler would automatically append \texttt{super()} to constructor of the subclass.

    Java prohibits instantiation from superclass inside its subclass.
  \end{block}
\end{frame}

\section{Method Overriding}

\begin{frame}[fragile]
  \frametitle{Inheritance}
  \begin{block}{Method Overriding}
    An instance method in a subclass with the same signature as an instance method in the superclass overrides the superclass's method. Method overriding is used to provide specific implementation of a method that is already provided by its super class.
  \end{block}
\end{frame}

\begin{frame}[fragile]
  \frametitle{Inheritance}
  \begin{block}{Remember}
    An instance method can be overridden only if it is accessible.

    Static methods of superclass may be inherited but may not be overridden.

    Method overriding is different with method overloading.
  \end{block}
\end{frame}

\begin{frame}[fragile]
  \frametitle{Inheritance}
  \begin{block}{Method Hiding}
    If a subclass defines a static method with the same signature as a static method in the superclass, then the method in the subclass hides the one in the superclass.

    The hidden static method of the superclass may still be invoked by using the superclass name with the method name.
  \end{block}
\end{frame}

\plain{}{Keep Calm\\and\\Think Object-Oriented}

\end{document}
