% -----------------------------------------------------------------------------
% The MIT License (MIT)
%
% Copyright (c) 2015 Pejman Ghorbanzade
%
% Permission is hereby granted, free of charge, to any person obtaining a copy
% of this software and associated documentation files (the "Software"), to deal
% in the Software without restriction, including without limitation the rights
% to use, copy, modify, merge, publish, distribute, sublicense, and/or sell
% copies of the Software, and to permit persons to whom the Software is
% furnished to do so, subject to the following conditions:
%
% The above copyright notice and this permission notice shall be included in
% all copies or substantial portions of the Software.
%
% THE SOFTWARE IS PROVIDED "AS IS", WITHOUT WARRANTY OF ANY KIND, EXPRESS OR
% IMPLIED, INCLUDING BUT NOT LIMITED TO THE WARRANTIES OF MERCHANTABILITY,
% FITNESS FOR A PARTICULAR PURPOSE AND NONINFRINGEMENT. IN NO EVENT SHALL THE
% AUTHORS OR COPYRIGHT HOLDERS BE LIABLE FOR ANY CLAIM, DAMAGES OR OTHER
% LIABILITY, WHETHER IN AN ACTION OF CONTRACT, TORT OR OTHERWISE, ARISING FROM,
% OUT OF OR IN CONNECTION WITH THE SOFTWARE OR THE USE OR OTHER DEALINGS IN
% THE SOFTWARE.
% -----------------------------------------------------------------------------

\def \topDirectory {../..}

\documentclass[10pt, compress]{beamer}

\usepackage{\topDirectory/template/style/directives}
%%%%%%%%%%%%%%%%%%%%%%%%%%%%%%%%%%%%%%%%%%%%%%%%%%%%%%%%%%%%%%%%%%%%%%%%%%%%%%
% CS110: Introduction to Computing
% Copyright 2015 Pejman Ghorbanzade <mail@ghorbanzade.com>
% Creative Commons Attribution-ShareAlike 4.0 International License
% https://github.com/ghorbanzade/UMB-CS110-2015S/blob/master/LICENSE
%%%%%%%%%%%%%%%%%%%%%%%%%%%%%%%%%%%%%%%%%%%%%%%%%%%%%%%%%%%%%%%%%%%%%%%%%%%%%%

\course{id}{CS110}
\course{name}{Introduction to Computing}
\course{venue}{Tue/Thu, 5:30 PM - 6:45 PM}
\course{semester}{Spring 2015}
\course{department}{Department of Computer Science}
\course{university}{University of Massachusetts Boston}

\instructor{name}{Pejman Ghorbanzade}
\instructor{title}{}
\instructor{position}{Student Instructor}
\instructor{email}{pejman@cs.umb.edu}
\instructor{phone}{617-287-6419}
\instructor{office}{S-3-124B}
\instructor{office-hours}{Tue/Thu 19:00-20:30}
\instructor{address}{University of Massachusetts Boston, 100 Morrissey Blvd., Boston, MA}

\usepackage{\topDirectory/template/style/beamerthemeUmassLecture}
\doc{number}{6}
%\setbeamertemplate{footline}[text line]{}

\usepackage{wrapfig}
\usepackage{verbatimbox}

\begin{document}
\prepareCover

\section{Course Administration}

\begin{frame}[fragile]
\frametitle{Course Administration}
	Quiz 1(a) Released for Section 5 and 7 Students.

	Due Feb 24 at 11:00 PM. Submission is online.
\end{frame}

\begin{frame}[fragile]
	\frametitle{Overview}
	\begin{itemize}
		\item[] Conditionals
			\begin{itemize}
				\item[] \texttt{if} Statement
				\item[] \texttt{if else} Statement
				\item[] \texttt{switch} Statement
			\end{itemize}
		\item[] Loops
			\begin{itemize}
				\item[] \texttt{while} Statement
				\item[] \texttt{do while} Statement
				\item[] \texttt{for} Statement
				\item[] Nested Loops
			\end{itemize}
		\item[] Branches
			\begin{itemize}
				\item[] \texttt{break} Statement
				\item[] \texttt{continue} Statement
				\item[] \texttt{return} Statement
			\end{itemize}
	\end{itemize}
\end{frame}

\plain{}{Conditionals}

\section{\texttt{if} Statement}

\begin{frame}[fragile]
	\frametitle{\texttt{if} Statement}
	\begin{block}{Objective}
		Write a program \texttt{Quadratic.java} that takes three integers \texttt{a}, \texttt{b} and \texttt{c} as command line arguments and solves for \texttt{x} with three digits of precision the quadratic expression shown below.
		\begin{equation*}\label{eq1}
			a^2 x + b x + c = 0
		\end{equation*}
	\end{block}
\end{frame}

\begin{frame}[fragile]
	\frametitle{\texttt{if} Statement}
	\begin{block}{\texttt{Quadratic.java} (v1.0)}
		\begin{minted}[fontsize=\small,tabsize=8, linenos, firstnumber=1]{java}
public class Quadratic {
	public static void main(String[] args) {
		double a = Double.parseDouble(args[0]);
		double b = Double.parseDouble(args[1]);
		double c = Double.parseDouble(args[2]);
		double discriminant = Math.pow(b,2) - 4 * a * c;
		double sol1 = (- b + Math.sqrt(discriminant))/(2*a);
		double sol2 = (- b - Math.sqrt(discriminant))/(2*a);
		System.out.printf("Solutions are %.3f and %.3f\n",
				sol1,sol2);
	}
}
		\end{minted}
	\end{block}
\end{frame}

\begin{frame}[fragile]
	\frametitle{\texttt{if} Statement}
	\begin{block}{Problem Statement}
		\begin{minted}[fontsize=\small,tabsize=8]{text}
		> javac Quadratic.java
		> java Quadratic 1 -2 -4
			Solutions are -1.236 and 3.237
		> java Quadratic 9 12 4
			Solutions are -0.667 and -0.667
		> java Quadratic 3 4 2
			Solutions are NaN and NaN
		\end{minted}
	\end{block}
	\begin{block}{Proposed Solution}
		What if we could make the program decide what to print based on value of discriminant?
	\end{block}
\end{frame}

\begin{frame}[fragile]
	\frametitle{\texttt{if} Statement}
	\begin{block}{\texttt{Quadratic.java} (v1.1)}
		\begin{minted}[fontsize=\small,tabsize=8, linenos, firstnumber=3]{java}
double a = Double.parseDouble(args[0]);
double b = Double.parseDouble(args[1]);
double c = Double.parseDouble(args[2]);
double discriminant = Math.pow(b,2) - 4 * a * c;
if (discriminant > 0 ) {
	double sol1 = (- b + Math.sqrt(discriminant))/(2*a);
	double sol2 = (- b - Math.sqrt(discriminant))/(2*a);
	System.out.printf("Solutions are %.3f and %.3f\n",
		sol1, sol2);
}
		\end{minted}
	\end{block}
	\texttt{if-then} conditionals helps us develope more flexible programs.
\end{frame}

\begin{frame}[fragile]
	\frametitle{\texttt{if} Statement}
	\begin{block}{Problem Statement}
		\begin{minted}[fontsize=\small,tabsize=8]{text}
		> javac Quadratic.java
		> java Quadratic 1 -2 -4
			Solutions are -1.236 and 3.237
		> java Quadratic 9 12 4
		> java Quadratic 3 4 2
		\end{minted}
	\end{block}
	\begin{block}{Proposed Solution}
		Use more \texttt{if-then} conditionals to cover different conditions.
	\end{block}
\end{frame}

\begin{frame}[fragile]
	\frametitle{\texttt{if} Statement}
	\begin{block}{\texttt{Quadratic.java} (v1.2)}
		\begin{minted}[fontsize=\small,tabsize=8, linenos, firstnumber=6]{java}
double discriminant = Math.pow(b,2) - 4 * a * c;
if (discriminant > 0 ) {
	double sol1 = (- b + Math.sqrt(discriminant))/(2*a);
	double sol2 = (- b - Math.sqrt(discriminant))/(2*a);
	System.out.printf("Solutions are %.3f and %.3f\n",
		sol1, sol2);
}
if (discriminant == 0) {
	double sol1 = (- b )/(2*a);
	System.out.printf("Solution is %.3f\n", sol1);
}
if (discriminant < 0)
	System.out.printf("No real solution exists.");
		\end{minted}
	\end{block}
\end{frame}

\begin{frame}[fragile]
	\frametitle{\texttt{if} Statement}
	\begin{block}{Problem Statement}
		\begin{minted}[fontsize=\small,tabsize=8]{text}
		> javac Quadratic.java
		> java Quadratic 1 -2 -4
			Solutions are -1.236 and 3.237
		> java Quadratic 9 12 4
			Solution is -0.667
		> java Quadratic 3 4 2
			No real solution exists.
		\end{minted}
		No problem... Oh, Really?
	\end{block}
\end{frame}

\begin{frame}[fragile]
	\frametitle{\texttt{if} Statement}
	\begin{block}{\texttt{Quadratic.java} (v1.2.1)}
		\begin{minted}[fontsize=\small,tabsize=8, linenos, firstnumber=7]{java}
if (discriminant > 0 ) {
	System.out.println("Inside 1st loop.");
	double sol1 = (- b + Math.sqrt(discriminant))/(2*a);
	double sol2 = (- b - Math.sqrt(discriminant))/(2*a);
	System.out.printf("Solutions are %.3f and %.3f\n",
		sol1, sol2);
}
if (discriminant == 0) {
	System.out.println("Inside 2nd loop.");
	double sol1 = (- b )/(2*a);
	System.out.printf("Solution is %.3f\n", sol1);
}
if (discriminant < 0) {
	System.out.println("Inside 3rd loop.");
	System.out.printf("No real solution exists.");
}
		\end{minted}
	\end{block}
\end{frame}

\begin{frame}[fragile]
	\frametitle{\texttt{if} Statement}
	\begin{block}{Problem Statement}
		\begin{minted}[fontsize=\small,tabsize=8]{text}
		> javac Quadratic.java
		> java Quadratic 1 -2 -4
			Inside 1st Loop.
			Solutions are -1.236 and 3.237
			Inside 2nd Loop.
			Inside 3rd Loop.
		\end{minted}
	\end{block}
	\begin{block}{Proposed Solution}
		\texttt{if-then-else} conditionals helps us avoid unnecessary checks.
	\end{block}
\end{frame}

\section{\texttt{if else} Statement}

\begin{frame}[fragile]
	\frametitle{\texttt{if else} Statement}
	\begin{block}{\texttt{Quadratic.java} (v1.3)}
		\begin{minted}[fontsize=\small,tabsize=8, linenos, firstnumber=7]{java}
if (discriminant > 0 ) {
	System.out.println("Inside 1st loop.");
	double sol1 = (- b + Math.sqrt(discriminant))/(2*a);
	double sol2 = (- b - Math.sqrt(discriminant))/(2*a);
	System.out.printf("Solutions are %.3f and %.3f\n",
		sol1, sol2);
}
else if (discriminant == 0) {
	System.out.println("Inside 2nd loop.");
	double sol1 = (- b )/(2*a);
	System.out.printf("Solution is %.3f\n", sol1);
}
else {
	System.out.println("Inside 3rd loop.");
	System.out.printf("No real solution exists.");
}
		\end{minted}
	\end{block}
\end{frame}

\begin{frame}[fragile]
	\frametitle{\texttt{if else} Statement}
	\begin{block}{Result}
		\begin{minted}[fontsize=\small,tabsize=8]{text}
		> javac Quadratic.java
		> java Quadratic 1 -2 -4
			Inside 1st Loop.
			Solutions are -1.236 and 3.237
		> java Quadratic 9 12 4
			Inside 2nd Loop.
			Solution is -0.667
		> java Quadratic 3 4 2
			Inside 3rd Loop.
			No real solution exists.
		\end{minted}
	\end{block}
\end{frame}

\begin{frame}[fragile]
	\frametitle{\texttt{if else} Statement}
	\begin{block}{\texttt{Quadratic.java} (v1.3.1)}
		\begin{minted}[fontsize=\small,tabsize=8, linenos, firstnumber=7]{java}
double sol1 = (- b + Math.sqrt(discriminant))/(2*a);
double sol2 = (- b - Math.sqrt(discriminant))/(2*a);
if (discriminant > 0 )
	System.out.printf("Solutions are %.3f and %.3f\n",
		sol1, sol2);
else if (discriminant == 0)
	System.out.printf("Solution is %.3f\n", sol1);
else
	System.out.printf("No real solution exists.");
		\end{minted}
	\end{block}
\end{frame}

\begin{frame}[fragile]
	\frametitle{\texttt{if else} Statement}
	\begin{block}{Result}
		\begin{minted}[fontsize=\small,tabsize=8]{text}
		> javac Quadratic.java
		> java Quadratic 1 -2 -4
			Solutions are -1.236 and 3.237
		> java Quadratic 9 12 4
			Solution is -0.667
		> java Quadratic 3 4 2
			No real solution exists.
		\end{minted}
	\end{block}
\end{frame}

\section{\texttt{switch} Statement}

\begin{frame}[fragile]
	\frametitle{\texttt{switch} Statement}
	\begin{block}{Objective}
		Write a program \texttt{Grade.java} that takes a letter grade and converts it to its numerical equivalent, based on the table given below.
		\begin{table}[H]
			\begin{tabular}{c|c}
				Letter Grade & Numerical Grade\\
				\hline
				A & 4\\
				B & 3\\
				C & 2\\
				D & 1\\
				E & 0\\
			\end{tabular}
		\end{table}
	\end{block}
\end{frame}

\begin{frame}[fragile]
	\frametitle{\texttt{switch} Statement}
	\begin{block}{\texttt{Grade.java} (v1.0)}
		\begin{minted}[fontsize=\small,tabsize=8, linenos, firstnumber=1]{java}
public class Grade {
	public static void main (String[] args) {
		Scanner input = new Scanner(System.in);
		System.out.print("Insert a letter grade: ");
		String letter = input.nextLine();
		if (letter.equals("A"))
			System.out.println("GPA is 4");
		else if (letter.equals("B"))
			System.out.println("GPA is 3");
		else if (letter.equals("C"))
			System.out.println("GPA is 2");
		else if (letter.equals("D"))
			System.out.println("GPA is 1");
		else
			System.out.println("GPA is 0");
	}
}
		\end{minted}
	\end{block}
\end{frame}

\begin{frame}[fragile]
	\frametitle{\texttt{switch} Statement}
	\begin{block}{Problem Statement}
	Not smart enough to earn accolade. We were lucky grade letters end in \texttt{F}.
	\end{block}
	\begin{block}{Disagree?}
	Write a program \texttt{Month.java} that takes a number between 1 to 12 and prints full name of the month of the year corresponding to that number.
	\end{block}
	\begin{block}{Proposed Solution}
		\texttt{switch} conditionals help us avoid repetitive \texttt{if-then-else} conditionals.
	\end{block}
\end{frame}

\begin{frame}[fragile]
	\frametitle{\texttt{switch} Statement}
	\begin{block}{\texttt{Grade.java} (v2.0)}
		\begin{minted}[fontsize=\small,tabsize=8, linenos, firstnumber=3]{java}
Scanner input = new Scanner(System.in);
System.out.print("Insert a letter grade: ");
String letter = input.nextLine();
input.close();
double num;
switch (letter.charAt(0)) {
	case 'A': num = 4; break;
	case 'B': num = 3; break;
	case 'C': num = 2; break;
	case 'D': num = 1; break;
	default: num = 0; break;
}
System.out.printf("GPA is %.2f\n", num);
		\end{minted}
	\end{block}
\end{frame}

\begin{frame}[fragile]
	\frametitle{\texttt{switch} Statement}
	\begin{block}{Remember}
		\begin{itemize}
			\item[] \texttt{switch} is not always an alternative for the \texttt{if-then-else} conditional. \texttt{switch} is a very special case of more general \texttt{if-then-else} conditionals.
			\item[] It is always necessary to use \texttt{break} for cases when we want to prevent other cases to be checked.
			\item[] The \texttt{default} section handles all values that are not explicitly handled by one of the \texttt{case} sections.
		\end{itemize}
	\end{block}
\end{frame}

\plain{}{Loops}

\section{\texttt{while} Statement}

\begin{frame}[fragile]
	\frametitle{\texttt{while} Statement}
	\begin{block}{\texttt{HelloWorld.java}}
		\begin{minted}[fontsize=\small,tabsize=8, linenos, firstnumber=1]{java}
public class HelloWorld {
	public static void main(String[] args) {
		System.out.println("Hello World!");
	}
}
		\end{minted}
	\end{block}
	\begin{block}{Objective}
		Write a program \texttt{HelloWorld100.java} so that it prints \texttt{Hello World!} a hundred times.
	\end{block}
\end{frame}

\begin{frame}[fragile]
	\frametitle{\texttt{while} Statement}
	\begin{block}{\texttt{HelloWorld100.java}}
		\begin{minted}[fontsize=\small,tabsize=8, linenos, firstnumber=1]{java}
public class HelloWorld {
	public static void main(String[] args) {
		int i = 0;
		while (i < 100) {
			System.out.println("Hello World!");
			i++;
		}
	}
}
		\end{minted}
	\end{block}
\end{frame}

\begin{frame}[fragile]
	\frametitle{\texttt{while} Statement}
	\begin{block}{\texttt{while} Loop}
		The \texttt{while} statement continually executes a block of statements while a particular condition is \texttt{true}.
		\begin{minted}[fontsize=\small,tabsize=8]{text}
			while (condition) {
				statement(s);
			}
		\end{minted}
	\end{block}
	\begin{block}{Remember}
		Condition is always evaluated first.

		Statements are executed \textbf{while} condition is evaluated as \texttt{true}.

		Condition will be reevaluated once \textbf{all} statements are executed.
	\end{block}
\end{frame}

\section{\texttt{do while} Statement}

\begin{frame}[fragile]
	\frametitle{\texttt{do while} Statement}
	\begin{block}{Objective}
		Write a program \texttt{twoSix.java} that keeps rolling two dice until you get two six.
	\end{block}
\end{frame}

\begin{frame}[fragile]
	\frametitle{\texttt{do while} Statement}
	\begin{block}{\texttt{TwoSix.java} (v1.0)}
		\begin{minted}[fontsize=\small,tabsize=8, linenos, firstnumber=1]{java}
public class TwoSix {
	public static void main(String[] args) {
		boolean roll = true;
		int dice1, dice2;
		while (roll) {
			dice1 = (int) Math.ceil( Math.random() * 6 );
			dice2 = (int) Math.ceil( Math.random() * 6 );
			System.out.printf("Dice 1: %d\tDice2: %d\n",
				dice1, dice2);
			if (dice1 + dice2 == 12)
				roll = false;
		}
	}
}
		\end{minted}
	\end{block}
\end{frame}

\begin{frame}[fragile]
	\frametitle{\texttt{do while} Statement}
	\begin{block}{\texttt{TwoSix.java} (v2.0)}
		\begin{minted}[fontsize=\small,tabsize=8, linenos, firstnumber=1]{java}
public class TwoSix {
	public static void main(String[] args) {
		int dice1, dice2;
		do {
			dice1 = (int) Math.ceil( Math.random() * 6 );
			dice2 = (int) Math.ceil( Math.random() * 6 );
			System.out.printf("Dice 1: %d\tDice2: %d\n",
				dice1, dice2);
		} while (dice1 + dice2 < 12);
	}
}
		\end{minted}
	\end{block}
\end{frame}

\begin{frame}[fragile]
	\frametitle{\texttt{do while} Statement}
	\begin{block}{\texttt{do-while} Loop}
		The \texttt{do-while} loop is a variation of \texttt{while} loop which evaluates iteration condition after execution of block of statements.
		\begin{minted}[fontsize=\small,tabsize=8]{text}
			do {
				statement(s);
			} while (condition);
		\end{minted}
	\end{block}
	\begin{block}{Remember}
		Condition is evaluated once \textbf{all} statements are executed.

		Statements are reexecuted \textbf{while} condition is evaluated as \texttt{true}.

		Condition will be reevaluated once \textbf{all} statements are reexecuted.
	\end{block}
\end{frame}

\section{\texttt{for} Statement}

\begin{frame}[fragile]
	\frametitle{\texttt{for} Statement}
	\begin{block}{\texttt{HelloWorld100.java} (v1.0)}
		\begin{minted}[fontsize=\small,tabsize=8, linenos, firstnumber=3]{java}
int i = 0;
while (i < 100) {
	System.out.println("Hello World!");
	i++;
}
		\end{minted}
	\end{block}
	\begin{block}{Anathomy of a Code}
		\begin{minted}[fontsize=\small,tabsize=8]{text}
intitialize iteration counter;
while (iteration counter has not reached threshold) {
	execute block of statements;
	increment iteration counter;
}
		\end{minted}
	\end{block}
\end{frame}

\begin{frame}[fragile]
	\frametitle{\texttt{for} Statement}
	\begin{block}{\texttt{HelloWorld100.java} (v2.0)}
		\begin{minted}[fontsize=\small,tabsize=8, linenos, firstnumber=1]{java}
public class HelloWorld {
	public static void main(String[] args) {
		for (int i = 0; i < 100; i++)
			System.out.println("Hello World!");
	}
}
		\end{minted}
	\end{block}
\end{frame}

\begin{frame}[fragile]
	\frametitle{\texttt{for} Statement}
	\begin{block}{\texttt{for} Loop}
		The \texttt{for} loop is a special case of a \texttt{while} loop that executes a block of statements for a specified number of times. Any \texttt{for} loop can be converted to a \texttt{while} loop.
		\begin{minted}[fontsize=\small,tabsize=8]{text}
		for (initialization, condition, afterthought) {
			statement(s);
		}
		\end{minted}
		\begin{enumerate}
			\item Initialization\\Intitializes iteration counter
			\item Condition\\Specifies iteration condition
			\item Afterthought\\Increments/Decrements iteration counter
		\end{enumerate}
	\end{block}
\end{frame}

\section{Nested Loops}

\begin{frame}[fragile]
	\frametitle{Nested Loops}
	\begin{block}{Objective}
		Write a program \texttt{Triangle.java} that takes number $n$ from user and prints a traingle in $n$ lines using star characters, such that there are $k$ star characters in $k^{th}$ line.
\begin{verbbox}
*
**
***
****
*****
\end{verbbox}
		\begin{figure}[H]\centering
		\theverbbox
		\caption{Sample riangle generated for $n = 5$}
		\end{figure}
	\end{block}
\end{frame}

\begin{frame}[fragile]
	\frametitle{Nested Loops}
	\begin{block}{\texttt{Triangle.java}}
		\begin{minted}[fontsize=\small,tabsize=8, linenos, firstnumber=1]{java}
public class Triangle {
	public static void main(String[] args) {
		Scanner input = new Scanner(System.in);
		int numRow = input.nextInt();
		input.close();
		for (int i = 0; i < numRow; i++) {
			for (int j = 0; j <= i; j++)
				System.out.print("*");
			System.out.printf("\n");
		}
	}
}
		\end{minted}
	\end{block}
\end{frame}

\plain{}{Branches}

\section{\texttt{break} Statement}

\begin{frame}[fragile]
	\frametitle{\texttt{break} Statement}
	\begin{block}{Objective}
		Write a program \texttt{FindP.java} that takes a String as command-line argument and checks if it contains the letter \texttt{p} (case insensitive). Do not use method \texttt{String.contains()}.
	\end{block}
\end{frame}


\begin{frame}[fragile]
	\frametitle{\texttt{break} Statement}
	\begin{block}{FindP.java (v1.0)}
		\begin{minted}[fontsize=\small,tabsize=8, linenos, firstnumber=1]{java}
public class FindP {
	public static void main(String[] args) {
		String phrase = args[0].toLowerCase();
		char letter = 'p';
		boolean found = false;
		for (int i = 0; i < phrase.length(); i++) {
			if (phrase.charAt(i) == letter)
				found = true;
		}
		if (found)
			System.out.println("Found some '" + letter "'.");
		else
			System.out.println("No '"+ letter +"' found.");
	}
}
		\end{minted}
	\end{block}
\end{frame}

\begin{frame}[fragile]
	\frametitle{\texttt{break} Statement}
	\begin{block}{Result}
		\begin{minted}[fontsize=\small,tabsize=8]{text}
		> javac FindP.java
		> java FindP Hello
			No 'p' Found.
		> java FindP Pejman
			Found Some 'p'.
		\end{minted}
	\end{block}
\end{frame}

\begin{frame}[fragile]
	\frametitle{\texttt{break} Statement}
	\begin{block}{FindP.java (v2.0)}
		\begin{minted}[fontsize=\small,tabsize=8, linenos, firstnumber=1]{java}
public class FindP {
	public static void main(String[] args) {
		String phrase = args[0].toLowerCase();
		char letter = 'p';
		for (int i = 0; i < phrase.length(); i++) {
			if (phrase.charAt(i) == letter)
				System.out.print(letter);
			else
				System.out.print("-");
		}
	}
}
		\end{minted}
	\end{block}
\end{frame}

\begin{frame}[fragile]
	\frametitle{\texttt{break} Statement}
	\begin{block}{Result}
		\begin{minted}[fontsize=\small,tabsize=8]{text}
		> javac FindP.java
		> java FindP Happy
			--pp-
		> java FindP Pejman
			p-----
		\end{minted}
	\end{block}
	\begin{block}{Problem Statement}
		There are as many loop iterations as \texttt{phrase.length()}. Are they really required?
	\end{block}
	\begin{block}{Proposed Solution}
		\texttt{break} statement allows termination of a loop without execution of the entire block of statements.
	\end{block}
\end{frame}


\begin{frame}[fragile]
	\frametitle{\texttt{break} Statement}
	\begin{block}{FindP.java (v1.1)}
		\begin{minted}[fontsize=\small,tabsize=8, linenos, firstnumber=3]{java}
String phrase = args[0].toLowerCase();
char letter = 'p';
boolean found = false;
for (int i = 0; i < phrase.length(); i++)
	if (phrase.charAt(i) == letter) {
		found = true;
		break;
	}
if (found)
	System.out.println("Found some '" + letter "'.");
else
	System.out.println("No '"+ letter +"' found.");
		\end{minted}
	\end{block}
\end{frame}

\begin{frame}[fragile]
	\frametitle{\texttt{break} Statement}
	\begin{block}{FindP.java (v2.1)}
		\begin{minted}[fontsize=\small,tabsize=8, linenos, firstnumber=1]{java}
String phrase = args[0].toLowerCase();
char letter = 'p';
for (int i = 0; i < phrase.length(); i++) {
	if (phrase.charAt(i) == letter) {
		System.out.print(letter);
		break;
	}
	else
		System.out.print("-");
}
		\end{minted}
	\end{block}
	\begin{block}{Result}
		\begin{minted}[fontsize=\small,tabsize=8]{text}
		> javac FindP.java
		> java FindP Happy
			--p
		\end{minted}
	\end{block}
\end{frame}

\begin{frame}[fragile]
	\frametitle{\texttt{break} Statement}
	\begin{block}{Note}
	Unconditional branches can be avoided to enhance code readability.
	\end{block}
	\begin{block}{FindP.java (v3.0)}
		\begin{minted}[fontsize=\small,tabsize=8, linenos, firstnumber=1]{java}
String phrase = args[0].toLowerCase();
char letter = 'p';
boolean found = false;
int i = 0;
while (!found && i < phrase.length())
	if (phrase.charAt(i++) == letter)
		found = true;
System.out.println(found ? "Found" : "Not Found");
		\end{minted}
	\end{block}
\end{frame}

\section{\texttt{continue} Statement}

\begin{frame}[fragile]
	\frametitle{\texttt{continue} Statement}
	\begin{block}{Objective}
		Write a program \texttt{CountP.java} that takes a String as command-line argument and prints number of 'p' letters it contains. Comparison is case insensitive.
	\end{block}
\end{frame}

\begin{frame}[fragile]
	\frametitle{\texttt{continue} Statement}
	\begin{block}{CountP.java (v1.0)}
		\begin{minted}[fontsize=\small,tabsize=8, linenos, firstnumber=1]{java}
public class CountP {
	public static void main(String[] args) {
		String phrase = args[0].toLowerCase();
		char letter = 'p';
		int counter = 0;
		for (int i = 0; i < phrase.length(); i++) {
			if (phrase.charAt(i) != letter)
				continue;
			counter++;
		}
		System.out.printf("Found %d 'P'(s)\n", counter);
	}
}
		\end{minted}
	\end{block}
	The continue statement skips the current iteration of a loop.
\end{frame}

\begin{frame}[fragile]
	\frametitle{\texttt{continue} Statement}
	\begin{block}{Note}
	Unconditional branches can be avoided to enhance code readability.
	\end{block}
	\begin{block}{CountP.java (v2.0)}
		\begin{minted}[fontsize=\small,tabsize=8, linenos, firstnumber=3]{java}
String phrase = args[0].toLowerCase();
char letter = 'p';
int counter = 0;
for (int i = 0; i < phrase.length(); i++) {
	if (phrase.charAt(i) == letter)
		counter++;
}
System.out.printf("Found %d 'P'(s)\n", counter);
		\end{minted}
	\end{block}
\end{frame}


\section{\texttt{return} Statement}

\begin{frame}[fragile]
	\frametitle{\texttt{return} Statement}
	\begin{block}{Definition}
	Exits from the current method, and control flow returns to where the method was invoked.
	\end{block}
	Will be discussed later.
\end{frame}

\plain{}{Keep Calm\\and\\Practice}

\end{document}
