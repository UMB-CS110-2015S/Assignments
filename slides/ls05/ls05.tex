% -----------------------------------------------------------------------------
% The MIT License (MIT)
%
% Copyright (c) 2015 Pejman Ghorbanzade
%
% Permission is hereby granted, free of charge, to any person obtaining a copy
% of this software and associated documentation files (the "Software"), to deal
% in the Software without restriction, including without limitation the rights
% to use, copy, modify, merge, publish, distribute, sublicense, and/or sell
% copies of the Software, and to permit persons to whom the Software is
% furnished to do so, subject to the following conditions:
%
% The above copyright notice and this permission notice shall be included in
% all copies or substantial portions of the Software.
%
% THE SOFTWARE IS PROVIDED "AS IS", WITHOUT WARRANTY OF ANY KIND, EXPRESS OR
% IMPLIED, INCLUDING BUT NOT LIMITED TO THE WARRANTIES OF MERCHANTABILITY,
% FITNESS FOR A PARTICULAR PURPOSE AND NONINFRINGEMENT. IN NO EVENT SHALL THE
% AUTHORS OR COPYRIGHT HOLDERS BE LIABLE FOR ANY CLAIM, DAMAGES OR OTHER
% LIABILITY, WHETHER IN AN ACTION OF CONTRACT, TORT OR OTHERWISE, ARISING FROM,
% OUT OF OR IN CONNECTION WITH THE SOFTWARE OR THE USE OR OTHER DEALINGS IN
% THE SOFTWARE.
% -----------------------------------------------------------------------------

\def \topDirectory {../..}

\documentclass[10pt, compress]{beamer}

\usepackage{\topDirectory/template/style/directives}
%%%%%%%%%%%%%%%%%%%%%%%%%%%%%%%%%%%%%%%%%%%%%%%%%%%%%%%%%%%%%%%%%%%%%%%%%%%%%%
% CS110: Introduction to Computing
% Copyright 2015 Pejman Ghorbanzade <mail@ghorbanzade.com>
% Creative Commons Attribution-ShareAlike 4.0 International License
% https://github.com/ghorbanzade/UMB-CS110-2015S/blob/master/LICENSE
%%%%%%%%%%%%%%%%%%%%%%%%%%%%%%%%%%%%%%%%%%%%%%%%%%%%%%%%%%%%%%%%%%%%%%%%%%%%%%

\course{id}{CS110}
\course{name}{Introduction to Computing}
\course{venue}{Tue/Thu, 5:30 PM - 6:45 PM}
\course{semester}{Spring 2015}
\course{department}{Department of Computer Science}
\course{university}{University of Massachusetts Boston}

\instructor{name}{Pejman Ghorbanzade}
\instructor{title}{}
\instructor{position}{Student Instructor}
\instructor{email}{pejman@cs.umb.edu}
\instructor{phone}{617-287-6419}
\instructor{office}{S-3-124B}
\instructor{office-hours}{Tue/Thu 19:00-20:30}
\instructor{address}{University of Massachusetts Boston, 100 Morrissey Blvd., Boston, MA}

\usepackage{\topDirectory/template/style/beamerthemeUmassLecture}
\doc{number}{5}
%\setbeamertemplate{footline}[text line]{}

\usepackage{xeCJK}
\setCJKmainfont{SimSun}
\usepackage{alltt}

\begin{document}
\prepareCover

\plain{}{祝贺大家新年好\\Happy New Lunar Year}

\section{Course Administration}

\begin{frame}[fragile]
\frametitle{Course Administration}
	Quiz 1(b) Released for Section 6 and 8 Students.

	Due Feb 19 at 11:00 PM. Submission is online.
\end{frame}

\begin{frame}[fragile]
	\frametitle{Overview}
	\begin{itemize}
		\item[] Conditionals
		\item[] Class Scanner
	\end{itemize}
\end{frame}

\section{Conditionals}

\begin{frame}[fragile]
	\frametitle{Conditionals}
	\begin{block}{Control Flow}
		Control Flow (Statement Sequencing) refers to the order in which statements are executed in a program.
	\end{block}
	\begin{block}{Control Flow Types}
		\begin{itemize}
			\item[] Unconditional Branch
			\item[] Conditional Branch
			\item[] Subroutine
			\item[] Interrupts
			\item[] Uncondigtional Halt
		\end{itemize}
	\end{block}
\end{frame}

\begin{frame}[fragile]
	\frametitle{Conditionals}
	\begin{block}{Unconditional Branch}
		\begin{alltt}
						ORG		 R00
						IN			R26, P
						LSL		 R26
						\alert{BRCC}		POSITIVE
						LDI		 R27, \$00
						OUT		 P, R27
						JMP		 END
POSITIVE:	 LDI		 R27, \$ff
						OUT		 P, R27
						JMP		 END
END:				NOP
		\end{alltt}
	\end{block}
\end{frame}

\begin{frame}[fragile]
	\frametitle{Conditionals}
	\begin{block}{Unconditional Branch}
		\begin{minted}[fontsize=\small,tabsize=8]{c}
	include <stdio.h>
	int main() {
		int number = scanf("%d", &n);
		if (number > 0)
			goto JUMP;
		elseif (number < 0)
			printf("number negative");
		else
			printf("number zeor");
		return 0;
	JUMP:
		printf("number positive");
		return 0;
	}
		\end{minted}
	\end{block}
\end{frame}

\begin{frame}[fragile]
	\frametitle{Conditionals}
	\begin{block}{Minimal Structured Control Flow}
		Böhm-Jacopini theorem states that any computing application can be realized using \emph{conditional branches} and \emph{subroutines}.
	\end{block}
	\begin{block}{Conditional Branch}
		\begin{minted}[fontsize=\small,tabsize=8]{java}
	if (a > b)
		System.out.println(a);
	else if (a < b)
		System.out.println(b);
	else
		System.out.println("Equal");
		\end{minted}
	\end{block}
\end{frame}

\begin{frame}[fragile]
	\frametitle{Conditionals}
	\begin{block}{Conditional Branch}
		\begin{minted}[fontsize=\small,tabsize=8]{java}
	public class SignDetector {
		public static void main(String[] args) {
			int number = Integer.parseInt(args[0]);
			if (number > 0) {
				System.out.println("Number is positive");
			}
			else if (number < 0) {
				System.out.println("Number is negative");
			}
			else {
				System.out.println("Number is zero");
			}
		}
	}
		\end{minted}
	\end{block}
\end{frame}

\section{Class Scanner}

\begin{frame}[fragile]
	\frametitle{Class Scanner}
	\begin{block}{Problem Definition}
		\begin{itemize}
			\item[] Command-line arguments to be given before program execution.
				\begin{itemize}
					\item[] What if we want to get input when certain conditions are met?
				\end{itemize}
		\end{itemize}
	\end{block}
	\begin{block}{Solution}
		\begin{itemize}
			\item[] Scanner class provides a simple yet powerful way to ask for inputs during execution of program.
		\end{itemize}
	\end{block}
\end{frame}

\begin{frame}[fragile]
	\frametitle{Class Scanner}
	\begin{block}{Example}
		\begin{minted}[fontsize=\small,tabsize=8]{java}
	public class SignDetector {
		public static void main(String[] args) {
			Scanner input = new Scanner(System.in);
			System.out.print("a = ");
			int a = input.nextInt();
			System.out.print("b = ");
			int b = input.nextInt();
			input.close();
			System.out.println("a + b = " + (a + b));
		}
	}
		\end{minted}
	\end{block}
\end{frame}

\begin{frame}[fragile]
	\frametitle{Class Scanner}
	\begin{block}{Instantiation}
		\begin{minted}[fontsize=\small,tabsize=8]{java}
			Scanner input = new Scanner(System.in);
		\end{minted}
	\end{block}
	\begin{block}{Methods}
		\begin{columns}
			\begin{column}{0.5\textwidth}
				\begin{itemize}
					\item[] reset()
					\item[] skip()
					\item[] match()
					\item[] useDelimiter()
					\item[] close()
				\end{itemize}
			\end{column}
			\begin{column}{0.5\textwidth}
				\begin{itemize}
					\item[] next()
					\item[] nextLine()
					\item[] nextInt()
					\item[] nextDouble()
					\item[] hasNext()
				\end{itemize}
			\end{column}
		\end{columns}
	\end{block}
\end{frame}

\begin{frame}[fragile]
	\frametitle{Class Scanner}
	\begin{block}{Final Note}
		\begin{itemize}
			\item[] It is a good practice to use the \texttt{close()} method to close once you are sure you will no longer use it. This helps you avoid memory leakage and releases system resources.
			\item[] Once \texttt{Scanner} object is closed, you will no longer be able to ask for inputs from \texttt{System.in}. To avoid this issue, you must have another Scanner object on which \texttt{close()} method is not invoked yet.
		\end{itemize}
	\end{block}
\end{frame}

\plain{}{Keep Calm\\ and\\ Practice}

\end{document}
