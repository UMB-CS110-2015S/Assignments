\section*{Question 1}
Write a program \texttt{PrimeCounter.java} that takes a command-line argument \texttt{N} and finds the number of primes less than or equal to \texttt{N}. Use it to \textbf{efficiently} print out the number of primes less than or equal to 10 million.

\subsection*{Solution}
\lstset{language=Java,tabsize=2}
\begin{lstlisting}
public class PrimeCounter {
	public static void main(String args[]) {
		int i; // first loop (number counter)
		int j; // second loop (prime checker)
		int[] primeList = new int[100000]; // array of primes
		int counter = 0; // counter for array of primes
		boolean definitelyNotPrime; // flag for prime checking
		// args[0] is a string. We have to convert it to an integer first.
		int maxNumber = Integer.parseInt(args[0]);		
		System.out.println("Prime numbers less than " + args[0] + " are: ");
		for (i = 2; i <= maxNumber; i++) {
			// check if a number is prime
			// a number is prime if is not divisible by any smaller prime number
			// note that we can't afford to check if any smaller number divides our number
			definitelyNotPrime = false;
			for (j = 0; j < counter; j++) {
				if (i % primeList[j] == 0) {
					definitelyNotPrime = true;
				}
			}
			if (!definitelyNotPrime) {
				primeList[counter++] = i;
				System.out.printf("Prime %3d: %d\n", counter, i);
			}
		}
	}
}
\end{lstlisting}
\section*{Question 2}
Write a program \texttt{MatrixDeterminant.java} that asks for elements of a three-by-three matrix and prints its determinant.

\subsection*{Solution}
\lstset{language=Java,tabsize=2}
\begin{lstlisting}
import java.util.Scanner;
public class MatrixMultiply {
	public static void main(String[] args) {
		// Declaration of Variables
		int i; // counter
		int j; // counter
		int p; // counter
		int q; // counter
		int k;
		int row; // row of matrix
		int col; // column of matrix
		double matrix1[][] = new double[3][3];
		double matrix2[][] = new double[3][3];
		double multiplied[][] = new double[3][3];
		Scanner input = new Scanner(System.in);
		// ask for matrices
		for ( i = 0; i < 2; i++) {
			for ( j = 0; j < 9; j++) {
				row = j / 3;
				col = j % 3;
				if (i == 0) {
					System.out.print("Enter matrix1[" + row + "][" + col + "]: ");
					matrix1[row][col] = input.nextInt();					
				}
				else {
					System.out.print("Enter matrix2[" + row + "][" + col + "]: ");
					matrix2[row][col] = input.nextInt();					
				}
			}
			System.out.println("Thank you. Matrix initialized as follows:");
			for ( p = 0; p < 3; p++) {
				for ( q = 0; q < 3; q++) {
					System.out.printf("%.1f ",matrix1[p][q]);
				}
				System.out.println("");
			}
			System.out.println("");
		}
		// Scanner input shall be closed
		input.close();
		for ( i = 0; i < 3; i++) {
			for ( j = 0; j < 3; j++) {
				for ( k = 0; k < 3; k++) {
					multiplied[i][j] += matrix1[i][k] * matrix2[k][j];
				}
			}
		}
		// Multiply Matrices
		System.out.println("Answer is: ");
		for ( p = 0; p < 3; p++) {
			for ( q = 0; q < 3; q++) {
				System.out.printf("%.1f ",multiplied[p][q]);
			}
			System.out.println("");
		}
	}// end of main method
}// end of class MatrixMultiply
\end{lstlisting}

The following code computes multiplication of two \texttt{n}-by-\texttt{n} matrices.

\lstset{language=Java,tabsize=2}
\begin{lstlisting}
import java.util.Scanner;
public class MatrixMultiplyGeneral {
	public static void main(String[] args) {
		// Declaration of Variables
		int i; // counter
		int j; // counter
		int p; // counter
		int q; // counter
		int k;
		int row; // row of matrix
		int col; // column of matrix
		String str = "";
		double matrix1[][];
		double matrix2[][];
		double multiplied[][];
		int matrixSizes[][] = new int[2][2]; // contains number of rows and columns of matrix1 and matrix2
		Scanner input = new Scanner(System.in);
		// ask for dimensions of two matrices
		for (i = 0; i < 2; i++) {
			for (j = 0; j < 2; j++) {
				switch (j) {
					case 0: str = "row"; break;
					case 1: str = "column"; break;
				}
				System.out.print("Enter number of "+ str +"s of matrix "+ (i+1) + ": " );
				matrixSizes[i][j] = input.nextInt();
			}
		}
		// initialize matrices according to their number of rows and columns
		matrix1 = new double[matrixSizes[0][0]][matrixSizes[0][1]];
		matrix2 = new double[matrixSizes[1][0]][matrixSizes[1][1]];
		multiplied = new double[matrixSizes[0][0]][matrixSizes[1][1]];
		// ask for matrices
		for ( i = 0; i < 2; i++) {
			for ( j = 0; j < matrixSizes[i][0]*matrixSizes[i][1]; j++) {
				row = j / matrixSizes[i][0];
				col = j % matrixSizes[i][1];
				if (i == 0) {
					System.out.println("Enter matrix1[" + row + "][" + col + "]");
					matrix1[row][col] = input.nextInt();					
				}
				else {
					System.out.println("Enter matrix2[" + row + "][" + col + "]");
					matrix2[row][col] = input.nextInt();					
				}
			}
			System.out.println("Thank you. Matrix initialized as follows:");
			for ( p = 0; p < matrixSizes[i][0]; p++) {
				for ( q = 0; q < matrixSizes[i][1]; q++) {
					System.out.printf("%.1f ",matrix1[p][q]);
				}
				System.out.println("");
			}
		}
		// Scanner input shall be closed
		input.close();
		// check if number of columns of first matrix matches number of rows of second matrix
		if (matrix1[0].length == matrix2.length) {
			for ( i = 0; i < matrix2[0].length; i++) {
				for ( j = 0; j < matrix1.length; j++) {
					for ( k = 0; k < matrix1[0].length; k++) {
						multiplied[i][j] += matrix1[i][k] * matrix2[k][j];
					}
				}
			}
		}
		else {
			System.out.println("Matrices Dimensions Mismatch.");
		}
		// Multiply Matrices
		System.out.println("Answer is: ");
		for ( p = 0; p < matrixSizes[0][0]; p++) {
			for ( q = 0; q < matrixSizes[1][1]; q++) {
				System.out.printf("%.1f ",multiplied[p][q]);
			}
			System.out.println("");
		}	
	}	
}
\end{lstlisting}