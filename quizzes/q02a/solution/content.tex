\section*{Question 1}
Write a program \texttt{PrimeRelative.java} that asks for \texttt{N} from the user and prints out an \texttt{N}-by-\texttt{N} table such that there is an \texttt{*} in row \texttt{i} and column \texttt{j}, if the greatest common divisor of \texttt{i} and \texttt{j} is 1, and a space in that position if otherwise.

\subsection*{Solution}
\lstset{language=Java,tabsize=2}
\begin{lstlisting}
// we use scanner to ask for number N
import java.util.Scanner;
public class PrimeRelative {
	public static void main(String[] args) 
	{
		int i; // 1st for loop (matrix row)
		int j; // 2nd for loop (matrix col)
		int k; // 3rd for loop (division checker)
		int numberN;
		boolean notPrimeRelative;
		System.out.print("Please insert the number N: ");
		Scanner input = new Scanner(System.in);
		numberN = input.nextInt();
		System.out.println("Your entered number is " + numberN + ".");
		input.close(); // remember to close the scanner object
		numberN++; // to compensate our sacrifice of first row and column
		for (i = 0 ; i < numberN ; i++) {
			for (j = 0 ; j < numberN ; j++) {
				// if first row, print number of columns
				if (i == 0) {
					System.out.printf("%-2d ",j);
				}
				// else if first column, print number of rows
				else if (j == 0) {
					System.out.printf("%-2d ",i);
				}
				// for matrix elements
				else {
					// we initially assume gcd(i,j)=1
					notPrimeRelative = false;
					// now we check if a number k can be found
					// that divides both i and j
					for (k = 2; k <= Math.min(i, j); k++) {
						// if it divides both i and j
						// gcd(i,j) != 1 for sure
						if (i % k == 0 && j % k == 0) {
							// so they are not prime relative
							notPrimeRelative = true;
						}
					}
					if (notPrimeRelative) {
						System.out.print("   ");
					}
					else {
						System.out.print("*  ");
					}					
				}
			}
			// after each row, a feed line is inserted
			System.out.println();
		}
	}
}
\end{lstlisting}
\newpage
\section*{Question 2}
Write a program \texttt{MatrixDeterminant.java} that asks for elements of a three-by-three matrix and prints its determinant.

\subsection*{Solution}
\lstset{language=Java,tabsize=2}
\begin{lstlisting}
// import Scanner class to use Scanner
import java.util.Scanner;
public class MatrixDeterminant {
	public static void main(String[] args) {
		int i; // 1st loop (row)
		int j; // 2nd loop (col)
		int[][] matrix = new int[3][3]; // array of int for 3x3 matrix
		long determinant; // long is bigger than int
		Scanner input = new Scanner(System.in); // we create scanner object
		// ask user to initialize matrix
		for (i = 0; i < 3; i++) {
			for (j = 0; j < 3; j++) {
				System.out.printf("Insert element in row %d and col %d: ",i,j);
				matrix[i][j] = input.nextInt();
			}
		}
		// close Scanner object as we don't need it anymore
		input.close();
		// show initialized matrix to user
		System.out.println("Matrix is: ");
		for (i = 0; i < 3; i++) {
			for (j = 0; j < 3; j++) {
				System.out.printf("%5d ",matrix[i][j]);
			}
			System.out.println();
		}
		// show determinant
		System.out.print("Determinant for matrix is: ");
		determinant = matrix[0][0]*matrix[1][1]*matrix[2][2]
				+ matrix[1][0]*matrix[2][1]*matrix[1][2]
				+ matrix[0][1]*matrix[1][2]*matrix[2][0]
				- matrix[0][2]*matrix[1][1]*matrix[2][0]
				- matrix[0][0]*matrix[1][2]*matrix[2][1]
				- matrix[2][2]*matrix[1][0]*matrix[0][1];
		System.out.println(determinant);		
	}
}
\end{lstlisting}